\section*{MPI}
Message-passing programming paradigm has 2 key attributes:
\begin{enumerate}
\item partitioned address space (not shared), which implies
  \begin{itemize}
  \item each data element must belong to 1 partitioned space
  \item interactions (read-only/read-write) require coop of 2 procs
  \end{itemize}
\item supports only \emph{explicit} parallelization
\end{enumerate}
Two common causes of \textbf{deadlock} in MPI:
\begin{enumerate}
\item false order or \texttt{send} and \texttt{recv}
\item system send-buffer fill-up
\end{enumerate}

\subsection*{Starting and terminating MPI}
\begin{minipage}{0.5\linewidth}
  \begin{itemize}
  \item All the following snippets have a surrounding pair of \texttt{MPI\_Init} and \texttt{MPI\_Finalize}.
  \item After \texttt{MPI\_Finalize}, even another \texttt{MPI\_Init} is \emph{not} allowed.
  \item Both \texttt{Init} and \texttt{Finalize} must be called by all processes.
  \item In the end, \texttt{MPI\_SUCCESS} returned if nothing goes wrong.
  \end{itemize}
\end{minipage}
\begin{minipage}{0.48\linewidth}
  \centering
\begin{lstlisting}[language=c,xleftmargin=1pt]
// argc, argv from main
// No MPI calls can be here
MPI_Init(&argc, &argv);
// parallel computation
// .
// .
// .
// commu betwn procs
MPI_Finalize();
// No MPI calls can be here
\end{lstlisting}
\end{minipage}

\subsection*{Getting information}
\begin{minipage}{0.5\linewidth}
  \flushleft
  \begin{itemize}
  \item Each proc will call these two fns to get the info for comm.
  \item Output vars are \texttt{size} and \texttt{rank}.
  \item \texttt{size}/\texttt{rank} can be declared before \texttt{Init}.
  \item \textbf{in} args will be passed-by-value to MPI fns
  \item \textbf{out} args will be assigned values.
  \item \textbf{in}/\textbf{out} args must be declared before any usage.
  \end{itemize}
\end{minipage}
\begin{minipage}{0.48\linewidth}
  \centering
\begin{lstlisting}[language=c,xleftmargin=1pt,identifierstyle=]
// before any MPI comm
// get how many processors
int size; // or world_size
MPI_Comm_size(MPI_COMM_WORLD, &size);

// get the current proc rank
int myrank;
MPI_Comm_rank(MPI_COMM_WORLD, &myrank);
\end{lstlisting}
\end{minipage}

\subsection*{Sending and receiving messages (blocking commu)}
\begin{itemize}
\item commonly used for point-to-point communications.
\item blocking send blocks until msg gets copied out of the send buffer (either into a system buffer at src process or sent to dest process)
\item blocking recv returns only after msg gets received and copied into the recv buffer
\end{itemize}
\begin{lstlisting}[language=c]
// MPI defined const MPI_TAG_UB with guaranteed range [0,32767]
// data of type dt will be stored in `buf' (consecutive in mem)
// `buf' len is specified by `count'
int MPI_Send(void *buf,            /* in, data buffer  */
             int count,            /* in, data count   */
             MPI_Datatype dt,      /* in, data type    */
             int dest,             /* in, dest rank    */
             int tag,              /* in, msg type tag */
             MPI_Comm, comm        /* in, communicator */
);

// Recv can accept at most count data otherwise overflow
// with MPI_ERR_TRUNCATE.  So Recv needs not know the
// exact size of messages being sent
int MPI_Recv(void *buf,            /* out,data buffer  */
             int count,            /* in, data count   */
             MPI_Datatype dt,      /* in, data type    */
             int source,           /* in, source rank  */
             int tag,              /* in, msg type tag */
             MPI_Comm comm,        /* in, communicator */
             MPI_Status *status    /* out, recv status */
);
\end{lstlisting}
\subsection*{Non-blocking communication}
\begin{itemize}
\item commonly used to overlap communications with computation.
\item need appropriate hardware support.
\item \texttt{MPI\_Isend} returns before data copied out of the buffer.
\item \texttt{MPI\_Irecv} returns before data received and copied into the buffer.
\item later in the program, must check if send/recv is complete.
\item \texttt{MPI\_Irecv} not take \texttt{status} arg
\item \texttt{status} of recv is returned by \texttt{Test}/\texttt{Wait}
\item \texttt{req} (non-blocking) is different than \texttt{status} (blocking).
\item \texttt{req} is also returned by \texttt{MPI\_Test} and \texttt{MPI\_Wait}
\item \texttt{MPI\_Wait} until a non-blocking operation actually finishes.
\end{itemize}
\begin{lstlisting}[language=c]
// ISend returns immediately w/t blocking current proc
// Sending goes on in the background and later the proc
// must Wait/Test for completion of sending.  Only after
// that it's safe to reuse the `buf' (to overwrite it)
int MPI_Isend(void *buf,           /* in, data buffer  */
             int count,            /* in, data count   */
             MPI_Datatype dt,      /* in, data type    */
             int dest,             /* in, dest rank    */
             int tag,              /* in, msg type tag */
             MPI_Comm comm,        /* in, communicator */
             MPI_Request *req,     /* out,handler      */
);

// Irecv returns immediately w/t blocking current proc
// Receiving goes on in the background and later the proc
// must Wait/Test for completion of receiving.  Only after
// that it's safe to use received data (to compute)
// NOTE: no status arg needed for non-blocking Irecv
// status is returned by MPI_Test and MPI_Wait instead
int MPI_Irecv(void *buf,           /* out,data buffer  */
             int count,            /* in, data count   */
             MPI_Datatype dt,      /* in, data type    */
             int source,           /* in, source rank  */
             int tag,              /* in, msg type tag */
             MPI_Comm comm,        /* in, communicator */
             MPI_Request *req      /* out,handler      */
);
\end{lstlisting}

% \subsection*{MPI Test and Wait}
\begin{minipage}{0.5\linewidth}
\begin{lstlisting}[language=c,xrightmargin=2pt]
// test if a non-blocking
// operation has finished
// this fn is non-blocking
int MPI_Test(
 MPI_Request *request,/*in */
 int *flag,           /*out*/
 MPI_Status *status   /*out*/
);

// this fn is blocking
// it will return only when
// request indicates finish
int MPI_Wait(
 MPI_Request *request,/*in */
 MPI_Status *status   /*out*/
);
// Ok to Isend with blk Recv
\end{lstlisting}
\end{minipage}
\begin{minipage}{0.5\linewidth}
  \flushleft
  \begin{enumerate}[leftmargin=.8em]
  \item if a non-blocking op finished
    \begin{itemize}
    \item allowed periodical checking for completion
    \item \texttt{flag} set to \texttt{true} (non-zero in C)
    \item request object deallocated
    \item \texttt{*request} $\rightarrow$ \texttt{MPI\_REQUEST\_NULL}
    \item \texttt{status} object gets info about op
    \end{itemize}
  \item if a non-blocking op not finished
    \begin{itemize}
    \item \texttt{flag} set to \texttt{false} (zero in C)
    \item \texttt{*request} not modified
    \item \texttt{status} undefined
    \item \texttt{MPI\_Wait} blocks until \texttt{*request} indicates finish (see 1)
    \end{itemize}
  \item mix of blk \& non-blk is OK
  \end{enumerate}
\end{minipage}

\subsection*{Collective Communication \& Computation}
\begin{itemize}
\item Collective comm typically outperforms point-to-point comm
\item Code becomes more compact and easier to read
\item All collective comm fns in MPI take a communicator arg that defines the group of procs participating in the collective operation
\item All the procs belonging to this communicator participate in the op
\item All of them \textbf{must} call the collective comm fns (see below)
\item Collective comm ops \emph{do not} act like barriers (a proc could go past its call for the collective communication operation, even before other processes have reached it)
\item Collective comm ops acts like a \emph{virtual} sync step (no need for tags)
\item All procs in group (communicator) must specify same src-/target-proc
\item For most collective comm ops, MPI provides two different variants:
  \begin{enumerate}
  \item transfers equal-size data to or from each process
  \item transfers data that can be of different sizes
  \end{enumerate}
\end{itemize}

% Barrier
% \subsection*{Barrier}
\begin{minipage}{0.5\linewidth}
  \flushleft
  \begin{itemize}
  \item only arg is communicator
  \item returns only after all procs in group have called this fn
  \end{itemize}
\end{minipage}
\begin{minipage}{0.5\linewidth}
\begin{lstlisting}[language=c,xleftmargin=1pt]
// comm defines the group
// of procs that are synced
int MPI_Test(MPI_Comm comm);
\end{lstlisting}
\end{minipage}

% Broadcast
% \subsection*{Broadcast}
\begin{minipage}{0.5\linewidth}
  \flushleft
  \begin{itemize}
  \item only arg is communicator
  \item returns only after all procs in group have called this fn
  \item data stored in root (aka. src) buf
  \item data totaling count entries of \texttt{dt}
  \item root data amount \texttt{==} amount recv by each proc
  \item \texttt{count} and \texttt{dt} fields must match on all procs
  \end{itemize}
\end{minipage}
\begin{minipage}{0.5\linewidth}
\begin{lstlisting}[language=c,xleftmargin=1pt]
// for root, data goes out
// for others, data comes in
int MPI_Bcast(
  void *buf,       /*inout*/
  int count,       /*in   */
  MPI_Datatype dt, /*in   */
  int root,        /*in   */
  MPI_Comm comm    /*in   */
);
// each proc gets same data
\end{lstlisting}
\end{minipage}
