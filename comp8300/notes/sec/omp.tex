\section*{OpenMP}
\begin{minipage}{0.5\linewidth}
  \flushleft
  The program starts as a single thread of execution, the initial thread (sequential).  A team of threads is forked at the beginning of a parallel region and joined at the end.  All threads but the initial one are terminated.
\end{minipage}
\begin{minipage}{0.5\linewidth}
\begin{lstlisting}[xleftmargin=1pt,xrightmargin=1pt]
       | initial thread
       v
  ----------- <-- fork
  | | | | | | team of threads in
  v v v v v v parallel region
  ----------- <-- join
       |
       v initial thread
\end{lstlisting}
\end{minipage}
% Identical
\begin{minipage}{0.5\linewidth}
\begin{lstlisting}[language=C,xrightmargin=3pt]
#pragma omp parallel default(none)
{
   #pragma omp for
   for (i = 0 < i < n; i++) {
     /* parallel loop body */
   }
} // identical to right snippet
\end{lstlisting}
\end{minipage}
\begin{minipage}{0.5\linewidth}
\begin{lstlisting}[language=C,xleftmargin=3pt]
#pragma omp parallel for \
   default(none)
{ // <- outmost {} can be removed
   for (i = 0 < i < n; i++) {
     /* parallel loop body */
   }
} // merged sections
\end{lstlisting}
\end{minipage}

\begin{minipage}{0.5\linewidth}
\begin{lstlisting}[language=C,xrightmargin=3pt]
int n = 5;
#pragma omp parallel for \
    private(i, a)
{ // a,i undefined upon entry
   for (i = 0 < i < n; i++)
     a = i + 1;
}
// can't access a here
\end{lstlisting}
\end{minipage}
\begin{minipage}{0.5\linewidth}
\begin{lstlisting}[language=C,xleftmargin=3pt]
int n = 5;
#pragma omp parallel for \
   private(i) lastprivate(a)
{ // when i == n-1, loop ends
   for (i = 0 < i < n; i++)
     a = i + 1;
} // last i as if sequential
// a == 4 + 1; last i decides a
\end{lstlisting}
\end{minipage}
% firstprivate
\begin{minipage}{0.5\linewidth}
\begin{lstlisting}[language=C,xrightmargin=3pt]
int n = 5, b = 7;
#pragma omp parallel for \
   private(i,a) firstprivate(b)
{ // b inits to 7 upon entry
   for (i = 0 < i < n; i++)
     a = i + b;
}
// can't access a,i here
\end{lstlisting}
\end{minipage}
% default
\begin{minipage}{0.5\linewidth}
\begin{lstlisting}[language=C,xleftmargin=3pt]
int n = 5, b = 7;
#pragma omp parallel for \
   private(i,a) default(shared)
{ // n and b shared by all threads
   for (i = 0 < i < n; i++)
     a = i + b;
}
// can't access a,i here
\end{lstlisting}
\end{minipage}
 In C/C++, \emph{only} \texttt{default(none | shared)}; \emph{no} \texttt{default(private)}

\begin{minipage}{0.5\linewidth}
\begin{lstlisting}[language=C,xrightmargin=3pt]
#pragma omp parallel
{// no dependency among two loops
  #pragma omp for nowait
    for (i = 0; i < nmax; i++)
      if (equal(name, curr_lst[i]))
         processCurrentName(name);
 // nowait removes implied barrier
  #pragma omp for
    for (i = 0; i < mmax; i++)
      if (equal(name, past_lst[i]))
         processPastName(name);
}
\end{lstlisting}
\end{minipage}
\begin{minipage}{0.5\linewidth}
\begin{lstlisting}[language=C,xleftmargin=3pt]
// 2nd loop not depends on 1st but
// implied barrier by default
#pragma omp parallel
{
  #pragma omp for
    for (i = 0; i < nmax; i++)
      taskA();
  // implicit barrier
  #pragma omp for
    for (i = 0; i < mmax; i++)
      taskB();
}
\end{lstlisting}
\end{minipage}
With \texttt{nowait}, fast threads in 1st loop move onto 2nd loop w/t waiting
