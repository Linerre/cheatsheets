\section*{Avoiding Deadlocks}
In case of (blocking mutual) comm there 3 possibilities.
\subsection*{1. Both procs start with \texttt{send} followed by \texttt{recv}}
\begin{itemize}
\item For blocking non-buffered protocol, $P_0$ waits for matching \texttt{Recv} from $P_1$ (blocked by \texttt{Send}) $\rightarrow$ deadlock
\item For blocking buffered protocol, if msg size is smaller than sys buf, $P_0$ blocks only for the time data copied to sys buf
\item For blocking buffered protocol, large msg size will cause deadlock
\end{itemize}
\begin{minipage}{0.5\linewidth}
\begin{lstlisting}[language=C,xleftmargin=2pt]
// suppose 2 procs
// _not_ safe; may deadlock
if (rank==0) {
  Send(&a, 1, CHAR, 1);
  Recv(&b, 1, CHAR, 1, &st);
} else {
  Send(&b, 1, CHAR, 0);
  Recv(&a, 1, CHAR, 0, &st);
}
// swap send/recv order for
// one proc would work
\end{lstlisting}
\end{minipage}
\begin{minipage}{0.5\linewidth}
\begin{lstlisting}[language=C,xleftmargin=2pt]
// suppose 2 procs
// this is deadlock-free
if (rank==0) {
  Isend(&a, 1, CHAR, 1);
  Recv(&b, 1, CHAR, 1, &st);
  Wait(&request);
} else {
  Isend(&b, 1, CHAR, 0);
  Recv(&a, 1, CHAR, 0, &st);
  Wait(&request);
}
\end{lstlisting}
\end{minipage}

\subsection*{2. Both procs start with \texttt{recv} followed by \texttt{send}}
Ok to mix \texttt{Isend} with blocking \texttt{Recv} as shown in right snippet\\
\begin{minipage}{0.5\linewidth}
\begin{lstlisting}[language=C,xleftmargin=2pt]
// suppose 2 procs
// this always deadlock
if (rank==0) {
  Recv(&b, 1, CHAR, 1, &st);
  Send(&a, 1, CHAR, 1);
} else {
  Recv(&a, 1, CHAR, 0, &st);
  Send(&b, 1, CHAR, 0);
}
// regardless of whatever
\end{lstlisting}
\end{minipage}
\begin{minipage}{0.5\linewidth}
\begin{lstlisting}[language=C,xleftmargin=2pt]
// deadlock-free
if (rank==0) {
  Irecv(&b, 1, CHAR, 1, &st);
  Send(&a, 1, CHAR, 1);
  Wait(&request);
} else {
  Irecv(&a, 1, CHAR, 0, &st);
  Send(&b, 1, CHAR, 0);
  Wait(&request);
}
\end{lstlisting}
\end{minipage}

% \section*{A common ring program}
Write a program in which process $i$ sends a greeting to process $(i+1)\%p$.  This is a classic ring program
\begin{lstlisting}[language=c]
// i sends msg to i + 1 but last p sends to first
next_rank = (rank + 1) % p;
if (rank != 0) {
  // Receive a message from the previous process
  Recv(greeting, n, CHAR, rank - 1, 0);
} else {
  // Process 0 starts the chain
  snprintf(greeting, n, "Hello from proc %d", rank);
}

// Send the message to the next process
Send(greeting, strlen(greeting) + 1, CHAR, next_rank, 0);

// P0 receives msg from last proc to complete the chain
if (rank == 0) {
   Recv(greeting, MAX_STRING, CHAR, world_size - 1, 0);
}
\end{lstlisting}

\subsection*{3. One proc starts with \texttt{send} followed by \texttt{recv}, another vica versa}
proc $i$ sends a msg to proc $i+1$ (modulo the number of processes) and receives a msg from proc $i-1$ (module the number of processes)
\begin{lstlisting}[language=C,xleftmargin=2pt,xrightmargin=0pt]
// code is _safe_ and deadlock-free
int a[10], b[10], npes, myrank;
MPI_Status status;
// find out # of procs and my rank
MPI_Comm_size(MPI_COMM_WORLD, &npes);
MPI_Comm_rank(MPI_COMM_WORLD, &myrank);
if (myrank%2 == 1) {
  // odd proc i sends 10 ints to proc i + 1
  // then recvs 10 ints from proc i -1
  MPI_Send(a,10,INT,(myrank+1)%npes,1,MPI_COMM_WORLD);
  MPI_Recv(b,10,INT,(myrank-1+npes)%npes,1,MPI_COMM_WORLD);
}
else {
  // even proc i recvs 10 ints from proc i - 1
  // then sends 10 ints to proc i + 1
  MPI_Recv(b,10,INT,(myrank-1+npes)%npes,1,MPI_COMM_WORLD);
  MPI_Send(a,10,INT,(myrank+1)%npes,1,MPI_COMM_WORLD);
}
\end{lstlisting}
When \texttt{Send} is implemented using buffering, the code works correctly, since every call to \texttt{Send} will get buffered, allowing \texttt{Recv} to be performed. However, if \texttt{Send} blocks until the matching receive issued $\rightarrow$ deadlock (even for only two processes). \textbf{Unsafe version}:
\begin{lstlisting}[language=C,xleftmargin=2pt,xrightmargin=0pt]
int a[10], b[10], npes, myrank;
MPI_Status status;
MPI_Comm_size(MPI_COMM_WORLD, &npes);
MPI_Comm_rank(MPI_COMM_WORLD, &myrank);
MPI_Send(a,10,INT,(myrank+1)%npes,1,MPI_COMM_WORLD);
MPI_Recv(b,10,INT,(myrank-1+npes)%npes,1,MPI_COMM_WORLD);
\end{lstlisting}
% Status and Probe
\subsection*{Dynamic message size with \texttt{MPI\_Status} and \texttt{MPI\_Probe}}
\begin{minipage}{0.5\linewidth}
  \flushleft
  \begin{itemize}
  \item \texttt{MPI\_SOURCE} is rank of sender
  \item \texttt{MPI\_TAG} is tag of message
  \item \texttt{MPI\_ERROR} stores error-code of received message
  \item If \texttt{MPI\_Recv} gets \texttt{MPI\_ANY\_SOURCE} and \texttt{MPI\_ANY\_TAG}, \texttt{status} is only way to find out sender/msg tag
  \end{itemize}
\end{minipage}
\begin{minipage}{0.5\linewidth}
\begin{lstlisting}[language=C,xleftmargin=0pt]
// *status when declared
// &status when passed as arg
typedef struct MPI_Status {
  int MPI_SOURCE;
  int MPI_TAG;
  int MPI_ERROR;
};
\end{lstlisting}
\end{minipage}
\textbf{No} need to check status if src, tag, and $\#$ of received elements are known\\
Saves memory in user program and allows optimizations in MPI library

% MPI_Get_count and Probe
\begin{minipage}{0.5\linewidth}
\begin{lstlisting}[language=C,xleftmargin=1pt,xrightmargin=5pt]
MPI_Get_count(
  MPI_Status* status, // in
  MPI_Datatype dt,    // in
  int* count          // out
);
MPI_Probe(
  int source,        // in
  int tag,           // in
  MPI_Comm comm,     // in
  MPI_Status* status // out
);
\end{lstlisting}
\end{minipage}
\begin{minipage}{0.5\linewidth}
  \flushleft
  \begin{itemize}
  \item user passed \texttt{MPI\_Status} and \texttt{dt}
  \item \texttt{count} is the total number of \texttt{dt} elements received
  \item this fn is used to determine actual receive amount
  \item query msg size before actually recv it
  \item will block for a msg with a matching tag and sender
  \item \texttt{MPI\_Probe} \emph{not} actually recv msg
  \end{itemize}
\end{minipage}

\subsubsection*{Sending and Receiving Messages Simultaneously}
\begin{itemize}
\item \texttt{MPI\_Sendrecv} does \emph{not} suffer from circular deadlock problems (left, 3)
\item \texttt{MPI\_Sendrecv} allows data to travel for both send and recv
\item \texttt{MPI\_Sendrecv} args combines those of \texttt{MPI\_Send} and \texttt{MPI\_Recv}
\item send and receive buffers must be disjoint (different in memory)
\item source and dest of messages can be the same or different
\item \textbf{disjoint} buffers may force the use a temporary buffer and increase memory usage and the overall run time due to extra copy
\end{itemize}
\begin{minipage}{0.5\linewidth}
\begin{lstlisting}[language=C,xleftmargin=0pt,xrightmargin=1pt]
int MPI_Sendrecv(
  void *sendbuf,      // in
  int sendcount,      // in
  MPI_Datatype senddt,// in
  int dest,           // in
  int sendtag,        // in
  void *recvbuf,      // out
  int recvcount,      // in
  MPI_Datatype recvdt,// in
  int source,         // in
  int recvtag,        // in
  MPI_Comm comm,      // in
  MPI_Status *status  // out
);
\end{lstlisting}
\end{minipage}
\begin{minipage}{0.5\linewidth}
\begin{lstlisting}[language=C,xleftmargin=2pt,xrightmargin=0pt]
// Send and Recv use same buf
// data must be of same type
int MPI_Sendrecv_replace(
  void *buf,         // inout
  int count,         // in
  MPI_Datatype dt,   // in
  int dest,          // in
  int sendtag,       // in
  int source,        // in
  int recvtag,       // in
  MPI_Comm comm,     // in
  MPI_Status *status // out
  );
// recved replaces sent
\end{lstlisting}
\end{minipage}
\subsection*{Coping with boundaries}
\begin{lstlisting}[language=C]
// when boundary procs need NOT to send or recv:
if (myid == 0)
  src = MPI_PROC_NULL; // to make send/recv dummy
if (myid == ntasks - 1)
  dst = MPI_PROC_NULL;
MPI_Send(msg, msgsize, MPI_INTEGER, dst, tag, ...);
MPI_Recv(msg, msgsize, MPI_INTEGER, src, tag, ...);
\end{lstlisting}

% Drop if-else when possible
\textbf{Safe version} of the left example using \texttt{MPI\_Sendrecv}:
\begin{lstlisting}[language=C,]
// common practice: place MPI calls outside if-else when possible
int a[10], b[10], npes, myrank;
MPI_Status status;
MPI_Comm_size(MPI_COMM_WORLD, &npes);
MPI_Comm_rank(MPI_COMM_WORLD, &myrank);
MPI_SendRecv(a, 10, MPI_INT, (myrank+1)%npes, 1,
             b, 10, MPI_INT, (myrank-1+npes)%npes, 1,
             MPI_COMM_WORLD, &status);
\end{lstlisting}

\subsection*{Root rank (\texttt{== 0})}
\begin{lstlisting}[language=C]
// rank 0 (aka. root) often chosen for special tasks
// in this case it acts like MPI_Bcast in an naive way
// in practice, MPI_Bcast uses a faster tree algorithm
if (0 == myid)
  for (int i=1; i < ntasks; i++)
    MPI_Send(&data, 1, MPI_INT, i, 22, MPI_COMM_WORLD);
else
   MPI_Recv(&data, 1, MPI_INT, 0, 22, MPI_COMM_WORLD, &status);
\end{lstlisting}
\subsection*{Arbitrary recvs}
\begin{lstlisting}[language=C]
// ANY_SOURCE and ANY_TAG often intro perf overhead
// use only for clear benefit like load balancing
if (0 == myid) {
  for (int i=1; i < ntasks; i++) {
    MPI_Recv(&data, process(data) 1, MPI_INT,
             MPI_ANY_SOURCE, 22, MPI_COMM_WORL);
    process(data);
  } else
    MPI_Send(&data, 1, MPI_INT, 0, 22, MPI_COMM_WORLD, &status);
}
\end{lstlisting}
