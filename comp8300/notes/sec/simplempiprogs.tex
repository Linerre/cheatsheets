\section*{Ring and Pingpong programs using MPI}
\begin{lstlisting}[language=C,xrightmargin=2pt]
// Process i sends a greeting to process (i+1)\%p
// Use modulo/module to wrap around (a ring)
next_rank = (rank + 1) % p;
if (rank != 0) {
  Recv(greeting, n, CHAR, rank - 1, 0);
} else { // Process 0 starts the chain
  snprintf(greeting, n, "Hello from proc %d", rank);
}
Send(greeting, strlen(greeting) + 1, CHAR, next_rank, 0);
// P0 receives msg from last proc to complete the chain
if (rank == 0)
   Recv(greeting, MAX_STRING, CHAR, world_size - 1, 0);
\end{lstlisting}
\begin{lstlisting}[language=C,xrightmargin=2pt]
int *buf = calloc(len, sizeof(int));
for (int i=0; i < msg_reps; i++) {
  double t1 = MPI_Wtime(); // Measure walltime in each loop
  if (rank == 0) {
    MPI_Send(buf, len, MPI_INTEGER, 1, 0, MPI_COMM_WORLD);
    MPI_Recv(buf, len, MPI_INTEGER, 1, 0, MPI_COMM_WORLD, MPI_STATUS_IGNORE);
  } else if (rank == 1) {
    MPI_Recv(buf, len, MPI_INTEGER, 0, 0, MPI_COMM_WORLD, MPI_STATUS_IGNORE);
    MPI_Send(buf, len, MPI_INTEGER, 0, 0, MPI_COMM_WORLD);
  }
  double t2 = MPI_Wtime() - t1;
}
free(buffer);
\end{lstlisting}
