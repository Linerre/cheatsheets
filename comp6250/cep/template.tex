% Template for my Career Episode Piece essays
% Copy the template for further modification each time

\documentclass[12pt,a4paper]{article}

% --------------- Packages ----------------
% Layout
\usepackage[
  bottom=3cm,
  left=3.5cm,
  right=3.5cm,
  top=1cm,
]{geometry}

% Colors
\usepackage[dvipsnames]{xcolor}
\definecolor{ghblue}{HTML}{0969da}

% Bibliography
\usepackage[
  backend=biber,
  style=bath,
  block=space,
  date=year,
  isbn=false,
]{biblatex}

% URLs
\usepackage{hyperref}
\hypersetup{
  colorlinks=true,
  linkcolor=Red,
  citecolor=Black,
  filecolor=Cyan,
  urlcolor=Magenta,
}

% --------------- Customization ----------------

% --------------- Bib resources ----------------
% Add one or multiple bib files, each file on a single line
\addbibresource{refs/wiki.bib}
\addbibresource{refs/article.bib}
\addbibresource{refs/website.bib}

% Author, Title, Date
\title{Career Episode Piece X}  % Replace `X' with the appropriate number
\author{Zhiren Lin}
\date{\today}

% No indentation of paragraphs and 10pt vertical space between each
\setlength\parindent{0pt}
\setlength\parskip{10pt}


% --------------- Document Body ----------------
\begin{document}
\maketitle
This is a template for my career episode pieces assigned by \href{https://programsandcourses.anu.edu.au/2021/course/comp6250}{COMP6250}.

Despite its name, the degree of its `professional' is sometimes not as expected. For example,
it requires Harvard Referencing yet does \emph{not} clarify which flavor of Harvard References.

This causes confusion for sure, because ``there are many different versions of `Harverd' referencing'', pointed out by \textcite{site:ubathlib}.

Today I read an interesting article on `Game of Life' \parencite{gardner1970}.

According to \textcite{wiki:gol}, blah blah \ldots.

I also learned that \texttt{::=}\ is a syntax called Backus-Naur form \parencite{wiki:bnf}.
\printbibliography
\end{document}
