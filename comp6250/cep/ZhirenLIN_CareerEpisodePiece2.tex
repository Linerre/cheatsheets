\documentclass{cep}

\addbibresource{cep2.bib}

\begin{document}
% Copyright signature
\title{Career Episode Piece 2}
\author{Zhiren Lin}
\date{September 3rd}
\maketitle


The most enlightening moment I got from COMP6250’s week 5 lecture is how my misinterpretation of system per se and systems thinking as analytical thinking causes a good solution to backfire.

In libraries, a frequently mentioned term is “collection”: various shelves where library books are categorised according to Library of Congress Classification \parencites{site:loc}{wiki:loc}. When working at an academic library, although I was not so naive as to view it just as book collections, my understanding of it as a whole system was still rather superficial: a collection of involved parts: book collections, library staff, printers, just to name a few.

The key point I missed here is the interconnections and interactions between those components, as lecturer Nabavi \parentext{\citeyear{lec:week5}} explained in the lecture (by drawing a clear comparison between system and collection) that systems must contain “interrelated parts \ldots\ that form a complex and uniform whole” which also has a “specific purpose”.

From this system perspective, I can now see clearly why the library has three teams: Access, Reference, Instructional Technology \parencite{site:nyuteams}. It is exactly through these three types of services that the library connects its interrelated parts and thus makes itself a purposeful (mission) system \parencite{site:nyulibmission}. For example, a professor meeting a librarian for research consultation may lead to the librarian deciding to add several relevant books to the library’s collections, an acquisition process that usually requires the assistance from the Access team.

Misreading the concept of systems consequently gave me another false impression: “systemic” means “step-by-step” and thus equals “analytical”.  On the surface, it may seem to be my confusion of the words “systematic” (according to a system or method) and “systemic” (relating to or affecting an entire system) \parencite{mw:systematic}. However, after learning the difference between the two types of thinking in the lecture, I now think the root cause is my then mindset: the problem needed a technical solution only. I focused on the coding part (though it did help) so much that a shift of my goal went unnoticed.

Initially, the main goal of the whole project was to figure out a simple and quick solution to help the students/faculty access the library’s physical materials during lockdowns. Yet over time the focus was skewed and I ended up being obsessed with managing scanned PDF files using Google Drive/Sheet and a script bound to it. To my surprise, I observed that some colleagues were reluctant to use it to respond to users' requests, because they thought it entails too many steps and therefore is error-prone.

Back then, upon knowing the reason behind their unwillingness, I felt frustrated. It seemed the more effort I had put into it, the less useful it became! It was until in the lecture that I fully realised why this counterintuitive backfiring happened: I lack systems thinking and have long been mistaking it for analytical thinking. “just fixing problems won’t necessarily lead to improved performance”, as rightly pointed out by lecturer Nabavi \parentext{\citeyear{lec:week5}}. Besides, after analytically thinking, I did not put it back into the whole library system.

Looking back, due to the above misinterpretation, I overlooked the social systems in the project. In a narrow sense, the social system was just the university’s division of libraries (staff, librarians, students, faculty, books, buildings, devices, etc). In a broad sense, it also included publishers, authors, copyright laws and so on. I recall not long after the end of the project, I came across an interesting podcast episode titled “The E-book Wars” \parencite{site:npr}, which depicts exactly this broad social system I kept failing to recognize. In an internal meeting in around the same period, the library leadership emphasised that as the pandemic was about to end, the development or use of similar projects should cease. As I now understand it, when making this decision, they clearly had taken into consideration the broad social system.

Now that I have put straight my understanding of systems thinking, the next important step is to try applying systems thinking to my current and future complex projects. A good starting point may be picking up the Unified Modelling Language \parencite{wiki:uml} and first visualising the complex system(s). I should probably not expect a comprehensive description of a system from the outset, as it may well change and evolve. Instead, I should from time to time refer back to such visualisation of system(s) to review it, refine it, and once done with my analytical thinking, restore the systems thinking.

\printbibliography
\end{document}
