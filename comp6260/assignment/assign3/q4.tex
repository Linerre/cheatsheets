\textbf{Solution}.
The grammar can be expressed using the following productions:
\begin{enumerate}[noitemsep]
\item \(S \rightarrow S\; \cpp\; S\)
\item \(S \rightarrow O\)
\item \(O \rightarrow \texttt{ostream}\)
\item \(O \rightarrow \texttt{string}\)
\item \(O \rightarrow \texttt{int}\)
\end{enumerate}
\subsection{Leftmost derivation One}
% TODO: figure out how to use the format in automata theory 3rd, pp178
\begin{align*}
S & \Rightarrow S\; \cpp\; S && \text{Production 1}  \\
  & \Rightarrow O\; \cpp\; S && \text{Production 2} \\
  & \Rightarrow \texttt{ostream}\; \cpp\; S\; && \text{Production 3} \\
  & \Rightarrow \texttt{ostream}\; \cpp\; S\; \cpp\; S  && \text{Production 1} \\
  & \Rightarrow \texttt{ostream}\; \cpp\; O\; \cpp\; S  && \text{Production 2} \\
  & \Rightarrow \texttt{ostream}\; \cpp\; \texttt{string}\; \cpp\; S  && \text{Production 4} \\
  & \Rightarrow \texttt{ostream}\; \cpp\; \texttt{string}\; \cpp\; O && \text{Production 2} \\
  & \Rightarrow \texttt{ostream}\; \cpp\; \texttt{string}\; \cpp\; \texttt{int} && \text{Production 5}
\end{align*}
\subsection{Leftmost derivation Two}
\begin{align*}
S & \Rightarrow S\; \cpp\; S && \text{Production 1}  \\
  & \Rightarrow S\; \cpp\; S\; \cpp\; S && \text{Production 1} \\
  & \Rightarrow O\; \cpp\; S\; \cpp\; S && \text{Production 2} \\
  & \Rightarrow O\; \cpp\; O\; \cpp\; S && \text{Production 2} \\
  & \Rightarrow O\; \cpp\; O\; \cpp\; O && \text{Production 2} \\
  & \Rightarrow \texttt{ostream}\; \cpp\; O\; \cpp\; O  && \text{Production 3} \\
  & \Rightarrow \texttt{ostream}\; \cpp\; \texttt{string}\; \cpp\; O  && \text{Production 4} \\
  & \Rightarrow \texttt{ostream}\; \cpp\; \texttt{string}\; \cpp\; \texttt{int}  && \text{Production 5}
\end{align*}
\subsection{Associative}
\textbf{Solution}.
The binary operator \cpp should be left associative, i.e
\[
\texttt{ostream << string << int} = \texttt{(ostream << string) << int}
\]
The reasons I can think of are:
\begin{enumerate}
\item It is \emph{intuitive}, the execution order is the same as that of writing, from left to right
\item \cpp outputs its right argument as a side effect, hence right-associativity will cause its right argument to be output first, which is \emph{not} the desired result.  For example:
  \begin{enumerate}
  \item ``\texttt{This is a short line} \verb|\n|'' will output each word in a reverse order; The linebreak character ``\verb|\n|'' will be output first
  \item ``\texttt{3}'' \cpp ``\texttt{3}'' \cpp ``\texttt{1}'' \cpp ``\texttt{=}'' \cpp ``\texttt{4}'' may first output \texttt{ 1 = 4} (very confusing) if the outputing is done on a quite slow machine
  \end{enumerate}
\end{enumerate}
\subsection{Unambiguous}
\textbf{Solution}.\\
Add a new variable \emph{term} (denoted as \(T\)) to the original grammar.  \(T\) is an expression that cannot be broken apart by any \cpp on its right side.

With \(T\), the grammar now becomes:
\begin{itemize}[itemsep=0pt]
\item[] \(O\; \rightarrow\; \texttt{ostream}\; |\; \texttt{string}\; |\; \texttt{int} \)
\item[] \(T\; \rightarrow\; O \;|\; T\; \cpp\; O\)
\item[] \(S\; \rightarrow\; T \;|\; T\; \cpp\; S\)
\end{itemize}
The fact that grammar is unambiguous can be observed from the parse trees of \(T\) and \(S\) respectively:
\begin{figure}[t!]
  \centering
  % Parse tree of T
  \subcaptionbox{Parse tree of \(T\)}
  [0.4\linewidth]{
  \begin{forest}
      [T
        [T
        [T
        [T
        [\dots
        [,phantom]
        [T
        [T [O]]
        [\cpp]
        [O]]
        ]]
        [\cpp]
        [O]]
        [\cpp]
        [O]]
        [O]
      ]
    \end{forest}
  }
  % Parse tree of S
  \subcaptionbox{Parse tree of \(S\)}
  [0.4\linewidth]{
    \begin{forest}
      [S
        [T [O]]
        [S
        [T [O]]
        [\cpp]
        [S
        [T [O]]
        [\cpp]
        [S
        [\dots
        [,phantom]
        [S
        [T [O]]
        [\cpp]
        [T [O]]]
        ]]]]
      ]
      \end{forest}
  }
  \caption{Parse trees}
\end{figure}
