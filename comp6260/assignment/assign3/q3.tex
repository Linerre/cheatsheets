\subsection{\(\epsilon\)-closure}
\textbf{Solution.}
% step 1 find eclose of all states
\begin{itemize}
\item ECLOSE\((\s{0})\) = \(\{\s{0}, \s{1}, \s{4}, \s{2}\}\)
\item ECLOSE\((\s{1})\) = \(\{\s{1}, \s{4}, \s{2}, \s{0}\}\)
\item ECLOSE\((\s{2})\) = \(\{\s{2}\}\)
\item ECLOSE\((\s{3})\) = \(\{\s{3}\}\)
\item ECLOSE\((\s{4})\) = \(\{\s{4}, \s{2}, \s{0}, \s{1}\}\)
\end{itemize}
\subsection{Final NFA diagram}
\textbf{Solution.}\\
See diagram \ref{fig:nfa} on the next page.
% step 3 NFA
\clearpage
\begin{figure}[t]
  \centering
  \begin{tikzpicture}[bezier bounding box]% reduce vertical spacing after tikzpicture
  \node[state, initial, accepting] (s0) {\(\s{0}\)};
  \node[state, accepting, right of=s0] (s1) {\(\s{1}\)};
  \node[state, accepting, right of=s1] (s2) {\(\s{2}\)};
  \node[state, right of=s0, xshift=-1cm, yshift=-3cm] (s3) {\(\s{3}\)};
  \node[state, accepting, right of=s3] (s4) {\(\s{4}\)};

  \draw (s0) edge[loop above] node{\ntxt{0, 1}} (s0)
             edge[bend left, above] node{\ntxt{1}} (s2)
             edge[above] node[rotate=-60]{\ntxt{0, 1}} (s3)
             edge[out=-100, in=-120, looseness=1.2] node[below left, rotate=-25]{\ntxt{0}} (s4)
        (s1) edge[above] node{\ntxt{1}} (s2)
             edge[below, right=0.3] node{\ntxt{0}} (s4)
             edge[below, right=0.3] node[below, rotate=70]{\ntxt{0, 1}} (s3)
        (s2) edge[below, right=0.3] node{\ntxt{0}} (s4)
        (s3) edge[bend left, above] node{\ntxt{0}} (s4)
        (s4) edge[bend left] node[below]{\ntxt{1}} (s3)
             edge[out=-90, in=-140, looseness=1.8] node[below left, rotate=-25]{\ntxt{0, 1}} (s0);
  \end{tikzpicture}
  \caption{NFA without \(\epsilon\)}
  \label{fig:nfa}
\end{figure}
