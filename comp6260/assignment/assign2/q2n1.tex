Assume that $x = 6$ and $n = 4$, which satisfy the given condition $ x \geq n \;\land\; n \geq 0$.  It is then helpful to first take a look at what \texttt{Loop} does by listing the values of $y$, $m$ and $i$ in each \texttt{Loop} (which ends at $i = 4$):

\begin{table}[h]
  \centering
  \begin{tabular}{r||c|c|c}
    \hline
    $nth$ loop & y & m & i \\
    \hline
    0 & 1 & 6 & 0 \\
    1 & 6 & 5 & 1 \\
    2 & 30 & 4 & 2 \\
    3 & 120 & 3 & 3 \\
    4 & 360 & 2 & 4 \\
    \hline
  \end{tabular}
  \caption{Values in each loop}
  \label{tab:loop}
\end{table}

It is easy to see that in the $ith$ \texttt{Loop}, $y'$ is built up like:
\[
  y' =
  y_{initial} \;*\; m_{initial}\;*\;(m_{initial} - 1)\;*\;(m_{initial}-  2)\;*\;\ldots\;*\; (m_{initial} - i)
\]

That is, in any given $ith$ \texttt{Loop},  y$_{ith}$ value can be calculated as:
\begin{equation}
\label{eq:1}
y_{ith} = \frac{m_{initial}!}{(m_{initial} - i)!}
\end{equation}

The problem is, this equation requires $m > i$.  Yet as $m$ decreases by 1 and $i$ increases by 1 in each loop, $i$ will inevitably surpass $m$, thus making (\ref{eq:1}) invalid.  Table \ref{tab:loop} indicates the following is mathematically equivalent to (\ref{eq:1}):
\[
y_{ith} = \frac{(m_{ith}+i)!}{m_{ith}!}
\]
So one suitable \textbf{invariant} can be the following, regardless of the \texttt{Loop} times:
\begin{equation}
\label{eq:2}
  y = \frac{(m+i)!}{m!}
\end{equation}
