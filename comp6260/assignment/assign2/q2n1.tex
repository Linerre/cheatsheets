Assume that $x = 6$ and $n = 4$, which satisfy the given condition $ x \geq n \;\land\; n \geq 0$.  It is then helpful to first take a look at what \texttt{Loop} does by listing the values of $y$, $m$ and $i$ in each \texttt{Loop} (which ends at $i = 4$):

\begin{table}[h]
  \centering
  \begin{tabular}{r||c|c|c}
    \hline
    $nth$ loop & y & m & i \\
    \hline
    0 & 1 & 6 & 0 \\
    1 & 6 & 5 & 1 \\
    2 & 30 & 4 & 2 \\
    3 & 120 & 3 & 3 \\
    4 & 360 & 2 & 4 \\
    \hline
  \end{tabular}
  \caption{Values in each loop when \(x=6\;n=4\)}
  \label{tab:loop}
\end{table}

It is easy to see that in the $i$th \texttt{Loop}, $y'$ is built up like:
\[
  y' =
  y_{initial} \;*\; m_{initial}\;*\;(m_{initial} - 1)\;*\;(m_{initial}-  2)\;*\;\dotsb\;*\; (m_{initial} - i)
\]

That is, in any given $i$th \texttt{Loop},  the value of y$_{i}$ can be calculated as:
\begin{equation}
\label{eq:2.1.1}
y_{i} = \frac{m_{initial}!}{(m_{initial} - i)!}
\end{equation}
The problem is, this equation requires $m > i$.  Yet as $m$ decreases by 1 and $i$ increases by 1 in each \texttt{Loop}, $i$ will inevitably become equal to $m$, thus making \eqref{eq:2.1.1} invalid.  Given the initialization assignment \(m:=x\), the following is mathematically equivalent to \eqref{eq:2.1.1}:
\[
  y_{i} = \frac{(m_{i}+i)!}{m_{i}!} \quad \text{where}\quad m_{i}= x - i
\]
So one suitable \textbf{invariant} \(I\) can be the following:
\begin{equation}
\label{eq:2.1.2}
  y = \frac{(m+i)!}{m!}\:\land\:m=x-i\:\land\:0 \leq i \leq n\:\land\:n \geq 0\:\land\:x \geq n
\end{equation}
