\documentclass[a4paper,12pt]{article}
% Color
\usepackage[dvipsname]{xcolor}

% Math
\usepackage{mathtools}
\numberwithin{equation}{subsection}
\setlength{\jot}{8pt}% Add vertical space in align* env
\setlength{\abovedisplayskip}{0pt}%
\setlength{\belowdisplayskip}{0pt}%
\setlength{\abovedisplayshortskip}{0pt}%
\setlength{\belowdisplayshortskip}{0pt}%

% Layout
\usepackage[left=2cm]{geometry}

\begin{document}
\section{Solutions to Question 1}
% \input{assign2q1}

\section{Solutions to Question 2}
\subsection{Suitable Invariant}
\label{sec:2.1}
% Assume that $x = 6$ and $n = 4$, which satisfy the given condition $ x \geq n \;\land\; n \geq 0$.  It is then helpful to first take a look at what \texttt{Loop} does by listing the values of $y$, $m$ and $i$ in each \texttt{Loop} (which ends at $i = 4$):

\begin{table}[h]
  \centering
  \begin{tabular}{r||c|c|c}
    \hline
    $nth$ loop & y & m & i \\
    \hline
    0 & 1 & 6 & 0 \\
    1 & 6 & 5 & 1 \\
    2 & 30 & 4 & 2 \\
    3 & 120 & 3 & 3 \\
    4 & 360 & 2 & 4 \\
    \hline
  \end{tabular}
  \caption{Values in each loop when \(x=6\;n=4\)}
  \label{tab:loop}
\end{table}

It is easy to see that in the $i$th \texttt{Loop}, $y'$ is built up like:
\[
  y' =
  y_{initial} \;*\; m_{initial}\;*\;(m_{initial} - 1)\;*\;(m_{initial}-  2)\;*\;\dotsb\;*\; (m_{initial} - i)
\]

That is, in any given $i$th \texttt{Loop},  the value of y$_{i}$ can be calculated as:
\begin{equation}
\label{eq:2.1.1}
y_{i} = \frac{m_{initial}!}{(m_{initial} - i)!}
\end{equation}
The problem is, this equation requires $m > i$.  Yet as $m$ decreases by 1 and $i$ increases by 1 in each \texttt{Loop}, $i$ will inevitably become equal to $m$, thus making \eqref{eq:2.1.1} invalid.  Given the initialization assignment \(m:=x\), the following is mathematically equivalent to \eqref{eq:2.1.1}:
\[
  y_{i} = \frac{(m_{i}+i)!}{m_{i}!} \quad \text{where}\quad m_{i}= x - i
\]
So one suitable \textbf{invariant} \(I\) can be the following:
\begin{equation}
\label{eq:2.1.2}
  % y = \frac{(m+i)!}{m!}\:\land\:m=x-i\:\land\:0 \leq i \leq n\:\land\:n \geq 0\:\land\:x \geq n
  y = \frac{(m+i)!}{m!}\;\land\;m=x-i\;\land\;0 \leq i \leq n
\end{equation}
\newcommand{\R}{\(y=(m+i)!/m!\;\land\;m=x-i\;\land\; 0 \leq i \leq n\)}

\subsection{Proof for suitable invariant}

% TODO: give a tree-version of the following proof
Prove: \(\{I\,\land\,b\}\)\;\texttt{Body}\;\{I\} holds for each \texttt{Loop} where b is \(i < n\) and \(I\) is \(y=(m+i)!/m!\)
%TODO fix this align
\begin{flalign*}
\mathsf{1.}\: & \{y=(m+(i+1))!/m!\}\;\texttt{\textcolor{blue}{i:=i+1}}\;\{y = (m+i)!/m!\} & \mathsf{(Assignment)} \\
\mathsf{2.}\: & \{y=((m-1)+(i+1))!/(m-1)!\}\;\texttt{\textcolor{blue}{m:=m-1}}\;\{y=(m+(i+1))!/m!\} & \mathsf{(Assignment)} \\
\mathsf{3.}\: & \{y*m=((m-1)+(i+1))!/(m-1)!\}\;\texttt{\textcolor{blue}{y:=y*m}}\;\{y=((m-1)+(i+1))!/(m-1)!\}\ & \mathsf{(Assignment)}
\end{flalign*}
The precondition of the Hoare Triple 3 above can be simplified as follows:
\begin{align}
  y*m & =((m-1)+(i+1))!/(m-1)! \nonumber \\
  y*m & = (m+i)!/(m-1)! \nonumber \\
  y*\frac{m!}{(m-1)!} & = \frac{(m+i)!}{(m-1)!} \nonumber \\
  y & = \frac{(m+i)!}{m!} \label{eq:invar}
\end{align}
Therefore:
\begin{flalign*}
& \mathsf{4.}\: \{y*m=(m+i)!/m!\}\;\texttt{\textcolor{blue}{y:=y*m}}\;\{y=((m-1)+(i+1))!/(m-1)!\}\ && \mathsf{(3,\;Prec.\, Equiv.)} \\
& \mathsf{5.}\: \{y=(m+i)!/m!\}\;\texttt{\textcolor{blue}{y:=y*m;\:m:=m-1;\:i:=i+1;}}\;\{y=(m+i)!/m!\}\ && \mathsf{(1,2,4,\;Sequencing)}
\end{flalign*}
As long as \(b\) (i.e. \( i < n\)) holds, in each \texttt{Loop}, before and after \texttt{Body} executes, \(I\) holds, as proved above. So this establishes \(I\) in \eqref{eq:invar} as an invariant.

\subsection{Proof for Hoare Triple}
Following the above proof steps, I can further have:
\begin{flalign*}
\mathsf{1.}\; & \{y=(m+i)!/m!\;\land\;i < n\}\;\texttt{\textcolor{blue}{Body}}\;\{y=(m+i)!/m!\} & \mathsf{(Prec.\, Strengthen.)}  \\
\mathsf{2.}\; & \{y=(m+i)!/m!\;\land\;i < n\}\;\texttt{\textcolor{blue}{Loop}}\;\{y=x!/(x-n)!\} & \mathsf{(While.,\;Post.\, Weaken.)}
\end{flalign*}
Check \texttt{Init} establishes the invariant \(I\):\\
\begin{flalign*}
\mathsf{3.}\; & \{y=(x+0)!/x!\}\;\texttt{\textcolor{blue}{m:=x;\;y:=1;\;i:=0}}\;\{y=(x+0)!/x!\} & \mathsf{(Initialization)}
\end{flalign*}


\end{document}
