\documentclass[a4paper,12pt]{article}
% Layout
\usepackage[left=2cm, right=2cm]{geometry}

% No indentation
\setlength\parindent{0pt}

% Color
\usepackage[dvipsname]{xcolor}

% Math
\usepackage{mathtools}
\numberwithin{equation}{subsection}
\setlength{\jot}{8pt}% Add vertical space in align* env
\setlength{\abovedisplayskip}{0pt}%
\setlength{\belowdisplayskip}{0pt}%
\setlength{\abovedisplayshortskip}{0pt}%
\setlength{\belowdisplayshortskip}{0pt}%

% Proof
% \usepackage{bussproofs}
\usepackage{ebproof}

% Code
% verbatim always uses full-width (not-centering), and other verbatim
% packages do not address this problem specifically, so use listings.
\usepackage{listings}
\lstset{
  aboveskip=5pt,
  belowskip=5pt,
  basicstyle=\ttfamily,
  commentstyle=\upshape\ttfamily,
  columns=fullflexible,%remove space between letters
  keepspaces=true,
  language=Haskell,
  xleftmargin=4pt,
}
% I give up on centering code
% \usepackage{fancyvrb}

% ------------------ DOCUMENT ------------------
\begin{document}
\section{Solutions to Question 1}
\subsection{Lemma Equation}
% % Q1N1
If assume the list \texttt{xs} is empty \texttt{[]}, then \((\dagger)\) becomes:
\begin{lstlisting}
length([] ++ []) = length (dupAll([])) -- Assumption
                 = length []           -- D2
\end{lstlisting}
So it seems if I can have
\begin{lstlisting}
length([] ++ []) = length [] + length []
\end{lstlisting}
or more generally
\begin{lstlisting}
length(xs ++ ys) = length xs + length ys
\end{lstlisting}
I can use this lemma to complete the proof of \((\dagger)\).  Therefore, the needed lemma is
\[
  \forall\verb|xs|.\,\forall\verb|ys|.\:\:
 \verb|length (xs ++ ys) = length xs + length ys|
\]

\subsection{Proof of Lemma Equation}
% % Q1N2
I need to show that
\[
  \forall\verb|xs|.\,\forall\verb|ys|.\:\:
 \verb|length (xs ++ ys) = length xs + length ys|
\]
Let \texttt{ys} be an arbitrary list and treat \texttt{ys} as a constant and then I need to show that
\[
  \forall\verb|xs|.\:\:
 \verb|length (xs ++ ys) = length xs + length ys|
\]
\textbf{Base Case.} I need to show that
\[
 \verb|length ([] ++ ys) = length [] + length ys|
\]
\begin{lstlisting}
  length ([] ++ ys) = length ys                -- A1
                    = 0 + length ys            -- Arithmetic
                    = length [] + length ys    -- L1
\end{lstlisting}
\textbf{Step Case.} I need to show that
\[
  \forall\verb|xs|.\,\forall\verb|ys|.\:\:
 \verb|length (xs ++ ys) = length xs + length ys|
\]
Let \texttt{x} be an arbitrary list element and \texttt{xs}, an arbitrary non-empty list, and assume the inductive hypothesis as
\begin{equation*}
  \verb|length (xs + ys) = length xs ++ length ys| \tag{IH}
\end{equation*}
Then I need to show that
\begin{equation*}
  \verb|length ((x:xs) + ys) = length (x:xs) ++ length ys|
\end{equation*}
It is easier to prove this by starting with RHS:
\begin{lstlisting}
  length (x:xs) + length ys = 1 + length xs + length ys   -- L1
                            = 1 + length (xs ++ ys)       -- IH
                            = length (x:(xs ++ ys))       -- L2
                            = length ((x:xs) ++ ys)       -- A2
\end{lstlisting}
thus this finishes the proof and I now get
\begin{equation*}
  \forall\verb|xs|.\,\forall\verb|ys|.\:\:
  \verb|length (xs ++ ys) = length xs + length ys| \tag{\texttt{Lemma}}
\end{equation*}

\subsection{Proof of \((\dagger)\)}
% % Q1N3
I need to show that
\begin{equation*}
  \forall\verb|xs|.\:\:
  \verb|length (xs ++ xs) = length (dupAll xs)|
\end{equation*}
\textbf{Base Case.} When \texttt{xs} is an empty list \texttt{[]}, I need to show
\begin{equation*}
  \verb|length ([] ++ []) = length (dupAll [])|
\end{equation*}
This is easy to prove
\begin{lstlisting}
  length ([] ++ [])  = length [] + length []  -- Lemma
                     = 0 + 0                  -- L1
                     = 0                      -- Arithmetic
                     = length []              -- L1
                     = length (dupAll [])     -- D1
\end{lstlisting}
\textbf{Step Case.} Now let \texttt{x} be an arbitrary list element and \texttt{xs} an arbitrary list, and assume that
\begin{equation*}
  \verb|length (xs ++ xs) = length (dupAll xs)| \tag{IH}
\end{equation*}
I need to show that
\begin{equation*}
  \verb|length ((x:xs) ++ (x:xs)) = length (dupAll (x:xs))|
\end{equation*}
\begin{lstlisting}
  length (dupAll (x:xs)) = length (x:x:(dupAll xs))       -- D2
                         = 1 + length (x:(dupAll xs))     -- L2
                         = 1 + 1 + length (xs ++ xs)      -- IH
                         = 1 + 1 + length xs + length xs  -- Lemma
                         = 1 + length xs + 1 + length xs  -- Commutativity
                         = length (x:xs) + length (x:xs)  -- L2
                         = length ((x:xs) + (x:xs))       -- Lemma
\end{lstlisting}
which finishes the proof.

\section{Solutions to Question 2}
\subsection{Suitable Invariant}
\label{sec:2.1}
Assume that $x = 6$ and $n = 4$, which satisfy the given condition $ x \geq n \;\land\; n \geq 0$.  It is then helpful to first take a look at what \texttt{Loop} does by listing the values of $y$, $m$ and $i$ in each \texttt{Loop} (which ends at $i = 4$):

\begin{table}[h]
  \centering
  \begin{tabular}{r||c|c|c}
    \hline
    $nth$ loop & y & m & i \\
    \hline
    0 & 1 & 6 & 0 \\
    1 & 6 & 5 & 1 \\
    2 & 30 & 4 & 2 \\
    3 & 120 & 3 & 3 \\
    4 & 360 & 2 & 4 \\
    \hline
  \end{tabular}
  \caption{Values in each loop when \(x=6\;n=4\)}
  \label{tab:loop}
\end{table}

It is easy to see that in the $i$th \texttt{Loop}, $y'$ is built up like:
\[
  y' =
  y_{initial} \;*\; m_{initial}\;*\;(m_{initial} - 1)\;*\;(m_{initial}-  2)\;*\;\dotsb\;*\; (m_{initial} - i)
\]

That is, in any given $i$th \texttt{Loop},  the value of y$_{i}$ can be calculated as:
\begin{equation}
\label{eq:2.1.1}
y_{i} = \frac{m_{initial}!}{(m_{initial} - i)!}
\end{equation}
The problem is, this equation requires $m > i$.  Yet as $m$ decreases by 1 and $i$ increases by 1 in each \texttt{Loop}, $i$ will inevitably become equal to $m$, thus making \eqref{eq:2.1.1} invalid.  Given the initialization assignment \(m:=x\), the following is mathematically equivalent to \eqref{eq:2.1.1}:
\[
  y_{i} = \frac{(m_{i}+i)!}{m_{i}!} \quad \text{where}\quad m_{i}= x - i
\]
So one suitable \textbf{invariant} \(I\) can be the following:
\begin{equation}
\label{eq:2.1.2}
  % y = \frac{(m+i)!}{m!}\:\land\:m=x-i\:\land\:0 \leq i \leq n\:\land\:n \geq 0\:\land\:x \geq n
  y = \frac{(m+i)!}{m!}\;\land\;m=x-i\;\land\;0 \leq i \leq n
\end{equation}
\newcommand{\R}{\(y=(m+i)!/m!\;\land\;m=x-i\;\land\; 0 \leq i \leq n\)}

\subsection{Proof of suitable invariant}
\label{sec:2.2}
% TODO: give a tree-version of the following proof
Prove: \(\{I\,\land\,b\}\)\;\texttt{Body}\;\{I\} holds for each \texttt{Loop} where b is \(i < n\) and \(I\) is \(y=(m+i)!/m!\)
%TODO fix this align
\begin{flalign*}
\mathsf{1.}\: & \{y=(m+(i+1))!/m!\}\;\texttt{\textcolor{blue}{i:=i+1}}\;\{y = (m+i)!/m!\} & \mathsf{(Assignment)} \\
\mathsf{2.}\: & \{y=((m-1)+(i+1))!/(m-1)!\}\;\texttt{\textcolor{blue}{m:=m-1}}\;\{y=(m+(i+1))!/m!\} & \mathsf{(Assignment)} \\
\mathsf{3.}\: & \{y*m=((m-1)+(i+1))!/(m-1)!\}\;\texttt{\textcolor{blue}{y:=y*m}}\;\{y=((m-1)+(i+1))!/(m-1)!\}\ & \mathsf{(Assignment)}
\end{flalign*}
The precondition of the Hoare Triple 3 above can be simplified as follows:
\begin{align}
  y*m & =((m-1)+(i+1))!/(m-1)! \nonumber \\
  y*m & = (m+i)!/(m-1)! \nonumber \\
  y*\frac{m!}{(m-1)!} & = \frac{(m+i)!}{(m-1)!} \nonumber \\
  y & = \frac{(m+i)!}{m!} \label{eq:invar}
\end{align}
Therefore:
\begin{flalign*}
& \mathsf{4.}\: \{y*m=(m+i)!/m!\}\;\texttt{\textcolor{blue}{y:=y*m}}\;\{y=((m-1)+(i+1))!/(m-1)!\}\ && \mathsf{(3,\;Prec.\, Equiv.)} \\
& \mathsf{5.}\: \{y=(m+i)!/m!\}\;\texttt{\textcolor{blue}{y:=y*m;\:m:=m-1;\:i:=i+1;}}\;\{y=(m+i)!/m!\}\ && \mathsf{(1,2,4,\;Sequencing)}
\end{flalign*}
As long as \(b\) (i.e. \( i < n\)) holds, in each \texttt{Loop}, before and after \texttt{Body} executes, \(I\) holds, as proved above. So this establishes \(I\) in \eqref{eq:invar} as an invariant.

\subsection{Proof of Hoare Triple}
\label{sec:2.3}
% Following the above proof steps, now check that \texttt{Init} has \(I\) as the post-condition:

\newcommand{\MX}{\;\wedge\;m:=x-i}
\newcommand{\IN}{\;\wedge\;0 \leq i \leq n}
I now have:
\begin{displaymath}
\begin{prooftree}
  \hypo{\{I\;\wedge\;b\}\quad \texttt{Body}\quad \{I\}}
  \infer1{\{I\}\;\mathsf{while}\;b\;\mathsf{do}\;\texttt{Body}\;\{I\;\wedge\;\neg b\}}
\end{prooftree}
\end{displaymath}

\begin{flalign*}
& \mathsf{9.}\quad \neg(i < n)\; \rightarrow\; i=n && \mathsf{(Logic)} \\
& \mathsf{10.}\: \{y*m=(m+i)!/m!\MX\IN\;\wedge\; i=n\}\;\rightarrow \{y=x!/(x-n)!\} && \mathsf{(Post.\, Weakeaning)}
\end{flalign*}

Which finishes the proof.

\subsection{Loop Variant}
\subsection{Proof of Loop Variant}

\end{document}
% ------------------ DOCUMENT ------------------
