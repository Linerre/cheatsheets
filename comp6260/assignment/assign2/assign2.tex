\documentclass[a4paper,12pt]{article}
% Color
\usepackage[dvipsname]{xcolor}

% Math
\usepackage{mathtools}
\numberwithin{equation}{subsection}
\setlength{\jot}{8pt}% Add vertical space in align* env
\setlength{\abovedisplayskip}{0pt}%
\setlength{\belowdisplayskip}{0pt}%
\setlength{\abovedisplayshortskip}{0pt}%
\setlength{\belowdisplayshortskip}{0pt}%

% No indentation
\setlength\parindent{0pt}

% Proof
% \usepackage{bussproofs}
\usepackage{ebproof}

% Layout
\usepackage[left=2cm, right=2cm]{geometry}


\begin{document}
\section{Solutions to Question 1}
% \input{assign2q1}

\section{Solutions to Question 2}
\subsection{Suitable Invariant}
\label{sec:2.1}
% Assume that $x = 6$ and $n = 4$, which satisfy the given condition $ x \geq n \;\land\; n \geq 0$.  It is then helpful to first take a look at what \texttt{Loop} does by listing the values of $y$, $m$ and $i$ in each \texttt{Loop} (which ends at $i = 4$):

\begin{table}[h]
  \centering
  \begin{tabular}{r||c|c|c}
    \hline
    $nth$ loop & y & m & i \\
    \hline
    0 & 1 & 6 & 0 \\
    1 & 6 & 5 & 1 \\
    2 & 30 & 4 & 2 \\
    3 & 120 & 3 & 3 \\
    4 & 360 & 2 & 4 \\
    \hline
  \end{tabular}
  \caption{Values in each loop}
  \label{tab:loop}
\end{table}

It is easy to see that in the $ith$ \texttt{Loop}, $y'$ is built up like:
\[
  y' =
  y_{initial} \;*\; m_{initial}\;*\;(m_{initial} - 1)\;*\;(m_{initial}-  2)\;*\;\ldots\;*\; (m_{initial} - i)
\]

That is, in any given $ith$ \texttt{Loop},  y$_{ith}$ value can be calculated as:
\begin{equation}
\label{eq:1}
y_{ith} = \frac{m_{initial}!}{(m_{initial} - i)!}
\end{equation}

The problem is, this equation requires $m > i$.  Yet as $m$ decreases by 1 and $i$ increases by 1 in each loop, $i$ will inevitably surpass $m$, thus making (\ref{eq:1}) invalid.  Table \ref{tab:loop} indicates the following is mathematically equivalent to (\ref{eq:1}):
\[
y_{ith} = \frac{(m_{ith}+i)!}{m_{ith}!}
\]
So one suitable \textbf{invariant} can be the following, regardless of the \texttt{Loop} times:
\begin{equation}
\label{eq:2}
  y = \frac{(m+i)!}{m!}
\end{equation}

\subsection{Solution for suitable invariant}

% TODO: give a tree-version of the following proof
First, I need to ensure the invariant \(I\) \eqref{eq:2.1.2} is initially true when \texttt{Loop} is reached:
\begin{equation*}
\mathsf{1.}\: \{y=(m+0)!/m!\;\land\;m=x-0\;\land\;0 \leq i \leq n\;\land\;n \geq 0\;\land\;x \geq n\}\;\texttt{\textcolor{blue}{i:=0}}\;\{y=(m+i)!/m!\} \mathsf{(Assignment)}
\end{equation*}
\begin{equation*}
\mathsf{2.}\: \{1=(m+0)!/m!\;\land\;m=x-0\;\land\;0 \leq i \leq n\;\land\;n \geq 0\;\land\;x \geq n\}\;\texttt{\textcolor{blue}{y:=1}}\;\{y=(m+0)!/m!\;\land\;m=x-0\;\land\;0 \leq i \leq n\;\land\;n \geq 0\;\land\;x \geq n\} \mathsf{(Assignment)}
\end{equation*}
\begin{equation*}
\mathsf{3.}\: \{1=(x+0)!/x!\}\;\texttt{\textcolor{blue}{m:=x}}\;\{1=(m+0)!/m!\} \mathsf{(Assignment)}
\end{equation*}

% \mathsf{4.}\: \{1=(x+0)!/x!\}\;\texttt{\textcolor{blue}{m:=x;\;y:=1;\;i:=0;}}\;\{1=(m+0)!/m!\} \mathsf{(6,7,8,\;Sequencing)}
% \end{equation*}


Clearly, in \(\mathsf{9}\): \(1=(x+0)!/x!\; \equiv\; 1=1\; \equiv\; True\). Use Precondition Strengthening rule, I can have:
% Equation 2.2.1
\begin{equation}
\label{eq:2.2.1}
\{x \geq n\;\wedge\; n \geq 0\}\; \texttt{S}\; \{I\}
\end{equation}

Prove: \(\{I\,\land\,b\}\)\;\texttt{Body}\;\{I\} holds for each \texttt{Loop} where b is \(i < n\) and \(I\) is \(y=(m+i)!/m!\)
%TODO fix this align
\begin{flalign*}
\mathsf{1.}\: & \{y=(m+(i+1))!/m!\}\;\texttt{\textcolor{blue}{i:=i+1}}\;\{y = (m+i)!/m!\} & \mathsf{(Assignment)} \\
\mathsf{2.}\: & \{y=((m-1)+(i+1))!/(m-1)!\}\;\texttt{\textcolor{blue}{m:=m-1}}\;\{y=(m+(i+1))!/m!\} & \mathsf{(Assignment)} \\
\mathsf{3.}\: & \{y*m=((m-1)+(i+1))!/(m-1)!\}\;\texttt{\textcolor{blue}{y:=y*m}}\;\{y=((m-1)+(i+1))!/(m-1)!\}\ & \mathsf{(Assignment)}
\end{flalign*}
The precondition of the Hoare Triple 3 above can be simplified as follows:
\begin{align}
  y*m & =((m-1)+(i+1))!/(m-1)! \nonumber \\
  y*m & = (m+i)!/(m-1)! \nonumber \\
  y*\frac{m!}{(m-1)!} & = \frac{(m+i)!}{(m-1)!} \nonumber \\
  y & = \frac{(m+i)!}{m!} \label{eq:invar}
\end{align}
Therefore:
\begin{flalign*}
& \mathsf{4.}\: \{y*m=(m+i)!/m!\}\;\texttt{\textcolor{blue}{y:=y*m}}\;\{y=((m-1)+(i+1))!/(m-1)!\}\ && \mathsf{(3,\;Prec.\, Equiv.)} \\
& \mathsf{5.}\: \{y=(m+i)!/m!\}\;\texttt{\textcolor{blue}{y:=y*m;\:m:=m-1;\:i:=i+1;}}\;\{y=(m+i)!/m!\}\ && \mathsf{(1,2,4,\;Sequencing)}
\end{flalign*}
As long as \(b\) (i.e. \( i < n\)) holds, in each \texttt{Loop}, before and after \texttt{Body} executes, \(I\) holds, as proved above. So this establishes \(I\) in \eqref{eq:invar} as an invariant.

\subsection{Solution}
Following the above proof steps, now check that \texttt{Init} has \(I\) as the post-condition:
\begin{flalign*}
\mathsf{6.}\: & \{y=(m+0)!/m!\}\;\texttt{\textcolor{blue}{i:=0}}\;\{y=(m+i)!/m!\} & \mathsf{(Asst.\, Axiom)} \\
\mathsf{7.}\: & \{1=(m+0)!/m!\}\;\texttt{\textcolor{blue}{y:=1}}\;\{y=(m+0)!/m!\} & \mathsf{(Asst.\, Axiom)} \\
\mathsf{8.}\: & \{1=(x+0)!/x!\}\;\texttt{\textcolor{blue}{m:=x}}\;\{1=(m+0)!/m!\} & \mathsf{(Asst.\, Axiom)} \\
\mathsf{9.}\: & \{1=(x+0)!/x!\}\;\texttt{\textcolor{blue}{m:=x;\;y:=1;\;i:=0;}}\;\{1=(m+0)!/m!\} & \mathsf{(6,7,8,\;Sequencing)}
\end{flalign*}
Hence, before starting \texttt{Loop}, \(I,\;\text{i.e},\;y=(m+i)!/m!\) holds.  In addition, as \(i\) increases by 1 in each \texttt{Loop}, it will at some point becomes equal to \(n\) (i.e. \(i = n\)) and thus make \texttt{Loop} entry condition false.  So one now has:
\begin{table}[h]
  \centering
  \begin{tabular}{l@{\hspace{4pt}}l@{\hspace{4pt}}l}
  \(\{y=(m+i)!/m!\}\) & \(y:=y\,*\,m;\;m:=m-1;\;i:=i+1\) & \(\{y=(m+i)!/m!\}\)\\
  \noalign{\smallskip}
  \hline
  \noalign{\smallskip}
    \(\{y=(m+i)!/m!\}\) & \(\mathsf{while}\; (i < n)\; \mathsf{do}\) &\\
                      &\(\quad y:=y\,*\,m;\;m:=m-1;\;i:=i+1\) &\\
    & & \(\{(y=(m+i)!/m!)\;\land\;(i = n)\}\)
  \end{tabular}
  % \caption{}
  % \label{}
\end{table}

To complete the proof, one still needs to show that
\begin{equation}
\label{eq:6}
y=(m+i)!/m!\;\land\;i = n\;\rightarrow\;y=x!/(x-n)!\;\land\;i \geq n
\end{equation}

If one can show that
\begin{equation}
\label{eq:7}
y=(m+i)!/m!\;\land\;i=n\;\equiv\;y=x!/(x-i)!\;\land\;i=n
\end{equation}
holds during entire \texttt{Compute}, then using post-condition weakening rule, it is easy to get \eqref{eq:6}.

% TODO: re-write sentences to meet the proof order
Start with \(\neg(i < n)\), i.e. \(\neg\,b\):
\begin{flalign*}
\mathsf{10.}\: & y=(m+i)!/m!\;\land\;i = n\;\rightarrow\;i \geq n & \mathsf{(Post.\, Weaken.)} \\
\mathsf{11.}\: & y=(m+i)!/m!\;\land\;i \geq n\;\leftrightarrow\;\neg(i < n) & \mathsf{(Post.\, Equiv.)}
\end{flalign*}
Next, one needs to show
\[
y=(m+i)!/m!\;\land\;i=n\;\rightarrow\; y=x!/(x-n)!
\]

% TODO: clarify there
Since the proof for \(y=(m+i)!/m!\) has been given above, the equivalent proof is to show that \(m=x-i\) holds before and after each \texttt{Loop} and \texttt{Init} has its as the post-condition.

Assume \(m=x-i\), one can have:
\begin{flalign*}
\mathsf{12.}\: & \{y=x!/(x-(i+1))!\}\;\texttt{\textcolor{blue}{i:=i+1}}\;\{y=x!/(x-i)!\} & \mathsf{(Asst.\, Axiom)} \\
\mathsf{13.}\: & \{y=x!/(x-(i+1))!\}\;\texttt{\textcolor{blue}{m:=m-1}}\;\{y=x!/(x-(i+1))!\} & \mathsf{(Asst.\, Axiom)} \\
\mathsf{14.}\: & \{y=x!/(x-(i+1))!\}\;\texttt{\textcolor{blue}{(x-i):=(x-i)-1}}\;\{y=x!/(x-(i+1))!\} & \mathsf{(13, Premise\, Equiv.)} \\
\mathsf{15.}\: & \{y*m=x!/(x-(i+1))!\}\;\texttt{\textcolor{blue}{y:=y*m}}\;\{y=x!/(x-(i+1))!\}\ & \mathsf{(Asst.\, Axiom)} \\
\mathsf{16.}\: & \{y*(x-i)=x!/(x-(i+1))!\}\;\texttt{\textcolor{blue}{y:=y*(x-i)}}\;\{y=x!/(x-(i+1))!\}\ & \mathsf{(15, Premise\, Equiv.)}
\end{flalign*}
According to the precondition of \(\mathsf{16}\), one can have:
\begin{align*}
  y\;*\;(x-i) & = \frac{x!}{(x-(i+1))!}\\
  y\;*\*\frac{(x-i)!}{((x-i)-1)!} & = \frac{x!}{((x-i)-1)!}\\
  y & = \frac{x!}{(x-i)!}
\end{align*}
Therefore, one can have:
% TODO: figure out a better way to handle the \hspace
\begin{flalign*}
\mathsf{17.}\: & \{y*(x-i)=x!/(x-(i+1))!\}\;\texttt{\textcolor{blue}{y:=y*(x-i)}}\;\{y=x!/(x-(i+1))!\}\ & \mathsf{(15, Premise\, Equiv.)} \\
\mathsf{18.}\: & \{y=x!/(x-i)!\}\;\texttt{\textcolor{blue}{y:=y*(x-i);\;(x-i):=(x-i)-1;\;i:=i-1;}}\;\{y=x!/(x-i)!\}\ & \mathsf{(12,14,16,\,Seq)} \\
\mathsf{19.}\: & \{y=x!/(x-i)!\}\;\texttt{\textcolor{blue}{y:=y*m;\;m:=m-1;\;i:=i-1;}}\;\{y=x!/(x-i)!\}\ & \hspace{-40pt}\mathsf{(12,13,15,\,Seq.\,Premis\, Equiv.)}
\end{flalign*}
Next, one needs to check \texttt{Init} has the post-condition \(y=x!/(x-i)!\):
\begin{flalign*}
\mathsf{20.}\: & \{y=x!/(x-0)!\}\;\texttt{\textcolor{blue}{i:=0}}\;\{y=x!/(x-i)!\} & \mathsf{(Asst.\, Axiom)} \\
\mathsf{21.}\: & \{1=x!/(x-0)!\}\;\texttt{\textcolor{blue}{y:=1}}\;\{y=x!/(x-0)!\} & \mathsf{(Asst.\, Axiom)} \\
\mathsf{22.}\: & \{1=x!/(x-0)!\}\;\texttt{\textcolor{blue}{m:=x}}\;\{1=x!/(x-0)!\} & \mathsf{(Asst.\, Axiom)} \\
\mathsf{23.}\: & \{1=x!/(x-0)!\}\;\texttt{\textcolor{blue}{m:=x;\;y:=1;\;i:=0;}}\;\{1=x!/(x-0)!\} & \mathsf{(20,21,22,\;Sequencing)}
\end{flalign*}
Now one has:
\[
y=(m+i)!/m!\;\land\;m=x-i\;\equiv\; y=x!/(x-i)!
\]

\end{document}
