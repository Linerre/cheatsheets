Following the above proof steps, now check that \texttt{Init} has \(I\) as the post-condition:
\begin{flalign*}
\mathsf{6.}\: & \{y=(m+0)!/m!\}\;\texttt{\textcolor{blue}{i:=0}}\;\{y=(m+i)!/m!\} & \mathsf{(Assignment)} \\
\mathsf{7.}\: & \{1=(m+0)!/m!\}\;\texttt{\textcolor{blue}{y:=1}}\;\{y=(m+0)!/m!\} & \mathsf{(Assignment)} \\
\mathsf{8.}\: & \{1=(x+0)!/x!\}\;\texttt{\textcolor{blue}{m:=x}}\;\{1=(m+0)!/m!\} & \mathsf{(Assignment)} \\
\mathsf{9.}\: & \{1=(x+0)!/x!\}\;\texttt{\textcolor{blue}{m:=x;\;y:=1;\;i:=0;}}\;\{1=(m+0)!/m!\} & \mathsf{(6,7,8,\;Sequencing)}
\end{flalign*}


Clearly, in \(\mathsf{9}\): \(1=(x+0)!/x!\; \equiv\; 1=1\; \equiv\; True\). Use Precondition Strengthening rule, I can have:
% Equation 2.3.1
\begin{equation}
\label{eq:2.3.1}
\{x \geq n\;\wedge\; n \geq 0\}\; \texttt{S}\; \{I\}
\end{equation}


Hence, before starting \texttt{Loop}, \(I,\;\text{i.e},\;y=(m+i)!/m!\) holds.  In addition, as \(i\) increases by 1 in each \texttt{Loop}, it will at some point becomes equal to \(n\) (i.e. \(i = n\)) and thus make \texttt{Loop} entry condition false.  So I now have:
\begin{table}[h]
  \centering
  \begin{tabular}{l@{\hspace{4pt}}l@{\hspace{4pt}}l}
  \(\{y=(m+i)!/m!\}\) & \(y:=y\,*\,m;\;m:=m-1;\;i:=i+1\) & \(\{y=(m+i)!/m!\}\)\\
  \noalign{\smallskip}
  \hline
  \noalign{\smallskip}
    \(\{y=(m+i)!/m!\}\) & \(\mathsf{while}\; (i < n)\; \mathsf{do}\) &\\
                      &\(\quad y:=y\,*\,m;\;m:=m-1;\;i:=i+1\) &\\
    & & \(\{(y=(m+i)!/m!)\;\land\;(i = n)\}\)
  \end{tabular}
  % \caption{}
  % \label{}
\end{table}

To complete the proof, I still need to show that
\[
y=(m+i)!/m!\;\land\;i = n\;\rightarrow\;y=x!/(x-n)!\;\land\;i \geq n
\]
To prove it, first I need to show that
% Equation 2.3.2
\begin{equation}
\label{eq:2.3.2}
y=(m+i)!/m!\;\equiv\;y=x!/(x-i)!
\end{equation}
holds during entire \texttt{Compute}, then I can use Post-condition Weakening rule to finish the proof.

Since the proof for \(y=(m+i)!/m!\) has been given above, I need an equation that I can use to replace \(m\) with \(x\) at some point during my proof.  Given the \texttt{Init} and \texttt{Loop}, it seems that \(m=x-i\) is plausible.  So assume \(m=x-i\), I can have:
\begin{flalign*}
\mathsf{12.}\: & \{y=x!/(x-(i+1))!\}\;\texttt{\textcolor{blue}{i:=i+1}}\;\{y=x!/(x-i)!\} & \mathsf{(Assignment)} \\
\mathsf{13.}\: & \{y=x!/(x-(i+1))!\}\;\texttt{\textcolor{blue}{m:=m-1}}\;\{y=x!/(x-(i+1))!\} & \mathsf{(Assignment)} \\
\mathsf{14.}\: & \{y=x!/(x-(i+1))!\}\;\texttt{\textcolor{blue}{(x-i):=(x-i)-1}}\;\{y=x!/(x-(i+1))!\} & \mathsf{(13, Premise\, Equiv.)} \\
\mathsf{15.}\: & \{y*m=x!/(x-(i+1))!\}\;\texttt{\textcolor{blue}{y:=y*m}}\;\{y=x!/(x-(i+1))!\}\ & \mathsf{(Assignment)} \\
\mathsf{16.}\: & \{y*(x-i)=x!/(x-(i+1))!\}\;\texttt{\textcolor{blue}{y:=y*(x-i)}}\;\{y=x!/(x-(i+1))!\}\ & \mathsf{(15, Premise\, Equiv.)}
\end{flalign*}
According to the precondition of \(\mathsf{16}\), I have:
\begin{align*}
  y\;*\;(x-i) & = \frac{x!}{(x-(i+1))!}\\
  y\;*\*\frac{(x-i)!}{((x-i)-1)!} & = \frac{x!}{((x-i)-1)!}\\
  y & = \frac{x!}{(x-i)!}
\end{align*}
Therefore, I can have:
% TODO: figure out a better way to handle the \hspace
\begin{flalign*}
\mathsf{17.}\: & \{y=x!/(x-i)!\}\;\texttt{\textcolor{blue}{y:=y*(x-i);\,(x-i):=(x-i)-1;\,i:=i-1;}}\;\{y=x!/(x-i)!\}\ & \mathsf{(12,14,16,\,Seq.)} \\
\mathsf{18.}\: & \{y=x!/(x-i)!\}\;\texttt{\textcolor{blue}{y:=y*m;\,m:=m-1;\,i:=i-1;}}\;\{y=x!/(x-i)!\}\ & \hspace{-42pt}\mathsf{(12,13,15,\,Seq.\,Premise\, Equiv.)}
\end{flalign*}
Next, I need to check \texttt{Init} has the post-condition \(y=x!/(x-i)!\):
\begin{flalign*}
\mathsf{19.}\: & \{y=x!/(x-0)!\}\;\texttt{\textcolor{blue}{i:=0}}\;\{y=x!/(x-i)!\} & \mathsf{(Assignment)} \\
\mathsf{20.}\: & \{1=x!/(x-0)!\}\;\texttt{\textcolor{blue}{y:=1}}\;\{y=x!/(x-0)!\} & \mathsf{(Assignment)} \\
\mathsf{21.}\: & \{1=x!/(x-0)!\}\;\texttt{\textcolor{blue}{m:=x}}\;\{1=x!/(x-0)!\} & \mathsf{(Assignment)} \\
\mathsf{22.}\: & \{1=x!/(x-0)!\}\;\texttt{\textcolor{blue}{(x-i):=(x-i)}}\;\{1=x!/(x-0)!\} & \mathsf{(Premise\,Equiv.)} \\
\mathsf{23.}\: & \{1=x!/(x-0)!\}\;\texttt{\textcolor{blue}{m:=x;\;y:=1;\;i:=0;}}\;\{1=x!/(x-0)!\} & \mathsf{(20,21,22,\;Sequencing)}
\end{flalign*}
Thus, I proved \eqref{eq:2.3.2} holds throughout \texttt{Compute} and \texttt{Init}.  Clearly, in \(\mathsf{23}\), \(1=x!/(x-0)! \;\equiv\; 1=1\; \equiv\; True\).  Using Precondition Strengthening rule, I can have
% Equation 2.3.3
\begin{equation}
\label{eq:2.3.3}
\{x \geq n\;\wedge\; n \geq 0\}\; \texttt{S}\; \{y=x!/(x-i)!\}
\end{equation}
Now I have:
% Equation 2.3.4
\begin{equation}
\label{eq:2.3.4}
y=(m+i)!/m!\;\land\;m=x-i\;\equiv\; y=x!/(x-i)!
\end{equation}
So using Post-condition Weakening rule, I can have
% Equation 2.3.5
\begin{equation}
  \label{eq:2.3.5}
y=(m+i)!/m!\;\land\;i = n\;\rightarrow\;y=x!/(x-n)!\;\land\;i \geq n
\end{equation}
Therefor putting the above proof together I now have
\begin{table}[h]
  \centering
  \begin{tabular}{l@{\hspace{4pt}}l@{\hspace{4pt}}l}
  \(\{y=(m+i)!/m!\}\) & \(y:=y\,*\,m;\;m:=m-1;\;i:=i+1\) & \(\{y=(m+i)!/m!\}\)\\
  \noalign{\smallskip}
  \hline
  \noalign{\smallskip}
    \(\{y=(m+i)!/m!\}\) & \(\mathsf{while}\; (i < n)\; \mathsf{do}\) &\\
                        & \(\quad y:=y\,*\,m;\;m:=m-1;\;i:=i+1\)     &\\
    & & \(\{y=x!/(x-i!)\;\land\;\neg (i < n)\}\)
  \end{tabular}
\end{table}

Which can be simplified using the referencing letters/words as follows:
\begin{displaymath}
\begin{prooftree}
  \hypo{\{I\;\wedge\;b\}\quad \texttt{Body}\quad \{I\}}
  \infer1{\{I\}\;\mathsf{while}\;b\;\mathsf{do}\;\texttt{Body}\;\{I\;\wedge\;\neg b\}}
\end{prooftree}
\end{displaymath}
Putting together \eqref{eq:2.3.1}, \eqref{eq:2.3.3}, and \eqref{eq:2.3.5}, according to While rule, I now proved that:
\begin{equation*}
\{x \geq n\;\wedge\; n \geq 0\}\; \texttt{Compute}\; \{y=x!/(x-n)!\}
\end{equation*}
