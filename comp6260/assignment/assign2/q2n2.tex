% TODO: give a tree-version of the following proof
First, ensure the invariant \(I\) \eqref{eq:2.1.2} is initially true when \texttt{Loop} is reached:
\begin{equation*}
  \mathsf{1.}\; \{y=(m+0)!/m!\;\land\;m=x-0\;\land\;0 \leq 0 \leq n\}\; \texttt{\textcolor{blue}{i:=0}}\; \{y=(m+i)!/m!\;\land\;m=x-i\;\land\;0 \leq i \leq n\} \tag{\(\mathsf{Assignment}\)}
\end{equation*}
\begin{equation*}
  \mathsf{2.}\; \{1=(m+0)!/m!\;\land\;m=x-0\;\land\;0 \leq 0 \leq n\}\; \texttt{\textcolor{blue}{y:=1}}\;\{y=(m+0)!/m!\;\land\;m=x-0\;\land\;0 \leq i \leq n\} \tag{\(\mathsf{Assignment}\)}
\end{equation*}
\begin{equation*}
\mathsf{3.}\; \{1=(x+0)!/x!\;\land\;x=x-0\;\land\;0 \leq 0 \leq n\}\;\texttt{\textcolor{blue}{m:=x}}\;\{1=(m+0)!/m!\;\land\;m=x-0\;\land\;0 \leq 0 \leq n\} \tag{\(\mathsf{Assignment}\)}
\end{equation*}

% \mathsf{4.}\: \{1=(x+0)!/x!\}\;\texttt{\textcolor{blue}{m:=x;\;y:=1;\;i:=0;}}\;\{1=(m+0)!/m!\} \mathsf{(6,7,8,\;Sequencing)}
% \end{equation*}

% TODO: reword this part
Clearly, in \(\mathsf{3}\), I have
\begin{align*}
  1& =(x+0)!/x!\; \equiv\; 1=1\;  \equiv\; True \\
  x& =x-0      \; \equiv\; x=x\;  \equiv\; True \\
  0& \leq 0 \leq n\; \rightarrow\; 0 = 0\;\land 0 < n
\end{align*}
Use Precondition Strengthening rule, I can have:
% Equation 2.2.1
\begin{equation}
\label{eq:2.2.1}
\{x \geq n\;\wedge\; n \geq 0\}\; \texttt{Init}\; \{S\}\; \wedge\; S\; \rightarrow\; I
\end{equation}

Next, using Precondition Weakening I can further weaken \(I\) to \(I'\)
\[
I\; \rightarrow\; I'\; \equiv\; y=(m+i)!/m!
\]
Then if \(\{I'\,\land\,b\}\;\texttt{Body}\;\{I'\}\) holds for each \texttt{Loop} where b is \(i < n\) and \(I'\) is \(y=(m+i)!/m!\), then so does \(I\) by precondition strengthening rule.
%TODO fix this align
\begin{flalign*}
\mathsf{4.}\: & \{y=(m+(i+1))!/m!\}\;\texttt{\textcolor{blue}{i:=i+1}}\;\{y = (m+i)!/m!\} & \mathsf{(Assignment)} \\
\mathsf{5.}\: & \{y=((m-1)+(i+1))!/(m-1)!\}\;\texttt{\textcolor{blue}{m:=m-1}}\;\{y=(m+(i+1))!/m!\} & \mathsf{(Assignment)} \\
\mathsf{6.}\: & \{y*m=((m-1)+(i+1))!/(m-1)!\}\;\texttt{\textcolor{blue}{y:=y*m}}\;\{y=((m-1)+(i+1))!/(m-1)!\}\ & \mathsf{(Assignment)}
\end{flalign*}
The precondition of the Hoare Triple 3 above can be simplified as follows:
\begin{align}
  y*m & =((m-1)+(i+1))!/(m-1)! \nonumber \\
  y*m & = (m+i)!/(m-1)! \nonumber \\
  y*\frac{m!}{(m-1)!} & = \frac{(m+i)!}{(m-1)!} \nonumber \\
  y & = \frac{(m+i)!}{m!} \label{eq:invar}
\end{align}
Therefore:
\begin{flalign*}
& \mathsf{7.}\: \{y*m=(m+i)!/m!\}\;\texttt{\textcolor{blue}{y:=y*m}}\;\{y=((m-1)+(i+1))!/(m-1)!\}\ && \mathsf{(6,\;Prec.\, Equiv.)} \\
& \mathsf{8.}\: \{y=(m+i)!/m!\}\;\texttt{\textcolor{blue}{y:=y*m;\:m:=m-1;\:i:=i+1;}}\;\{y=(m+i)!/m!\}\ && \mathsf{(4,5,7,\;Sequencing)}
\end{flalign*}
So this proves that \(I\) in \eqref{eq:invar} as an invariant.
