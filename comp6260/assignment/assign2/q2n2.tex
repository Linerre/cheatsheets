% TODO: give a tree-version of the following proof
First, I need to ensure the invariant \(I\) \eqref{eq:2.1.2} is initially true when \texttt{Loop} is reached:
\begin{equation*}
\mathsf{1.}\: \{y=(m+0)!/m!\;\land\;m=x-0\;\land\;0 \leq i \leq n\;\land\;n \geq 0\;\land\;x \geq n\}\;\texttt{\textcolor{blue}{i:=0}}\;\{y=(m+i)!/m!\} \mathsf{(Assignment)}
\end{equation*}
\begin{equation*}
\mathsf{2.}\: \{1=(m+0)!/m!\;\land\;m=x-0\;\land\;0 \leq i \leq n\;\land\;n \geq 0\;\land\;x \geq n\}\;\texttt{\textcolor{blue}{y:=1}}\;\{y=(m+0)!/m!\;\land\;m=x-0\;\land\;0 \leq i \leq n\;\land\;n \geq 0\;\land\;x \geq n\} \mathsf{(Assignment)}
\end{equation*}
\begin{equation*}
\mathsf{3.}\: \{1=(x+0)!/x!\}\;\texttt{\textcolor{blue}{m:=x}}\;\{1=(m+0)!/m!\} \mathsf{(Assignment)}
\end{equation*}

% \mathsf{4.}\: \{1=(x+0)!/x!\}\;\texttt{\textcolor{blue}{m:=x;\;y:=1;\;i:=0;}}\;\{1=(m+0)!/m!\} \mathsf{(6,7,8,\;Sequencing)}
% \end{equation*}


Clearly, in \(\mathsf{9}\): \(1=(x+0)!/x!\; \equiv\; 1=1\; \equiv\; True\). Use Precondition Strengthening rule, I can have:
% Equation 2.2.1
\begin{equation}
\label{eq:2.2.1}
\{x \geq n\;\wedge\; n \geq 0\}\; \texttt{S}\; \{I\}
\end{equation}

Prove: \(\{I\,\land\,b\}\)\;\texttt{Body}\;\{I\} holds for each \texttt{Loop} where b is \(i < n\) and \(I\) is \(y=(m+i)!/m!\)
%TODO fix this align
\begin{flalign*}
\mathsf{1.}\: & \{y=(m+(i+1))!/m!\}\;\texttt{\textcolor{blue}{i:=i+1}}\;\{y = (m+i)!/m!\} & \mathsf{(Assignment)} \\
\mathsf{2.}\: & \{y=((m-1)+(i+1))!/(m-1)!\}\;\texttt{\textcolor{blue}{m:=m-1}}\;\{y=(m+(i+1))!/m!\} & \mathsf{(Assignment)} \\
\mathsf{3.}\: & \{y*m=((m-1)+(i+1))!/(m-1)!\}\;\texttt{\textcolor{blue}{y:=y*m}}\;\{y=((m-1)+(i+1))!/(m-1)!\}\ & \mathsf{(Assignment)}
\end{flalign*}
The precondition of the Hoare Triple 3 above can be simplified as follows:
\begin{align}
  y*m & =((m-1)+(i+1))!/(m-1)! \nonumber \\
  y*m & = (m+i)!/(m-1)! \nonumber \\
  y*\frac{m!}{(m-1)!} & = \frac{(m+i)!}{(m-1)!} \nonumber \\
  y & = \frac{(m+i)!}{m!} \label{eq:invar}
\end{align}
Therefore:
\begin{flalign*}
& \mathsf{4.}\: \{y*m=(m+i)!/m!\}\;\texttt{\textcolor{blue}{y:=y*m}}\;\{y=((m-1)+(i+1))!/(m-1)!\}\ && \mathsf{(3,\;Prec.\, Equiv.)} \\
& \mathsf{5.}\: \{y=(m+i)!/m!\}\;\texttt{\textcolor{blue}{y:=y*m;\:m:=m-1;\:i:=i+1;}}\;\{y=(m+i)!/m!\}\ && \mathsf{(1,2,4,\;Sequencing)}
\end{flalign*}
As long as \(b\) (i.e. \( i < n\)) holds, in each \texttt{Loop}, before and after \texttt{Body} executes, \(I\) holds, as proved above. So this establishes \(I\) in \eqref{eq:invar} as an invariant.
