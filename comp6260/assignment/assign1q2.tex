% --------------------- a ---------------------
\subsection{a}
\textbf{Solution}

Assume that
\begin{equation}
\label{eq:1}
  A = (p \lor q)
\end{equation}
\begin{align*}
  LHS
  & \equiv A \lor (\neg p \land \neg q)
  && \text{by \eqref{eq:1}} \\
  & \equiv (A \lor \neg p) \land (A \lor \neg q)
  && \text{Associativity} \\
  & \equiv ((p \lor q) \lor \neg p) \land ((p \lor q) \lor \neg q)
  && \text{by \eqref{eq:1}} \\
  & \equiv ((p \lor \neg p) \lor q) \land (p \lor (q \lor \neg q))
  && \text{Associativity, Commutativity} \\
  & \equiv (T \lor q) \land (p \lor T)
  && \text{Complements}
\end{align*}

Using truth table, it is easy to prove that whatever the truth value of \(p\) is, \(p \lor T\) always equals \(T\).

\begin{center}
\begin{tabular}{|l|l|c|}
  \hline
  \(p\) & \(T\) & \(p \lor T\) \\
  \hline
  T & T & T  \\
  F & T & T  \\
  \hline
\end{tabular}
\end{center}

The same also goes for \(T \lor q\) regardless of \(q\)'s truth value. Thus, we have:
\begin{align*}
  LHS
  & \equiv (T \lor q) \land (p \lor T) \\
  & \equiv T \land T \\
  & \equiv T \\
  & \equiv RHS
\end{align*}

% --------------------- b ---------------------
\subsection{b}
\textbf{Solution}
Similarly, assume that
\begin{equation}
\label{eq:2}
  A = (p \lor q)
\end{equation}
\begin{align*}
  LHS
  & \equiv A \land (\neg p \land \neg q)
  && \text{by \eqref{eq:2}} \\
  & \equiv (A \land \neg p) \land \neg q
  && \text{Associativity} \\
  & \equiv (\neg p \land (p \lor q)) \land \neg q
  && \text{by \eqref{eq:2}, Commutativity} \\
  & \equiv ((p \land \neg p) \lor (\neg p \land q)) \land \neg q
  && \text{Distributivity} \\
  & \equiv (F \lor (\neg p \land q)) \land \neg q
  && \text{Complements} \\
  & \equiv (\neg p \land q) \land \neg q
  && \text{Identity} \\
  & \equiv \neg p \land (q \land \neg q)
  && \text{Associativity}  \\
  & \equiv \neg p \land F && \text{Complements}
\end{align*}

Make the truth table for \(\neg p \land F\):
\begin{center}
\begin{tabular}{|l|l|l|c|}
  \hline
  \(p\) & \(\neg p\) & \(F\) & \(\neg p \land F\) \\
  \hline
  T & F & F & F \\
  F & T & F & F \\
  \hline
\end{tabular}
\end{center}
Thus, we have:
\begin{align*}
  LHS
  & \equiv \neg p \land F \\
  & \equiv F \\
  & \equiv RHS
\end{align*}
