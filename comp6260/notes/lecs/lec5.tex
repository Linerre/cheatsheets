\subsection*{Core}
Find the \textbf{weakest precondition} (most general).\\
\textcolor{bl}{Weakest} possible: \textcolor{gr}{\texttt{True}} \\
\textcolor{bl}{Strongest} possible: \textcolor{gr}{\texttt{False}} \\
\textbf{Strong} pre/post-conditions: more \textbf{specific} (say \emph{\textcolor{bl}{more}}) \\
\textbf{Weaker} pre/post-conditions: more \textbf{general} (say \emph{\textcolor{bl}{less}})

\subsection*{1/6 Assignment Rule}
\subsubsection*{Question Type 1}
Given post-condition \& one single assignment, ask for pre-condition:
\(\{?\}\,S\,\{Q\}\)

For example: \(\{?\}\ i := 2 * i \{i < 6\}\)

\textbf{Solution}
\begin{enumerate}
\item \textcolor{Cyan}{Copy} Q over to P: \(\{i_{post} < 6\}\)
\item Replace all LHS vars with RHS vars: \(\{2 * i_{orig} < 6\}\)
\item See if any math/logic equivalence (simplify P if possible): \(\{i_{orig} < 3\}\)
\end{enumerate}

\subsubsection*{Question Type 2}
Prove the given Hoare Triple: \(\{P\}\,V := E\,\{Q\}\)

For example: \(\{x > 3\}\;x:=x+2\;\{x>5\}\)
\textbf{Solution}
\begin{enumerate}
\item \textcolor{Cyan}{Copy} Q over to P: \(\{x_{post} > 5\}\)
\item Replace all LHS vars with RHS vars: \(x_{assign}+2 > 5\}\)
\item See if any math/logic equivalence (simplify P if possible): \(\{x_{orig} > 3\}\)
\end{enumerate}



\subsection*{2/6 Pre-condition Strengthening}

Logic: \(P_{strong} \rightarrow P_{orig} \)
\begin{displaymath}
  \frac{P_{strong} \rightarrow P_{weak} \quad \{P_{weak}\}\,S\,\{Q\} } {\{P_{s}\}\,S\,\{Q\}}
\end{displaymath}

\subsection*{3/6 Post-condition Weakening}
Logic: \(Q_{orig} \rightarrow Q_{weak} \)
\begin{displaymath}
  \frac{Q_{strong}\rightarrow Q_{weak} \quad \{P\}\,S\,\{Q_{strong}\} } {\{P\}\,S\,\{Q_{w}\}}
\end{displaymath}

\subsection*{4/6 Sequencing}
\subsubsection*{Question Type}
{
  % Reducing vertical spacing
  \setlength{\abovedisplayskip}{0pt}
  \setlength{\belowdisplayskip}{3pt}
  \setlength{\abovedisplayshortskip}{0pt}
  \setlength{\belowdisplayshortskip}{3pt}

  \[\textbf{Prove:}\quad \{P\}\,S_{1};\,S_{2};\,\ldots ;\,S_{n}\,\{Q\}\]
}
\textbf{Solution} Start backwards and use Assignment Rule:
\begin{enumerate}
\item\label{step1} \textcolor{Cyan}{Copy} \(Q_{n}\) as \(P_{n}\) for \(S_{n}\)
\item\label{step2} \textcolor{Cyan}{Use} assignment rule and math equivalence to simplify \(P_{n}\) if possible
\item \textcolor{Cyan}{Use} \(P_{n}\) as \(P_{n-1}\) for \(S_{n-1}\)
\item \textcolor{Cyan}{Repeat} \ref{step1} and \ref{step2} until getting \(P_{1}\)
\item \textcolor{Cyan}{Strengthen} precondition \(\{P\}\rightarrow \{P_{1}\}\)
\item (Optional) Sometimes need (0th/boolean) logic to simplify compound propositions.

\end{enumerate}

\subsection*{5/6 Conditionals}
\subsubsection*{Question Type}
{
  % Reducing vertical spacing
  \setlength{\abovedisplayskip}{0pt}
  \setlength{\belowdisplayskip}{3pt}
  \setlength{\abovedisplayshortskip}{0pt}
  \setlength{\belowdisplayshortskip}{3pt}

\[\textbf{Prove:}\quad \{P\}\;if\:b\:then\:S_{1}\:else\:S_{2}\;\{Q\}\]
}
\textbf{Solution} Use Conditional Rule to prove both:
\begin{description}
\item [premise1] \(\{P \land b\}\:S_{1}\:\{Q\}\)
\item [premise2] \(\{P \land \neg b\}\:S_{2}\:\{Q\}\)
\end{description}
For each premise:
\begin{enumerate}
\item\label{step1} \textbf{Use} Assignment rule to get new \(P_{assign}\)
\item\label{step2} \textbf{Use} Propositional logic, math (precondition equivalence) to \textbf{strengthen} the \(P_{assign}\) for \(P_{s}\) (which is the \(P\) needed in the premise1)
\end{enumerate}
Then \textbf{use} Conditional Rule to combine the proved premises.

\subsubsection*{Complete conditionals}
Conditionals with \texttt{else} branch are \emph{complete}.

\texttt{if b then S} $\Longleftrightarrow$ \texttt{if b then S else (do nothing)}

% TODO: Add `no-else' rule slides pp.41
