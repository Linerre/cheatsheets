\documentclass[10pt,a4paper,landscape]{article}

% -- Layout ----
\usepackage[top=1cm, bottom=1cm, left=0.5cm, right=0.5cm, landscape]{geometry}

% -- Titles ----
\usepackage[
  tiny,                     % text size title
  compact                   % reduce vertical space before/after title
]{titlesec}
% \titlespacing*

% -- Colors ----
\usepackage[dvipsnames]{xcolor}
\definecolor{gr}{HTML}{008000}
\definecolor{bl}{HTML}{2020C0}

% -- Lists -----
\usepackage[inline]{enumitem}
\setlist{noitemsep}

% -- Code listing ---
\usepackage{listings}
\lstset{
  aboveskip=3pt,
  belowskip=3pt,
  basicstyle=\small\ttfamily,
  commentstyle=\upshape\ttfamily,
  frame=single,
  language=Haskell,
}

% -- Multi Columns --
\usepackage{multicol}

% -- Global spacing settings ----
\setlength{\abovedisplayskip}{3pt}
\setlength{\belowdisplayskip}{3pt}
\setlength{\abovedisplayshortskip}{3pt}
\setlength{\belowdisplayshortskip}{3pt}
\setlength{\parindent}{0pt}
\setlength{\parskip}{0pt plus 0.5ex}
\begin{document}
% Suppress page number for all pages
\pagestyle{empty}
\begin{multicols*}{3}
\section{Boolean logic}

\section{Natural deduction}

\section{Structural induction}
Used to prove that some proposition P(x) holds for$\forall$ll x $\in$ recursively defined structure (e.g. lists, trees).

`\verb|()|' in Haskell changes the precedence (calling order) of functions. Nothing more.
\subsection*{Induction hypothesis}
\begin{itemize}
\item given as the known premise of a proof
\item ties recursive know in a proof
\item \textbf{must} be used
\end{itemize}
When proving, use only the below:
\begin{itemize}
\item the function definitions
\item the induction hypothesis
\item basic arithmetic
\end{itemize}
Sometimes prove from LHS to RHS; sometimes, from RHS to LHS; still sometimes, meet in the middle!
\subsection*{List induction rule}
\begin{displaymath}
\frac{P([])\qquad \forall x.\; \forall xs.\; P(xs) \rightarrow\; P(x:xs)}{\forall xs P(xs)}
\end{displaymath}
\begin{enumerate}
\item show that base case (empty list) $P([])$ holds
\item show that for any non-empty list (\verb|(x:xs)|) $P(x:xs)$ holds
\end{enumerate}
Step 2 \textbf{needs to use} induction hypothesis.

\subsection*{Choose proper formulation}
If a proof handles two or more lists, treat the one that remains unchanged in the definition(s) as \textbf{constant}:
\begin{lstlisting}
[] ++ ys = ys                       -- ( APP1 )
( x : xs ) ++ ys = x : ( xs ++ ys ) -- ( APP2 )
\end{lstlisting}
Then, choose induction on \texttt{xs} and treat \texttt{ys} as constant.
\subsection*{Len and map}
\begin{lstlisting}
length [] = 0                     -- ( LEN1 )
length ( x : xs ) = 1 + length xs -- ( LEN2 )

map f [] = []                     -- ( MAP1 )
map f ( x : xs ) = f x : map f xs -- ( MAP2 )
\end{lstlisting}

\subsection*{Tree induction rule}
\begin{displaymath}
\frac{P(Nul)\qquad \forall l.\; \forall x.\; \forall r.\; P(l) \land P(r) \rightarrow\; P(Node\,l\,x\,r:)}{\forall t.\; P(t)}
\end{displaymath}
\begin{enumerate}
\item show that base case (\verb|Node Nul|) $P(Nul)$ holds
\item show that $P(Node\;l\,x\,r)$ (\verb|Node lxr|) holds whenever both $P(l)$ and $P(r)$ are true
\end{enumerate}
\subsection*{Termination}

\section{Hoare logic}
%Find the \textbf{weakest precondition} (most general).\\
\textbf{Weakest} possible: \textbf{\texttt{True}} \\
\textbf{Strongest} possible: \textbf{\texttt{False}} \\
\textbf{Strong} pre/post-conditions: more \textbf{specific} (say \emph{more}) \\
\textbf{Weaker} pre/post-conditions: more \textbf{general} (say \emph{less})

\subsection*{1/6 Assignment Rule}
\subsubsection*{Question Type 1}
Given post-condition \& one single assignment, ask for pre-condition:
\(\{?\}\,S\,\{Q\}\)

For example: \(\{?\}\ i := 2 * i \{i < 6\}\)

\textbf{Solution}
\begin{enumerate}
\item \textbf{Copy} Q over to P: \(\{i_{post} < 6\}\)
\item Replace all LHS vars with RHS vars: \(\{2 * i_{orig} < 6\}\)
\item See if any math/logic equivalence (simplify P if possible): \(\{i_{orig} < 3\}\)
\end{enumerate}

\subsubsection*{Question Type 2}
Prove the given Hoare Triple: \(\{P\}\,V := E\,\{Q\}\)

For example: \(\{x > 3\}\;x:=x+2\;\{x>5\}\)
\textbf{Solution}
\begin{enumerate}
\item \textbf{Copy} Q over to P: \(\{x_{post} > 5\}\)
\item Replace all LHS vars with RHS vars: \(x_{assign}+2 > 5\}\)
\item See if any math/logic equivalence (simplify P if possible): \(\{x_{orig} > 3\}\)
\end{enumerate}



\subsection*{2/6 Pre-condition Strengthening}

Logic: \(P_{strong} \rightarrow P_{orig} \)
\begin{displaymath}
  \frac{P_{strong} \rightarrow P_{weak} \quad \{P_{weak}\}\,S\,\{Q\} } {\{P_{s}\}\,S\,\{Q\}}
\end{displaymath}

\subsection*{3/6 Post-condition Weakening}
Logic: \(Q_{orig} \rightarrow Q_{weak} \)
\begin{displaymath}
  \frac{Q_{strong}\rightarrow Q_{weak} \quad \{P\}\,S\,\{Q_{strong}\} } {\{P\}\,S\,\{Q_{w}\}}
\end{displaymath}

\subsection*{4/6 Sequencing}
\subsubsection*{Question Type}
{
  % Reducing vertical spacing
  \setlength{\abovedisplayskip}{0pt}
  \setlength{\belowdisplayskip}{3pt}
  \setlength{\abovedisplayshortskip}{0pt}
  \setlength{\belowdisplayshortskip}{3pt}

  \[\textbf{Prove:}\quad \{P\}\,S_{1};\,S_{2};\,\ldots ;\,S_{n}\,\{Q\}\]
}
\textbf{Solution} Start backwards and use Assignment Rule:
\begin{enumerate}
\item\label{step1} \textbf{Copy} \(Q_{n}\) as \(P_{n}\) for \(S_{n}\)
\item\label{step2} \textbf{Use} assignment rule and math equivalence to simplify \(P_{n}\) if possible
\item \textbf{Use} \(P_{n}\) as \(P_{n-1}\) for \(S_{n-1}\)
\item \textbf{Repeat} \ref{step1} and \ref{step2} until getting \(P_{1}\)
\item \textbf{Strengthen} precondition \(\{P\}\rightarrow \{P_{1}\}\)
\item (Optional) Sometimes need (0th/boolean) logic to simplify compound propositions.

\end{enumerate}

\subsection*{5/6 Conditionals}
\subsubsection*{Question Type}
{
  % Reducing vertical spacing
  \setlength{\abovedisplayskip}{0pt}
  \setlength{\belowdisplayskip}{3pt}
  \setlength{\abovedisplayshortskip}{0pt}
  \setlength{\belowdisplayshortskip}{3pt}

\[\textbf{Prove:}\quad \{P\}\;if\:b\:then\:S_{1}\:else\:S_{2}\;\{Q\}\]
}
\textbf{Solution} Use Conditional Rule to prove both:
\begin{description}
\item [premise1] \(\{P \land b\}\:S_{1}\:\{Q\}\)
\item [premise2] \(\{P \land \neg b\}\:S_{2}\:\{Q\}\)
\end{description}
For each premise:
\begin{enumerate}
\item\label{step1} \textbf{Use} Assignment rule to get new \(P_{assign}\)
\item\label{step2} \textbf{Use} Propositional logic, math (precondition equivalence) to \textbf{strengthen} the \(P_{assign}\) for \(P_{s}\) (which is the \(P\) needed in the premise1)
\end{enumerate}
Then \textbf{use} Conditional Rule to combine the proved premises.

\subsubsection*{Complete conditionals}
Conditionals with \texttt{else} branch are \emph{complete}.

\texttt{if b then S} $\Longleftrightarrow$ \texttt{if b then S else (do nothing)}
\begin{displaymath}
  \frac{\{P \land\; b\}\;S\{Q\} \quad \{P \land \;\neg b\}\;\texttt{(do nothing)}\;\{Q\}}{\{P\}\;if\;b\;then\;S\;\{Q\}}
\end{displaymath}
\texttt{do nothing} can be \textbf{strengthened} to sth like \verb|x := x|

\end{multicols*}
\end{document}
