At worst, NFA$_{n\text{-states}}$ converts to DFA$_{2^{n}\text{-states}}$\\
\vspace{-0.8em}
{\footnotesize
\[
  N_{\text{NFA}} = (Q_{N},\Sigma,\delta_{N},\mathbf{q_{0}},F_{N}) \Rightarrow (Q_{D},\Sigma,\delta_{D},{\mathbf{\{q_{0}\}}},F_{D}) = D_{\text{DFA}}
\]
}
% subset construction steps
{\footnotesize
\begin{minipage}{0.5\linewidth}
Extend NFA transition table (ref)
  \begin{enumerate}[align=left]
    \item\label{nfa:sub} Find \textbf{sets of states} that NFA's \(\{q_{0}\}\),\(\{q_{1}\}\)\ldots\(\{q_{n }\}\) can reach
    \item If any \textbf{new state set} in step \ref{nfa:sub}, see what \textbf{state set} it can reach
    \item Unreachable/dead states marked by \(\emptyset\)  (``thrown away'')
    \item If any state set found above has any \(q_{i} \in F_{N}\), mark the set with *
    \item (optional) mark state sets as \(S_{0\ldots i}\) and draw NFA transition diagram
  \end{enumerate}
\end{minipage}%
\begin{minipage}{0.45\linewidth}
  \centering
\begin{tabular}{r||c|c}
   & 0 & 1 \\
  \hline
  \(\emptyset\) & \(\emptyset\) & \(\emptyset\)\\
  \(\rightarrow \{q_{0}\}\) & \(\{q_{0},q_{1}\}\) & \(\{q_{0}\}\)\\
  \(\{q_{1}\}\) & \(\emptyset\) & \(\{q_{2}\}\)\\
  \(*\{q_{2}\}\) & \(\emptyset\) & \(\emptyset\)\\
  \(\{q_{0},q_{1}\}\) & \(\{q_{0},q_{1}\}\) & \(\{q_{0},q_{2}\}\)\\
  \(*\{q_{0},q_{2}\}\) & \(\{q_{0},q_{1}\}\) & \(\{q_{0}\}\)\\
  \(*\{q_{1},q_{2}\}\) & \(\emptyset\) & \(\{q_{0}\}\)\\
  \(*\{q_{0},q_{1},q_{2}\}\) & \(\{q_{0},q_{1}\}\) & \(\{q_{0},q_{2}\}\)\\
  \hline
\end{tabular}
\end{minipage}
}
