{\footnotesize
\begin{minipage}{0.5\linewidth}
  \begin{enumerate}[align=left]
    \item state \(q\) itself is in \textsc{eclose}(\(q\))
    \item\label{nfa:epi1} following all \(epsilon\) transitions out of \(q\)
    \item\label{nfa:epi2} each of \(q\)'s \(epsilon\)-reachable state \(p_{q\text{-}\epsilon}\) is in \textsc{eclose}(\(q\))
    \item each of \(p_{q\text{-}\epsilon}\)'s  \(epsilon\)-reachable state \(r_{p\text{-}\epsilon}\) is in \textsc{eclose}(\(q\))
    \item for state set \(S\), do above steps 1-4 for each in \(S\) and take the union
  \end{enumerate}
\end{minipage}%
\begin{minipage}{0.45\linewidth}
\begin{tikzpicture}[bezier bounding box,node distance=0.8cm]%reduce vertical space around graph
  \node[state, initial, minimum size=0.8em] (q1) {$q_{1}$};
  \node[state, minimum size=0.8em, above right=of q1] (q2) {$q_{2}$};
  \node[state, minimum size=0.8em, right=of q2] (q4) {$q_{4}$};
  \node[state, minimum size=0.8em, below=of q4] (q5) {$q_{5}$};
  \node[state, accepting, minimum size=0.5em, below right=of q1] (q3) {$q_{3}$};

  \path[->]
  (q1) edge node[above,sloped]{\(\varepsilon\)}(q2)
  (q1) edge node[below,sloped]{\(\varepsilon\)}(q3)
  (q3) node[below,yshift=-0.3cm]{\textsc{eclose}(\(q_{1}\))=\(\{q_{1},q_{2},q_{3},q_{4},q_{5}\}\)}
  (q2) edge node[above]{\(\varepsilon\)}(q4)
  (q4) edge node[left]{\(\varepsilon\)}(q5)
  (q5) edge node[above,sloped]{a}(q3);
\end{tikzpicture}
\end{minipage}
}
