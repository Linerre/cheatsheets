\documentclass[10pt,a4paper,landscape]{article}
% -- Layout ----
\usepackage[top=0.6cm, bottom=0.6cm, left=0.5cm, right=0.5cm, landscape]{geometry}

% -- Titles ----
\usepackage[
  tiny,                     % text size title
  compact                   % reduce vertical space before/after title
]{titlesec}
% \titlespacing*
\titleformat{\section}{\normalfont\small\bfseries}{\thesection}{0em}{} % Remove space before and after section titles
\titleformat{\subsection}{\normalfont\small\bfseries}{\thesubsection}{0em}{} % Remove space before and after subsection titles
\titlespacing*{\section}{0pt}{0pt}{0pt} % Remove space before/after section titles
\titlespacing*{\subsection}{0pt}{0pt}{0pt} % Remove space before/after subsec titles

% -- Colors ----
\usepackage[dvipsnames]{xcolor}
\definecolor{dmm}{RGB}{192,192,192} % Define a custom dimmed text color
\definecolor{cmt}{RGB}{61,123,123}

% -- Math ------
\usepackage{mathtools}
\usepackage{amssymb}
\usepackage{turnstile}%better vdash

% -- Lists -----
\usepackage[inline]{enumitem}
\setlist{noitemsep}% Remove vspace between items
% Set vspace before and after  list environments as well as the left margin
\setlist[itemize,1]{leftmargin=.6em,labelindent=0pt,labelsep=2pt,
  topsep=1pt,partopsep=1pt}
\setlist[enumerate,1]{leftmargin=1em,labelindent=0pt,labelsep=2pt,
  topsep=1pt,partopsep=1pt}
\setlist[itemize,2]{leftmargin=.3em,labelindent=1pt,topsep=1pt,partopsep=1pt}
\setlist[enumerate,2]{leftmargin=0.2em,labelindent=1pt,topsep=1pt,partopsep=1pt}
\setlist[description]{labelwidth=\linewidth,font=\small\bfseries,leftmargin=1em,topsep=1pt,partopsep=1pt}
% -- Code listing ---
\usepackage{listings}
\lstset{
  aboveskip=3pt,
  belowskip=3pt,
  basicstyle=\small\ttfamily,
  breaklines=true,
  % commentstyle=\upshape\ttfamily,
  captionpos=b,
  commentstyle=\color{cmt},
  frame=single,
  keepspaces=false,
  keywordstyle=\bfseries,
  showspaces=false,
  showstringspaces=false,
  showtabs=false,
  tabsize=2,
}

% Parse Trees
\usepackage{tikz}
\usetikzlibrary{ arrows, automata, bbox, calc, positioning,  decorations.pathmorphing, decorations.pathreplacing, decorations.shapes, }
\tikzset{
% ->, % makes the edges directed
>=stealth', % makes the arrow heads bold
node distance=1cm, % specifies minimum distance between two nodes
% small/.style={},
every state/.style={thick}, % sets the properties for each ’state’ node
every node/.style={inner sep=1pt},
initial text=start, % sets the text that appears on the start arrow
}

% Place a figure env right here via [H] option
\usepackage{float}

% Side by side figure
\usepackage{subcaption}
% \usepackage{caption}
% \captionsetup{belowskip=0pt, aboveskip=0pt}


% -- Multi-Col layout --
\usepackage{multicol}

% No indentation
\setlength\parindent{0pt}
\setlength\abovedisplayskip{-5pt}
\setlength\belowdisplayskip{-5pt}
\setlength\abovedisplayshortskip{-4pt}
\setlength\belowdisplayshortskip{-4pt}
\newcommand{\gor}{\;|\;}
\newcommand{\num}{\texttt{\#}~}
\renewcommand{\arraystretch}{1.2}


\begin{document}
% Suppress page number for all pages
\pagestyle{empty}

% Each section goes into this env
\begin{multicols*}{3}
\section*{An NFA which accepts all strings that end in `01'}
\begin{minipage}{\linewidth}
  \centering
  \begin{tikzpicture}[node distance=0.3cm]
    \node[comp, shape=circle, initial, initial where=left] (q0) at (0,0) {$q_{0}$};
    \node[comp, shape=circle] (q1) at (2,0) {$q_{1}$};
    \node[comp, shape=circle, accepting] (q2) at (4,0) {$q_{2}$};
    \path[->]
    (q0) edge[loop] node[above] {0,1} (q0)
    (q0) edge node[above] {0} (q1)
    (q1) edge node[above] {1} (q2)
    ;
  \end{tikzpicture}
\end{minipage}
\[
  A = (Q,\Sigma,\delta,q_{0},F) = (\{q_{0},q_{1},q_{2}\},\{0,1\},\delta,q_{0},\{q_{2}\})
\]
\begin{enumerate}
\item $Q$ is a finite set of states
\item $\Sigma$ is a finite set of input symbols (alphabet)
\item $q_{0} \in Q$ is the initial state
\item $F \subseteq Q$ (or $A \subseteq Q$) is the set of final or accepting states
\item $\delta$, \emph{transition function}, takes a state in $Q$ and an input symbol in $\Sigma$ as arguments and returns a subset of $Q$:
\end{enumerate}
\begin{equation*}
  Q \eqnmark[red]{n1}{\times} (\Sigma \cup \{\Lambda\}) \to 2^{Q}
\end{equation*}
\annotate[yshift=-.1em]{below,right}{n1}{Cartesian product}

\begin{minipage}{.5\linewidth}
  \centering
  \begin{tabular}{r||l|l}
    state & 0 & 1 \\
    \hline
    \hline
    $\to\; q_{0}$    & $\{q_{0}, q_{1}\}$ & $\{q_{0}\}$\\
    $q_{1}$          & $\emptyset$      & $\{q_{2}\}$\\
    $*q_{2}$         & $\emptyset$      & $\emptyset$
  \end{tabular}
\end{minipage}
\begin{minipage}{.5\linewidth}
  \flushleft
  \begin{itemize}
  \item transition table shows how transition function $\delta$ works
  \item For NFA, each $\delta(s_{i}, w)$ returns a set (may be $\emptyset$)
  \item $\delta(s_{i}, input)$ can also return $\emptyset$
  \end{itemize}
\end{minipage}

\section*{The Extended Transition Function $\delta^{*}$ (or $\hat{\delta}$) of an NFA}
Given an NFA $M = (Q,\Sigma,q_{0},A,\delta)$, its extended transition function is defined as:
\begin{equation*}
  {\delta^{*}}: Q \times \Sigma^{*} \to 2^{Q}
\end{equation*}
\begin{enumerate}
\item For every $q \in Q, \delta^{*}(q,\Lambda) = \Lambda(\{q\})$ ($\epsilon$-closure for every $q$)
\item For every $q \in Q$, every $y \in \Sigma^{*}$ and every $\sigma \in \Sigma$:
  \begin{equation*}
    \delta^{*}(q,y\sigma) = \Lambda\left(\bigcup\{\delta(p,\sigma)\;|\;p \in \delta^{*}(q,y)\}\right)
  \end{equation*}
\item input string $w = y\sigma$ where $\sigma$ is final symbol of $w$ and $y$ is the rest ($y$ often starts as empty $\Lambda$ and grows longer)
\item \ml{first} compute $\delta(q,y)$ and gets a set $P\subseteq Q:\{p_{1},p_{2},\cdots,p_{k}\}$
\item \ml{then} for each $p_{k} \in P$, compute $\delta(p_{k},\sigma) \to$ set for next loop
\item A string $x \in \Sigma^{*}$ is accepted by $M$ if $\delta^{*}(q_{0},x) \cap A \neq \emptyset$
\item The language $L(M)$ accepted by $M$ is the set of all strings accepted by $M$ ($\Sigma^{*}$ is the set of all strings over $\Sigma$)
\end{enumerate}
\section*{Processing `00101' for the NFA above using its $\delta^{*}$ from $q_{0}$}
\begin{enumerate}
\item $\delta^{*}(q_{0},\Lambda) = \{q_{0}\}$ (input string $w = 00101 = \Lambda 00101$)
\item $\delta^{*}(q_{0},0) = \delta(q_{0},0) = \{q_{0}, q_{1}\}$
\item $\delta^{*}(q_{0},00) = \delta(q_{0},0) \cup \delta(q_{1},0) = \{q_{0}, q_{1}\} \cup \emptyset = \{q_{0}, q_{1}\}$
\item $\delta^{*}(q_{0},001) = \delta(q_{0},1) \cup \delta(q_{1},1) = \{q_{0}\} \cup \{q_{2}\} = \{q_{0}, q_{2}\}$
\item $\delta^{*}(q_{0},0010) = \delta(q_{0},0) \cup \delta(q_{2},0) = \{q_{0}, q_{1}\} \cup \emptyset = \{q_{0}, q_{1}\}$
\item $\delta^{*}(q_{0},00101) = \delta(q_{0},1) \cup \delta(q_{1},1) = \{q_{0}\} \cup \{q_{2}\} = \{q_{0}, q_{2}\}$
\end{enumerate}
The above means: starting in state $q_{0}$, after processing $w$, the NFA will stay in any state in the set $\{q_{0},q_{2}\}$

% DFA constructed from the above NFA
\section*{NFA to DFA: subset construction}
\begin{enumerate}
\item In theory, given an NFA $N$ with $n$ states, its equivalent DFA $D$ will have (up to) $2^{n}$ states
\item In practice, most states of $D$ are \mb{inaccessible} from start state and can be thrown away
\item $A_{D}$ (or $F_{D}$) is a set. Each $S$ in $A_{D}$ is a set of states:
  \begin{itemize}
  \item $S$ is a set of some or all states from $Q_{N}$: $S \subseteq Q_{N}$
  \item $S$ includes \mb{at least 1} accepting state of $N$: $S \cap F_{N} \neq \emptyset$
  \end{itemize}
\item For each $S \subseteq Q_{N}$ and for each input symbol $a \in \Sigma$:
  \begin{equation*}
    \delta_{D}(S,a) = \bigcup_{p \in S}\;\delta_{N}(p,a)
  \end{equation*}
  \begin{enumerate}[label={\alph*.}]
  \item for every $p \in S$, find all $\delta(p,a)$ for $N$
  \item take the union of all those states resulted in the above
  \end{enumerate}
\end{enumerate}


\begin{minipage}{.6\linewidth}
  \centering
  \begin{tabular}{r||l|l}
    ($2^{n}$) states & 0 & 1 \\
    \hline
    \hline
    $\emptyset$ & $\emptyset$ & $\emptyset$ \\
    $\to\; \{q_{0}\}$    & $\{q_{0},q_{1}\}$ & $\{q_{0}\}$\\
    $\{q_{1}\}$          & $\emptyset$      & $\{q_{2}\}$\\
    $*\{q_{2}\}$         & $\emptyset$      & $\emptyset$\\
    $\{q_{0},q_{1}\}$ & $\{q_{0},q_{1}\}$ & $\{q_{0},q_{2}\}$\\
    $*\{q_{0},q_{2}\}$ & $\{q_{0},q_{1}\}$ & $\{q_{0}\}$\\
    $*\{q_{1},q_{2}\}$ & $\emptyset$ & $\{q_{2}\}$\\
    $*\{q_{0},q_{1},q_{2}\}$ & $\{q_{0},q_{1}\}$ & $\{q_{0},q_{2}\}$\\
    \hline
    \hline
    \multicolumn{3}{l}{*: set with accepting state of $N$}
  \end{tabular}
\end{minipage}
\begin{minipage}{.4\linewidth}
  \begin{tabular}{r||l|l}
    states & 0 & 1 \\
    \hline
    \hline
    $A$ & $A$ & $A$ \\
    $\to\;B$  & $E$ & $B$\\
    $C$       & $A$ & $D$\\
    $*D$      & $A$ & $A$\\
    $E$       & $E$ & $F$\\
    $*F$      & $E$ & $B$\\
    $*G$      & $A$ & $D$\\
    $*H$    & $E$ & $F$\\
    \hline
    \hline
    \multicolumn{3}{l}{*set is accepting state}
  \end{tabular}
\end{minipage}
\begin{enumerate}
\item Starting in $B$, in this case, only $B$, $E$, $F$ are reachable: $B\to E$, $E\to F$ and $F\to B$
\item Other 5 states inaccessible from $B$ (may not exist at all)
\item \mo{lazy evaluation} for constructing transition table entries:
  \begin{enumerate}[label={\alph*.}]
  \item $\{q_{0}\} (\in Q_{N})$ as $N$'s start state, is accessible
  \item If $S \subseteq Q_{n}$ is accessible, then for each input symbol $a$, $\delta_{D}(S,a)$ (\{states\}) will also be accessible. For example:
    \begin{align*}
      \delta_{D}(\{q_{0},q_{1}) & = \delta_{N}(q_{0},1) \cup \delta_{N}(q_{1}, 1)\\
                                & = \{q_{0}\} \cup \{q_{2}\} \\
                                & = \{q_{0},q_{2}\}
    \end{align*}
  \item Use these accessible set(s) as $Q_{D}$ to draw DFA diagram
  \end{enumerate}
\end{enumerate}
\begin{minipage}{\linewidth}
  \centering
  \begin{tikzpicture}[node distance=0.4cm]
    \node[comp,shape=ellipse, initial, initial where=left] (q0) at (0,0) {\{$q_{0}$\}};
    \node[comp,shape=ellipse] (q0q1) at (2,0) {\{$q_{0},q_{1}$\}};
    \node[comp,shape=ellipse, accepting] (q0q2) at (4.5,0) {\{$q_{0},q_{2}$\}};
    \path[->]
    (q0) edge[loop] node[above] {1} (q0)
    (q0) edge node[above] {0} (q0q1)
    (q0q1) edge[loop] node[above] {0} (q0q1)
    (q0q1) edge node[above] {1} (q0q2)
    (q0q2) edge[bend left] node[below] {0} (q0q1)
    (q0q2) edge[in=-45, out=-120] node[below] {1} (q0)
    ;
  \end{tikzpicture}
\end{minipage}

% Eclose
\section*{Compute $\epsilon$-CLOSE($q$) (or ECLOSE($q$)) with recursion}
% \begin{enumerate}
% \end{enumerate}
\begin{minipage}{0.52\linewidth}
  \flushleft
\begin{enumerate}
\item A state $q$, itself is in \textsc{eclose}($q$) (\mr{even w/t $\epsilon$ arc})
\item Suppose on input $\epsilon$, $q$ can transit to states $p_{1},\cdots,p_{n}$ (the result of $\delta(q,\epsilon)$), all these states are in \textsc{eclose}($q$)
\item For every $r \in \delta(p_{i},\epsilon)$ where $1 \leq i \leq n$, $r$ is in \textsc{eclose}($q$)
\end{enumerate}
\end{minipage}
\begin{minipage}{0.48\linewidth}
\begin{tikzpicture}[node distance=0.5cm]%reduce vertical space around graph
  \node[comp, shape=circle, initial] (q1) {$q_{1}$};
  \node[comp, shape=circle, above right=of q1] (q2) {$q_{2}$};
  \node[comp, shape=circle, right=of q2] (q4) {$q_{4}$};
  \node[comp, shape=circle, below=of q4] (q5) {$q_{5}$};
  \node[comp, shape=circle, accepting, below right=of q1] (q3) {$q_{3}$};
  \node[text width=3cm, below=of q1, yshift=-.3cm, xshift=-.6cm] (e) {
    \small{\textsc{eclose}(\(q_{1}\))=\(\{q_{1},q_{2},q_{3},q_{4},q_{5}\}\)}
  };

  \path[->]
  (q1) edge node[above,sloped]{\(\varepsilon\)}(q2)
  (q1) edge node[below,sloped]{\(\varepsilon\)}(q3)
  (q2) edge node[above]{\(\varepsilon\)}(q4)
  (q4) edge node[left]{\(\varepsilon\)}(q5)
  (q5) edge node[above,sloped]{a}(q3);
\end{tikzpicture}
\end{minipage}

\section*{Eliminate $\epsilon$-transitions from NFA $N = (Q_{N},\Sigma,\delta,q_{0},F_{N})$}
\begin{enumerate}
\item For each $q \in Q_{N}$, compute $\epsilon$-closure $\delta(q,\epsilon)$
\item Copy $N$ to get $N'$ and remove all $\epsilon$ transition arcs in $N'$
\item For every \(q \in Q_{N}\), if $\delta(q,\epsilon)$ contains any \(p \in F_{N}\), \mo{mark} $q$ as an accepting state in $N'$
\item For every \textsc{eclose}($q$) set in step 1: for every $s \in$ \textsc{eclose}($q$), if $s \overset{a}{\to}t \in Q_{N}$, then put an arc $q \overset{a}{\to} t$ in $N'$. For example: \textsc{eclose}($q_{0}$) = \{$q_{0}, q_{1}, q_{2}$\} and $q_{1}\overset{a}{\to}q_{3}$ in $N$: $q_{0}\overset{a}{\to}q_{3}$ in $N'$
\end{enumerate}
\begin{minipage}{0.5\linewidth}
  \centering
  \begin{tikzpicture}[node distance=0.6cm]
    \node[comp, shape=circle, initial] (q0) {$q_{0}$};
    \node[comp, shape=circle, right=of q0] (q1) {$q_{1}$};
    \node[comp, shape=circle, right=of q1] (q2) {$q_{2}$};
    \node[comp, shape=circle, below=of q1] (q3) {$q_{3}$};
    \node[comp, shape=circle, accepting, right=of q3] (q4) {$q_{4}$};
    \node[below=of q0, xshift=-1cm] (n1) {$N$};

    \path[->]
    (q0) edge node[above]{\(\varepsilon\)}(q1)
    (q0) edge node[below,sloped]{d}(q3)
    (q1) edge node[above]{\(\varepsilon\)}(q2)
    (q1) edge node[left]{a}(q3)
    (q2) edge[bend left] node[right]{c}(q4)
    (q3) edge node[above]{\(\varepsilon\)}(q4)
    (q4) edge[bend left] node[left]{b}(q2);
  \end{tikzpicture}
\end{minipage}
\begin{minipage}{0.5\linewidth}
  \begin{tikzpicture}[node distance=0.6cm]
    \node[comp, shape=circle, initial,] (q0) {$q_{0}$};
    \node[comp, shape=circle, right=of q0] (q1) {$q_{1}$};
    \node[comp, shape=circle, accepting, below=of q1] (q3) {$q_{3}$};
    \node[comp, shape=circle, right=of q3] (q2) {$q_{2}$};
    \node[comp, shape=circle, accepting, right=of q1] (q4) {$q_{4}$};
    \node[below=of q0, xshift=-1cm] (n1) {$N'$};

    \path[->]
    (q0) edge[draw=red] node[below,sloped]{a,d}(q3)
    (q0) edge[out=30, in=140,draw=red] node[above]{c}(q4)
    (q1) edge[draw=red] node[above]{c}(q4)
    (q1) edge node[left]{a}(q3)
    (q2) edge[bend left] node[left]{c}(q4)
    (q3) edge[draw=red] node[above]{b}(q2)
    (q4) edge[bend left] node[right]{b}(q2);
  \end{tikzpicture}
\end{minipage}
\section*{Minimize DFA}
Given an DFA, there is an equivalent DFA with minimum-number of states and such DFA is unique
\begin{enumerate}
\item \mr{remove} any state unreachable from start state
\item \mr{split} states into \mb{final} group and \mb{non-final} group
\item \mr{find} where each state goes on some input
\item if 2 states with same input go to different groups, split out
\item if there is only one state in a group, the state is unique
\item repeat the steps 3-5 until no splitting for any group
\end{enumerate}
\vspace{-1em}
\begin{minipage}{0.45\linewidth}
  \begin{align*}
    & [[S_{0},S_{1},S_{2},T_{1},T_{2}],[S_{3},S_{4}]] && \text{(remove unreachable)}\\
    & [[S_{0},S_{1},S_{2},T_{1},T_{2}],[S_{3},S_{4}]] && \leftarrow 0\quad \text{(init)}\\
    & [[S_{0},S_{1},S_{2},T_{1},T_{2}],[S_{3},S_{4}]] && \leftarrow 1\\
    & [[S_{0},T_{1},T_{2}],[S_{1}],[S_{2}],[S_{3},S_{4}]] && \leftarrow 0\\
    & [[T_{1},T_{2}],[S_{0}],[S_{1}],[S_{2}],[S_{3},S_{4}]] &&  \leftarrow 1\\
    & [[T_{1},T_{2}],[S_{0}],[S_{1}],[S_{2}],[S_{3},S_{4}]] &&  \leftarrow 0\\
    & [[T_{1},T_{2}],[S_{0}],[S_{1}],[S_{2}],[S_{3},S_{4}]] && (\text{cannot split further}) \\
    & T_{1} \equiv T_{2}, S_{3} \equiv S_{4} && (\text{found equivalent states})
  \end{align*}
\end{minipage}
% DFA minimisation
\begin{minipage}{0.6\linewidth}
  \begin{tikzpicture}[node distance=0.6cm]
    \node[comp, shape=circle, initial, initial where=below] (s0) {$S_{0}$};
    \node[comp, shape=circle, right=of s0] (s1) {$S_{1}$};
    \node[comp, shape=circle, right=of s1] (s2) {$S_{2}$};
    \node[comp, shape=circle, accepting, right=of s2] (s3) {$S_{3}$};
    \node[comp, shape=circle, accepting, below=of s3] (s4) {$S_{4}$};
    \node[comp, shape=circle, below=of s1] (t2) {$T_{1}$};
    \node[comp, shape=circle, below=of s2] (r1) {$R_{1}$};

    % Calculate the midpoint between S3 and S4
    \coordinate (midpoint) at ($(s3)!0.5!(s4)$);

    \node[comp, shape=circle, right=1cm of midpoint] (t1) {$T_{2}$};
    \draw[red, line width=1pt] (r1.south west) -- (r1.north east);
    \draw[red, line width=1pt] (r1.north west) -- (r1.south east);

    \path[->]
    (s0) edge node[above]{0}(s1)
    (s0) edge node[below,sloped]{1}(t2)
    (s1) edge node[above]{1}(s2)
    (s1) edge node[right]{0}(t2)
    (s2) edge node[above]{1}(s3)
    (s2) edge node[below,sloped]{0}(t2)
    (s3) edge[bend left] node[right]{1}(s4)
    (s3) edge node[above,sloped]{0}(t1)
    (s4) edge node[below]{0}(t1)
    (s4) edge[bend left] node[right, xshift=-.1cm]{1}(s3)
    (r1) edge node[below,sloped,xshift=-.2cm,yshift=.1cm]{0}(s3)
    (r1) edge node[right]{1}(s2)
    (t1) edge[loop above] node{0,1}(t1);
  \end{tikzpicture}
\end{minipage}
\begin{minipage}{0.42\linewidth}
  \begin{tikzpicture}[node distance=.8cm]
    \node[comp, shape=circle, initial, initial where=below] (s0) {$S_{0}$};
    \node[comp, shape=circle, right=of s0] (s1) {$S_{1}$};
    \node[comp, shape=circle, right=of s1] (s2) {$S_{2}$};
    \node[comp, shape=circle, accepting, below=of s2] (s3) {$S_{3}$};
    \node[comp, shape=circle, below=of s1] (t1) {$T_{1}$};

    \path[->]
    (s0) edge node[above]{0}(s1)
    (s0) edge node[above,sloped]{1}(t1)
    (s1) edge node[above]{1}(s2)
    (s1) edge node[right]{0}(t1)
    (s2) edge node[right]{1}(s3)
    (s2) edge node[below,sloped]{0}(t1)
    (s3) edge node[below]{0}(t1)
    (t1) edge[loop left] node{0,1}(t1);
  \end{tikzpicture}
\end{minipage}
\end{multicols*}
\end{document}
