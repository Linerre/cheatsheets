\documentclass[10pt,a4paper,landscape]{article}
% -- Layout ----
\usepackage[top=0.6cm, bottom=0.6cm, left=0.5cm, right=0.5cm, landscape]{geometry}

% -- Titles ----
\usepackage[
  tiny,                     % text size title
  compact                   % reduce vertical space before/after title
]{titlesec}
% \titlespacing*
\titleformat{\section}{\normalfont\small\bfseries}{\thesection}{0em}{} % Remove space before and after section titles
\titleformat{\subsection}{\normalfont\small\bfseries}{\thesubsection}{0em}{} % Remove space before and after subsection titles
\titlespacing*{\section}{0pt}{0pt}{0pt} % Remove space before/after section titles
\titlespacing*{\subsection}{0pt}{0pt}{0pt} % Remove space before/after subsec titles

% -- Colors ----
\usepackage[dvipsnames]{xcolor}
\definecolor{dmm}{RGB}{192,192,192} % Define a custom dimmed text color
\definecolor{cmt}{RGB}{61,123,123}

% -- Math ------
\usepackage{mathtools}
\usepackage{amssymb}
\usepackage{turnstile}%better vdash

% -- Lists -----
\usepackage[inline]{enumitem}
\setlist{noitemsep}% Remove vspace between items
% Set vspace before and after  list environments as well as the left margin
\setlist[itemize,1]{leftmargin=.6em,labelindent=0pt,labelsep=2pt,
  topsep=1pt,partopsep=1pt}
\setlist[enumerate,1]{leftmargin=1em,labelindent=0pt,labelsep=2pt,
  topsep=1pt,partopsep=1pt}
\setlist[itemize,2]{leftmargin=.3em,labelindent=1pt,topsep=1pt,partopsep=1pt}
\setlist[enumerate,2]{leftmargin=0.2em,labelindent=1pt,topsep=1pt,partopsep=1pt}
\setlist[description]{labelwidth=\linewidth,font=\small\bfseries,leftmargin=1em,topsep=1pt,partopsep=1pt}
% -- Code listing ---
\usepackage{listings}
\lstset{
  aboveskip=3pt,
  belowskip=3pt,
  basicstyle=\small\ttfamily,
  breaklines=true,
  % commentstyle=\upshape\ttfamily,
  captionpos=b,
  commentstyle=\color{cmt},
  frame=single,
  keepspaces=false,
  keywordstyle=\bfseries,
  showspaces=false,
  showstringspaces=false,
  showtabs=false,
  tabsize=2,
}

% Parse Trees
\usepackage{tikz}
\usetikzlibrary{ arrows, automata, bbox, calc, positioning,  decorations.pathmorphing, decorations.pathreplacing, decorations.shapes, }
\tikzset{
% ->, % makes the edges directed
>=stealth', % makes the arrow heads bold
node distance=1cm, % specifies minimum distance between two nodes
% small/.style={},
every state/.style={thick}, % sets the properties for each ’state’ node
every node/.style={inner sep=1pt},
initial text=start, % sets the text that appears on the start arrow
}

% Place a figure env right here via [H] option
\usepackage{float}

% Side by side figure
\usepackage{subcaption}
% \usepackage{caption}
% \captionsetup{belowskip=0pt, aboveskip=0pt}


% -- Multi-Col layout --
\usepackage{multicol}

% No indentation
\setlength\parindent{0pt}
\setlength\abovedisplayskip{-5pt}
\setlength\belowdisplayskip{-5pt}
\setlength\abovedisplayshortskip{-4pt}
\setlength\belowdisplayshortskip{-4pt}
\newcommand{\gor}{\;|\;}
\newcommand{\num}{\texttt{\#}~}
\renewcommand{\arraystretch}{1.2}


\begin{document}
% Suppress page number for all pages
\pagestyle{empty}
% Each section goes into this env
\begin{multicols*}{3}
% \input{sec/LR0}
% \section*{Conflicts in LR parsing}
\begin{itemize}
\item \mb{shift} the dot past the input symbol currently before the dot
\item[] Before: $X\to \lrd A\beta$; After: $X\to A\lrd\beta$
\item \mb{reduce}: when dot at the end of item, RHS of the production can be replaced with the LHS
\item In a state $I$, expect to have either a reduce or a shift action, not both. Otherwise, there is a \mo{conflict}:
  \begin{enumerate}
  \item \mb{S}/\mb{R} conflict: state $I$ has a reduce \textbf{AND} a shift action
  \item \mb{R}/\mb{R} conflict: state $I$ has two reduce actions
  \end{enumerate}
\end{itemize}
\section*{Example: A grammar with S/R conflict}
\begin{minipage}{.5\linewidth}
\begin{align*}
  _0\quad S&\to E\$   \\
  _1\quad E&\to T + E
\end{align*}
\end{minipage}
\begin{minipage}{.5\linewidth}
\begin{align*}
  _2\quad E&\to T \\
  _3\quad T&\to x
\end{align*}
\end{minipage}
\section*{grammar DFA with R/S conflict}
\begin{tikzpicture}
  % I1
  \node (i1) [lrst, anchor=north east] {
    $S\to \lrd E\$$\\
    $E\to \lrd T + E$\\
    $E\to \lrd T$\\
    $T\to \lrd x$
  };
  \node [above=1mm of i1.north east, font=\footnotesize\bf]{1};

  %I2
  \node (i2) [lrst, right=12mm of i1.north east, anchor=north west] {
    $S\to E\lrd\$$
  };
  \node (m2) [above=1mm of i2.north east, font=\footnotesize\bf]{2};

  %I3
  \node (i3) [lrst, right=12mm of i1.east, anchor=north west] {
    $E\to T\lrd + E$\\
    $E\to T\lrd $
  };
  \node (m3) [right=1mm of i3.north east, font=\footnotesize\bf]{3};

  %I5
  \node (i5) [lrst, below=15mm of i1.south, anchor=north] {
    $T\to x\lrd$
  };
  \node (m5) [above=1mm of i5.north east, font=\footnotesize\bf]{5};

  %I4
  \node (i4) [lrst, below=6mm of i3, anchor=north] {
    $E\to T + \lrd E$\\
    $E\to \lrd T$\\
    $E\to \lrd T + E$ \\
    $T\to \lrd x$
  };
  \node (m4) [right=1mm of i4.north east, font=\footnotesize\bf]{4};

  %I6
  \node (i6) [lrst, right=10mm of i4, anchor=west] {
    $E\to T + E\lrd$
  };
  \node (m6) [above=1mm of i6.north east, font=\footnotesize\bf]{6};

  % note
  \node(note)[right=8mm of i3.north east, anchor=west, text width=3cm, font=\small] {
    In $I_3$, there are a shift action (over $+$) and a reduce action (dot at the end) $\to$ a S/R conflict in $I_3$ and the grammar is \mr{not} LR(0)
  };

  % paths
  \draw[->] (i1.east |- i2.west) -- (i2.west) node[midway, above]{$E$}  ;
  \draw[->] (i1.mid east |- i3.west) -- (i3.west) node[midway, above]{$T$}  ;
  \draw[->] (i1) edge node[left]{$x$} (i5) ;
  \draw[->]
  ([xshift=4mm]i3.south) -- ([xshift=4mm]i4.north) node[midway,right] {$+$};
  \draw[->]
  ([xshift=-4mm]i4.north) -- ([xshift=-4mm]i3.south) node[midway,left] {$T$};
  \draw[->] (i4) edge node[above]{$E$} (i6);
  \draw[->] (i4.west |- i5.east) -- (i5.east) node[midway, above]{$x$};

\end{tikzpicture}
\section*{Grammar parse table with R/S conflict}
\begin{minipage}{.7\linewidth}
\begin{center}
\begin{tabular}{l|lll|ll}
  State &\multicolumn{3}{c}{Action$^1$} & \multicolumn{2}{c}{Goto$^2$}  \\
  \hline
  $I$ & + & $x$  &\$       & $T$ & $E$  \\
  \hline
   1  &   & $s5$ &         &$g3$& $g2$ \\
   2  &   &      &$a$      && \\
   3  &$s4,r2$&$r2$&$r2$   && \\
   4  &   &$s5$&           &$g3$&$g6$ \\
   5  &$r3$&$r3$&$r3$      &    & \\
   6  &$r1$&$r1$&$r1$      &    & \\
  \hline
  \multicolumn{6}{l}{\footnotesize 1: Action (s/r) only on terminals and \$}\\
  \hline
  \multicolumn{6}{l}{\footnotesize 2: Goto only on non-terminals}\\
  \hline
\end{tabular}
\end{center}
\end{minipage}
\begin{minipage}{.3\linewidth}
  {\small
    In state 3, on symbol +, there is a duplicate entry: The parser must shift
    into state 4 and also reduce by production 2. This is a conflict and indicates
    that the grammar is \mr{not} LR(0).
  }
\end{minipage}
\section*{Use FOLLOW to build SLR and remove S/R conflict}
\begin{enumerate}
\item in each state $I$, identify all the reduce items like $A\to \alpha\lrd$ (dot at the end) where $\alpha \in \Sigma$ (terminals plus \$).
  \begin{itemize}
  \item In $I_2$, $S\to E\lrd\$$, need  $\mfn{follow}(S)$
  \item In $I_3$, $E\to T\lrd$, need $\mfn{follow}(E)$
  \item In $I_5$, $T\to x\lrd$, need $\mfn{follow}(T)$
  \item In $I_6$, $E\to T + E\lrd$, need $\mfn{follow}(E)$
  \end{itemize}
\item for each reduce item above, compute \textsf{FOLLOW}(LHS)
  \begin{itemize}
  \item $\mfn{follow}(S) = \mset{\$}$ (just init the set to include \$)
  \item $\mfn{follow}(T) = \mfn{follow}(E) \cup \mfn{first}(+) = \mset{+,\$}$
  \item $\mfn{follow}(E) = \mset{\$}$
  \end{itemize}
\item for each token $X$ in each computed follow set above, put a reduce action $(I, X, A\to \alpha)$ (on lookahead $X$, reduce by rule $A\to \alpha$) in SLR table
  \begin{itemize}
  \item $(I_2, \$, S\to E\$)$, put $a$ ($r0$) at $(I_2, \$)$
  \item $(I_3, \$, E\to T)$, put $r2$ at $(I_3, \$)$
  \item $(I_5, +, T\to x)$, put $r3$ at $(I_5, +)$
  \item $(I_5, \$, T\to x)$, put $r3$ at $(I_5, \$)$
  \item $(I_6, \$, E\to T + E)$, put $r1$ at $(I_6, \$)$
  \end{itemize}
\end{enumerate}
\begin{center}
\begin{tabular}{l|lll|ll}
  State &\multicolumn{3}{c}{Action} & \multicolumn{2}{c}{Goto}  \\
  \hline
  $I$ & + & $x$  &\$       & $T$ & $E$  \\
  \hline
   1  &   & $s5$ &         &$g3$& $g2$ \\
   2  &   &      &$a$      &    & \\
   3  &$s4$&     &$r2$     && \\
   4  &   &$s5$&           &$g3$&$g6$ \\
   5  &$r3$&&$r3$          &    & \\
   6  &    &    &$r1$      &    & \\
  \hline
\end{tabular}
\end{center}

\section*{LR(1) items $(A\to \alpha\lrd\beta, x)$ with 3 parts; closure, goto}
\begin{itemize}
\item a grammar production: $A\to\ldots$
\item a right-hand-side position (represent by the \mr{dot} $\lrd$)
\item a lookahead symbol $(A\to \alpha\lrd\beta, \mo{x})$
\item The idea is that an item $(A\to\alpha\beta\lrd,x)$ indicates that
  \begin{itemize}
  \item the sequence $\alpha$ is on top of the stack
  \item at the head of the input is a string derivable from $\beta x$
  \end{itemize}
\item $\beta$ can be terminal or non-terminal, $x$ can be $\$$(\texttt{EOF})
\item In pseudo code, $X$ in item $(A\to \alpha\lrd X\beta, z)$ is non-terminal
\end{itemize}
\includegraphics*[width=\linewidth]{img/LR1_closure_goto}
\section*{Example: A LR(1) grammar and its parsing table}
\begin{minipage}{.5\linewidth}
\begin{align*}
  _0\quad S'&\to S\$  \\
  _1\quad S&\to V = E \\
  _2\quad S&\to E
\end{align*}
\end{minipage}
\begin{minipage}{.5\linewidth}
\begin{align*}
  _3\quad  E&\to V   \\
  _4\quad  V&\to x   \\
  _5\quad  V&\to *E
\end{align*}
\end{minipage}
\begin{itemize}
\item According to the pseudo code, lookahead for prod 0 is $\mfn{first}(\$)$. ``()'' around items are dropped for simplicity

  \begin{minipage}{\linewidth}
    \centering
    \begin{tikzpicture}
      \node (i0)[lrst] {
        \dm{(} $S' \to\lrd S$,\quad\mr{?} \dm{)}
      };
      \node (m1)[right=0.2mm of i0.north east,font=\small\bf] {\dm{1}};

      \node (i1)[lrst, right=20mm of i0.east] {
        $S' \to\lrd S$,\$
      };
      \node (m1)[right=1mm of i1.north east,font=\small\bf] {1};
      \draw[->] (i0) -- (i1);
    \end{tikzpicture}
  \end{minipage}
\item $S$ is non terminal, need to compute closure($I_1$):
  \begin{enumerate}
  \item Add items via closure rule; for each item, lookahead is unknown(?) for now, except prod 0
  \item For $S\to V = E$, $w=\mfn{first}(=E\$)$, need to add items $(V\to \lrd x, =)$ and $(V\to \lrd * E, =)$ to $I_1$
  \item For $S\to E$, $w=\mfn{first}(\$)$, $(E\to \lrd V, \$)$ already in $I_1$
  \item For $E\to V$, $w=\mfn{first}(\$) = \mset{\$}$, $(V\to \lrd\cdots, \$)$ in $I_1$
  \end{enumerate}
\end{itemize}
\begin{minipage}{\linewidth}
  \begin{tikzpicture}
    \node (i0)[lrst] {
      \begin{tabular}{ll}
        \dm{(} $S' \to\lrd S$   &, \$     \dm{)} \\
        \dm{(} $S \to\lrd V=E $ &, \mr{?} \dm{)} \\
        \dm{(} $S \to\lrd E   $ &, \mr{?} \dm{)} \\
        \dm{(} $E \to\lrd V   $ &, \mr{?} \dm{)} \\
        \dm{(} $V \to\lrd x   $ &, \mr{?} \dm{)} \\
        \dm{(} $V \to\lrd *E  $ &, \mr{?} \dm{)}
      \end{tabular}
    };
    \node (m0)[right=0.2mm of i0.north east,font=\small\bf] {\dm{1}};

    \node (i1)[lrst, right=6mm of i0.east] {
      \begin{tabular}{ll}
        $S' \to\lrd S$   &, \$ \\
        $S \to\lrd V=E $ &, \$  \\
        $S \to\lrd E   $ &, \$ \\
        $E \to\lrd V   $ &, \$ \\
        $V \to\lrd x   $ &, \$ $\gor$ = \\
        $V \to\lrd *E  $ &, \$ $\gor$ =
      \end{tabular}
    };
    \node (m1)[right=0.1mm of i1.north east,font=\small\bf] {1};
    \draw[->] (i0) -- (i1);
  \end{tikzpicture}
\end{minipage}
\begin{itemize}
\item Repeat the above steps to get the entire DFA diagram
\end{itemize}
\includegraphics*[width=\linewidth]{img/LR1_DFA}
\includegraphics*[width=\linewidth]{img/LR1_parsing_table}
\begin{itemize}
\item Wherever dot at a prod end, there is a reduce for that prod
\item Whenever dot is to the left of a terminal or non-terminal, there is a corresponding shift or goto
\end{itemize}


\section*{Example of LR(1) DFA and parsing table $a.k.a$ CLR(1)}
% \begin{minipage}{.5\linewidth}
\begin{align*}
  _1\quad S&\to XX \\
  _2\quad X&\to aX \\
  _3\quad X&\to b
\end{align*}
% \end{minipage}
% \begin{minipage}{.5\linewidth}
% \begin{align*}
%   _0\quad S'&\to S \\
%   _1\quad S&\to XX \\
%   _2\quad X&\to aX \\
%   _3\quad X&\to b
% \end{align*}
% \end{minipage}
\section*{Step 1: Add a new a start symbol (see above right)}
\begin{align*}
  _0\quad S'&\to S \\
  _1\quad S&\to XX \\
  _2\quad X&\to aX \\
  _3\quad X&\to b
\end{align*}
\section*{Step 2: Compute closures and gotos to build the DFA}
\begin{itemize}
\item For start state $I_0$, compute \textbf{closure}($I_0$) and \textbf{goto}($I_0$). For special prod $S'\to S\$$, its lookahead is \textbf{always} \$
\begin{minipage}{\linewidth}
  \centering
  \begin{tikzpicture}
    \node (i0) [lrst] {
      \begin{tabular}{ll}
        $S'\to \lrd S$ & ,\$
      \end{tabular}
    };
    \node (m0) [right=0.1mm of i0.north east,font=\footnotesize\bf] {0};
  \end{tikzpicture}
\end{minipage}
\item $S$ is non-terminal, need to add new items to $I_0$


\end{itemize}
\begin{minipage}{\linewidth}
  \begin{tikzpicture}
    \node (i0) [lrst] {
      \begin{tabular}{ll}
        $S'\to \lrd S$ & ,\$ \\
        $S\to \lrd XX$ & ,\mr{?}
      \end{tabular}
    };

    \node (i1) [lrst,right=2mm of i0.east] {
      \begin{tabular}{ll}
        $S'\to \lrd S$ & ,\$ \\
        $S\to \lrd XX$ & ,$\mfn{first}(\$)$
      \end{tabular}
    };

    \node (i2) [lrst,right=2mm of i1.east] {
      \begin{tabular}{ll}
        $S'\to \lrd S$ & ,\$ \\
        $S\to \lrd XX$ & ,\$
      \end{tabular}
    };
  \end{tikzpicture}
\end{minipage}
\begin{itemize}
\item $X$ is non-terminal, need to add new items
\end{itemize}
\begin{minipage}{\linewidth}
  \centering
  \begin{tikzpicture}
    \node (i1) [lrst,right=5mm of i0.east] {
      \begin{tabular}{ll}
        $S'\to \lrd S$ & ,\$ \\
        $S\to \lrd XX$ & ,$\mfn{first}(\$)$ \\
        $X\to \lrd aX$ & ,$\mfn{first}(X\$)$ \\
        $X\to \lrd b$ & ,$\mfn{first}(X\$)$
      \end{tabular}
    };
    \node (m1) [right=0.1mm of i1.north east,font=\footnotesize\bf] {\dm{0}};

    \node (i2) [lrst,right=5mm of i1.east] {
      \begin{tabular}{ll}
        $S'\to \lrd S$ & ,\$ \\
        $S\to \lrd XX$ & ,\$ \\
        $X\to \lrd aX$ & ,$a \gor b$ \\
        $X\to \lrd b$  & ,$a \gor b$
      \end{tabular}
    };
    \node (m2) [right=0.1mm of i2.north east,font=\footnotesize\bf] {0};

    \draw[->] (i1) -- (i2);
  \end{tikzpicture}
\end{minipage}
\begin{itemize}
\item when $I$ \textbf{goto} $I'$, item$_1$ in $I'$ copies LA from its prev in $I$
\end{itemize}
\begin{minipage}{\linewidth}
  \begin{tikzpicture}
    % I0
    \node (i0) [lrst] {
      \begin{tabular}{ll}
        $S'\to \lrd S$ & ,\$ \\
        $S\to \lrd XX$ & ,\$ \\
        $X\to \lrd aX$ & ,$a \gor b$ \\
        $X\to \lrd b$  & ,$a \gor b$
      \end{tabular}
    };
    \node (m0) [above=0.1mm of i0.north east,font=\footnotesize\bf] {0};

    %I1
    \node (i1) [lrst,right=10mm of i0.north east, anchor=north west] {
      \begin{tabular}{ll}
        $S'\to S\lrd$ & ,\$ \\
      \end{tabular}
    };
    \node (m1) [above=0.1mm of i1.north east,font=\footnotesize\bf] {1};

    %I2
    \node (i2) [lrst,below=3mm of i1] {
      \begin{tabular}{ll}
        $S\to X\lrd X$ & ,\$ \\
        $X \to \lrd aX$ & ,\$ \\
        $X \to \lrd b$  & ,\$
      \end{tabular}
    };
    \node (m2) [above=0.1mm of i2.north east,font=\footnotesize\bf] {2};
    \node (note2) [right=1.5mm of i2.east,text width=2.5cm, font=\footnotesize]{
      In new state $I_2$, item$_1$ copies over \mo{LA}s from its prev in $I_0$:
      $S\to X\lrd X$ in $I_2$ copies \mo{\$} from $S\to \lrd XX$ in $I_0$\\
    };

    %I3
    \node (i3) [lrst,below=8mm of i0.south west, anchor=north west] {
      \begin{tabular}{ll}
        $X \to b\lrd$  & ,$a \gor b$
      \end{tabular}
    };
    \node (m3) [above=0.1mm of i3.north west,font=\footnotesize\bf] {3};

    %I4
    \node (i4) [lrst,below left=8mm of i2.south,anchor=north] {
      \begin{tabular}{ll}
        $X \to a\lrd X$  & ,$a \gor b$ \\
        $X \to \lrd aX$  & ,$a \gor b$ \\
        $X \to \lrd b$  & ,$a \gor b$
      \end{tabular}
    };
    \node (m4) [above=0.1mm of i4.north east,font=\footnotesize\bf] {4};

    \node(note0)[below=2mm of i3, text width=3.2cm,font=\footnotesize] {
      In $I_4$, LAs of item$_2$ and item$_3$ are computed as $\mfn{first}(X\$) = a\gor b$
    };

    \draw[->] (i0.south -| i3.north) -- (i3.north) node[midway,left]{$b$};
    \draw[->] (i0.south) -- (i4.north) node[midway,right]{$a$};
    \draw[->] (i0.east) -- (i2.west) node[midway,above]{$X$};
    \draw[->] (i0.east |- i1.west) -- (i1.west) node[midway,above]{$S$};
  \end{tikzpicture}
\end{minipage}
\begin{itemize}
\item \textbf{goto}($I_2, a$) does \mr{\emph{not}} lead to $I_4$,  but lead to $I_6$
\end{itemize}
\begin{tikzpicture}
  % I2
  \node (i2) [lrst] {
    \begin{tabular}{ll}
      $S\to X\lrd X$ & ,\$ \\
      $X \to \lrd aX$ & ,\$ \\
      $X \to \lrd b$  & ,\$
    \end{tabular}
  };
  \node (m2) [above=0.1mm of i2.north east,font=\footnotesize\bf] {2};

  % I5
  \node (i5) [lrst,right=8mm of i2.north east,anchor=north west] {
    \begin{tabular}{ll}
      $S \to XX\lrd$  & ,\$
    \end{tabular}
  };
  \node (m5) [above=0.1mm of i5.north east,font=\footnotesize\bf] {5};

  % I6
  \node (i6) [lrst,below=5mm of i5] {
    \begin{tabular}{ll}
      $X \to a\lrd X$  & ,\$ \\
      $X \to \lrd aX$  & ,\$ \\
      $S \to \lrd b$   & ,\$ \\
    \end{tabular}
  };
  \node (m6) [above=0.1mm of i6.north east,font=\footnotesize\bf] {6};
  \node (note6)[
  right=4mm of m5.north east,anchor=north west,text width=2.8cm,font=\footnotesize] {
    In $I_2$, shift $a$ leads to $X\to a\lrd X,\$$, due to LA \$ (\mr{not} $a\gor b$), need a new state $I_6$ with all below items\\
    $(X\to \cdots, \mfn{first}(\$))$ added to $I_6$
  };

  % I7
  \node (i7) [lrst,below=5mm of i2] {
    \begin{tabular}{ll}
      $X \to \lrd b$  & ,\$
    \end{tabular}
  };
  \node (m7) [below=0.1mm of i7.south east,font=\footnotesize\bf] {7};

  % I8
  \node (i8) [lrst,right=6mm of i6.south east,anchor=south west] {
    \begin{tabular}{ll}
      $X \to aX\lrd $  & ,\$
    \end{tabular}
  };
  \node (m8) [below=0.1mm of i8.south east,font=\footnotesize\bf] {8};
  \draw[->] (i2.east |- i5.west) -- (i5.west) node[midway, above]{$X$};
  \draw[->] (i2.east) -- (i6.west) node[midway, above]{$a$};
  \draw[->] (i2.south) -- (i7.north) node[midway, left]{$b$};
  \draw[->] (i6.west |- i7.east) -- (i7.east) node[midway, above]{$b$};
  \draw[->] (i6.east |- i8.west) -- (i8.west) node[midway, above]{$X$};
  \draw[->]
  ([xshift=4mm]i6.south west) to [bend right=60] node[above]{$a$}
  ([xshift=-4mm]i6.south east);
\end{tikzpicture}
\begin{itemize}
\item \mb{complete DFA} is therefore constructed as
\end{itemize}
\begin{minipage}{\linewidth}
  \begin{tikzpicture}
    % I0
    \node (i0) [lrst] {
      \begin{tabular}{ll}
        $S'\to \lrd S$ & ,\$ \\
        $S\to \lrd XX$ & ,\$ \\
        $X\to \lrd aX$ & ,$a \gor b$ \\
        $X\to \lrd b$  & ,$a \gor b$
      \end{tabular}
    };
    \node (m0) [above=0.1mm of i0.north east,font=\footnotesize\bf] {0};

    % I1
    \node (i1) [lrst,right=10mm of i0.north east, anchor=north west] {
      \begin{tabular}{ll}
        $S'\to S\lrd$ & ,\$ \\
      \end{tabular}
    };
    \node (m1) [above=0.1mm of i1.north east,font=\footnotesize\bf] {1};

    % I2
    \node (i2) [lrst,below=3mm of i1] {
      \begin{tabular}{ll}
        $S\to X\lrd X$ & ,\$ \\
        $X \to \lrd aX$ & ,\$ \\
        $X \to \lrd b$  & ,\$
      \end{tabular}
    };
    \node (m2) [above=0.1mm of i2.north east,font=\footnotesize\bf] {2};

    % I3
    \node (i3) [lrst,below=8mm of i0.south west, anchor=north west] {
      \begin{tabular}{ll}
        $X \to b\lrd$  & ,$a \gor b$
      \end{tabular}
    };
    \node (m3) [above=0.1mm of i3.north west,font=\footnotesize\bf] {3};

    % I4
    \node (i4) [lrst,below=4mm of i2.south,anchor=north] {
      \begin{tabular}{ll}
        $X \to a\lrd X$  & ,$a \gor b$ \\
        $X \to \lrd aX$  & ,$a \gor b$ \\
        $X \to \lrd b$  & ,$a \gor b$
      \end{tabular}
    };
    \node (m4) [below=0.1mm of i4.south east,font=\footnotesize\bf] {4};

    % I5
    \node (i5) [lrst,right=8mm of i1.north east,anchor=north west] {
      \begin{tabular}{ll}
        $S \to XX\lrd$  & ,\$
      \end{tabular}
    };
    \node (m5) [above=0.1mm of i5.north east,font=\footnotesize\bf] {5};

    % I6
    \node (i6) [lrst,below=5mm of i5] {
      \begin{tabular}{ll}
        $X \to a\lrd X$  & ,\$ \\
        $X \to \lrd aX$  & ,\$ \\
        $S \to \lrd b$   & ,\$ \\
      \end{tabular}
    };
    \node (m6) [above=0.1mm of i6.north east,font=\footnotesize\bf] {6};

    % I8
    \node (i8) [lrst,below=6mm of i6.south,anchor=north] {
      \begin{tabular}{ll}
        $X \to aX\lrd $  & ,\$
      \end{tabular}
    };
    \node (m8) [above=0.1mm of i8.north east,font=\footnotesize\bf] {8};

    % I7
    \node (i7) [lrst,below=3mm of i8.south, anchor=north] {
      \begin{tabular}{ll}
        $X \to \lrd b$  & ,\$
      \end{tabular}
    };
    \node (m7) [below=0.1mm of i7.south east,font=\footnotesize\bf] {7};

    %I9
    \node (i9) [lrst,left=8mm of i4.south west, anchor=south east] {
      \begin{tabular}{ll}
        $X \to aX\lrd $  & ,$a\gor b$
      \end{tabular}
    };
    \node (m9) [below=0.1mm of i9.south west, font=\footnotesize\bf] {9};

    \draw[->] (i0.south -| i3.north) -- (i3.north) node[midway,left]{$b$};
    \draw[->] (i0.south east) -- (i4.north west) node[midway,right]{$a$};
    \draw[->] (i0.east) -- (i2.west) node[midway,above]{$X$};
    \draw[->] (i0.east |- i1.west) -- (i1.west) node[midway,above]{$S$};
    \draw[->] (i2.east) -- (i5.south west) node[midway, left,xshift=1mm]{$X$};
    \draw[->] (i2.east) -- (i6.west) node[midway, above]{$a$};
    \draw[->] (i2.south east) -- (i7.west) node[near start, right]{$b$};
    \draw[->] (i6.south east) to [bend right=-40] node[right]{$b$} (i7.east);
    \draw[->] (i6) -- (i8) node[midway, left]{$X$};
    \draw[->]
    ([yshift=-4mm]i6.north east) to [bend right=-45] node[right]{$a$}
    ([yshift=4mm]i6.south east);
    \draw[->] (i4.west) -- (i3.east) node[midway,right]{$b$};
    \draw[->]
    ([xshift=-4mm]i4.south east) to [bend right=-45] node[above]{$a$}
    ([xshift=4mm]i4.south west);
    \draw[->] (i4.west |- i9.east) -- (i9.east) node[midway,below]{$X$};
  \end{tikzpicture}
\end{minipage}
\section*{Step 3: Draw the LR(1) parsing table}
\begin{enumerate}
\item Start with $I_0$, see where it goto on what symbol:
\item For each \mb{terminal} $t$ , shift $sn$ at $(I,t)$,  $n$ is target state
\item For each \mb{non-terminal} $X$, goto $gn$ at $(I,t)$, $n$ is target state
\item For state with a single prod $A\to \cdots \lrd$, (dot at the end)
  \begin{itemize}
  \item if item is the special one (like in $I_1$), $a$ at ($I, \$$)
  \item if item has LA \$, reduce $n$ at $(I, \$)$, $n$ is the prod number
  \item if item has LA $t_1,t_2,\cdots$, reduce $n$ at $(I, t_i)$ for each $t$
  \end{itemize}
\end{enumerate}
\begin{minipage}{.5\linewidth}
  \begin{tabular}{l|lll|ll}
    $I$ & $a$  & $b$  & \$  & $S$  & $X$ \\
    \hline
    $0$ & $s4$ & $s3$ &     & $g1$ & $g2$ \\
    $1$ &      &      & $a$ &      &      \\
    $2$ & $s6$ & $s7$ &     &      & $g5$ \\
    $3$ & $r3$ & $r3$ &   &   &  \\
    $4$ & $s4$  & $s3$  &   &   & $g9$ \\
    $5$ &   &   & $r1$  &   &  \\
    $6$ & $s6$  & $s7$  &   &   & $g8$ \\
    $7$ &   &   & $r3$  &   &  \\
    $8$ &   &   & $r2$  &   &  \\
    $9$ & $r2$  & $r2$   & &   &  \\
    \hline
  \end{tabular}
\end{minipage}
\begin{minipage}{.5\linewidth}
  \begin{tabular}{l|lll|ll}
    $I$ & $a$  & $b$  & \$  & $S$  & $X$ \\
    \hline
    $0$ & $s4$ & $s3$ &     & $g1$ & $g2$ \\
    $1$ &      &      & $a$ &      &      \\
    $2$ & $s6$ & $s7$ &  &      & $g5$ \\
    $37$ & $r3$ & $r3$ & $r3$  &   &  \\
    $46$ & $s46$  & $s37$  &   &   & $g89$ \\
    $5$ &   &   & $r1$  &   &  \\
    $89$ & $r2$  & $r2$  & $r2$ &   &  \\
    \hline
    \multicolumn{6}{l}{or just leave 1 state in a pair}\\
    \multicolumn{6}{l}{e.g. in $(I_4, I_6)$, merge to $I_4$}  \\
    \hline
  \end{tabular}
\end{minipage}
\section*{LALR parsing table (see the right table in middle col)}
\begin{itemize}
\item In the LR(1) DFA, there're several states with same productions but diff LAs: $I_3$ and $I_7$; $I_8$ and $I_9$; $I_6$ and $I_4$
\item merge such similar states into one: $I_{37}$, $I_{89}$, $I_{46}$
\item update shifts so they shift to the merge states: e.g. $s4\to s46$
\item \mb{only merge} \mo{shift} actions and \mo{goto}s can be merged
\item \mb{copy} the reduce actions to the merge state (\mo{iff} no conflict)
\item if any S/R or R/R conflicts, we can't build LALR table
\end{itemize}
\section*{Patterns for computing LAs of an item in LR(1) grammar}
In a state $I$, given an item of the below form (parens dropped):
  \[
    A\to \alpha\lrd X\beta, z
  \]
\begin{itemize}
\item where $X$ is \mb{non-terminal} and $z$ is LAs (might be $t_1\gor t_2\gor\ldots$)
  \begin{itemize}[leftmargin=3em]
  \item if $\beta$ is non empty (terminal or non-terminal)
    \begin{enumerate}
    \item compute $\mfn{first}(\beta z)$ as a set
    \item Add all $X\to\lrd \ldots, w$ to $I$, where $w \in \mfn{first}(\beta z)$

      \begin{minipage}{.3\linewidth}
        \begin{align*}
          S&\to \lrd V = E &&,\$ \\
          V&\to \lrd x     &&,\mr{?}  \\
          V&\to \lrd *E    &&,\mr{?}
        \end{align*}
      \end{minipage}
      \begin{minipage}{.7\linewidth}
        \begin{align*}
          S&\to \lrd V = E& &,\$ \\
          V&\to \lrd x    & &,\mfn{first}(=E)  \\
          V&\to \lrd *E   & &,\mfn{first}(=E)
        \end{align*}
      \end{minipage}
      \begin{minipage}{.2\linewidth}
        \begin{align*}
          S&\to \lrd XX &&,\$ \\
          X&\to \lrd aX &&,\mr{?}  \\
          X&\to \lrd b  &&,\mr{?}
        \end{align*}
      \end{minipage}
      \begin{minipage}{.8\linewidth}
        \begin{align*}
          S&\to \lrd XX    &&,\$ \\
          X&\to \lrd aX    &&,\mfn{first}((a\gor b)\cup\$)  \\
          X&\to \lrd b     &&,\mfn{first}((a\gor b)\cup\$)
        \end{align*}
      \end{minipage}
      \item $\mfn{first}((a\gor b)\cup\$) = a\gor b$ because \$ is considered only when $\mfn{first}(a\gor b)$ is empty, which is \mr{false}
    \end{enumerate}
  \item if $\beta$ is empty \textbf{BUT} $z$ is \mr{not} \$
    \begin{enumerate}
    \item compute $\mfn{first}(z)$ as a set ($z$ might be $t_1\gor t_2\gor\ldots$)
    \item Add all $X\to\lrd \ldots, w$ to $I$, where $w \in \mfn{first}(z)$

      \begin{minipage}{.3\linewidth}
        \begin{align*}
          X&\to a\lrd X & &, a\gor b \\
          X&\to \lrd aX & &,\mr{?}  \\
          V&\to \lrd b  & &,\mr{?}
        \end{align*}
      \end{minipage}
      \begin{minipage}{.7\linewidth}
        \begin{align*}
          X&\to a\lrd X & &, a\gor b \\
          X&\to \lrd aX & &,\mfn{first}(a\gor b) = a\gor b \\
          V&\to \lrd b  & &,\mfn{first}(a\gor b) = a\gor b
        \end{align*}
      \end{minipage}
    \end{enumerate}
  \item if $\beta$ is empty \textbf{AND} $z$ \mr{is} \$
    \begin{enumerate}
    \item compute $\mfn{first}(\$)$
    \item Add all $X\to\lrd \ldots, w$ to $I$, where $w =\$$

      \begin{minipage}{.3\linewidth}
        \begin{align*}
          X&\to X\lrd X & &, \$ \\
          X&\to \lrd aX & &,\mr{?}  \\
          V&\to \lrd b  & &,\mr{?}
        \end{align*}
      \end{minipage}
      \begin{minipage}{.7\linewidth}
        \begin{align*}
          X&\to a\lrd X & &, \$ \\
          X&\to \lrd aX & &,\mfn{first}(\$) = \$ \\
          V&\to \lrd b  & &,\mfn{first}(\$) = \$
        \end{align*}
      \end{minipage}
    \end{enumerate}
  \end{itemize}
\item when $X$ is \mb{terminal}, this may lead to a new state with item $A\to \alpha X\lrd \beta, z$, where $z$ is simply copied over
\end{itemize}
\end{multicols*}
\end{document}
