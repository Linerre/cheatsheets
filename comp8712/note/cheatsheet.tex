\documentclass[10pt,a4paper,landscape]{article}
% -- Layout ----
\usepackage[top=0.6cm, bottom=0.6cm, left=0.5cm, right=0.5cm, landscape]{geometry}

% -- Titles ----
\usepackage[
  tiny,                     % text size title
  compact                   % reduce vertical space before/after title
]{titlesec}
% \titlespacing*
\titleformat{\section}{\normalfont\small\bfseries}{\thesection}{0em}{} % Remove space before and after section titles
\titleformat{\subsection}{\normalfont\small\bfseries}{\thesubsection}{0em}{} % Remove space before and after subsection titles
\titlespacing*{\section}{0pt}{0pt}{0pt} % Remove space before/after section titles
\titlespacing*{\subsection}{0pt}{0pt}{0pt} % Remove space before/after subsec titles

% -- Colors ----
\usepackage[dvipsnames]{xcolor}
\definecolor{dmm}{RGB}{192,192,192} % Define a custom dimmed text color
\definecolor{cmt}{RGB}{61,123,123}

% -- Math ------
\usepackage{mathtools}
\usepackage{amssymb}
\usepackage{turnstile}%better vdash

% -- Lists -----
\usepackage[inline]{enumitem}
\setlist{noitemsep}% Remove vspace between items
% Set vspace before and after  list environments as well as the left margin
\setlist[itemize,1]{leftmargin=.6em,labelindent=0pt,labelsep=2pt,
  topsep=1pt,partopsep=1pt}
\setlist[enumerate,1]{leftmargin=1em,labelindent=0pt,labelsep=2pt,
  topsep=1pt,partopsep=1pt}
\setlist[itemize,2]{leftmargin=.3em,labelindent=1pt,topsep=1pt,partopsep=1pt}
\setlist[enumerate,2]{leftmargin=0.2em,labelindent=1pt,topsep=1pt,partopsep=1pt}
\setlist[description]{labelwidth=\linewidth,font=\small\bfseries,leftmargin=1em,topsep=1pt,partopsep=1pt}
% -- Code listing ---
\usepackage{listings}
\lstset{
  aboveskip=3pt,
  belowskip=3pt,
  basicstyle=\small\ttfamily,
  breaklines=true,
  % commentstyle=\upshape\ttfamily,
  captionpos=b,
  commentstyle=\color{cmt},
  frame=single,
  keepspaces=false,
  keywordstyle=\bfseries,
  showspaces=false,
  showstringspaces=false,
  showtabs=false,
  tabsize=2,
}

% Parse Trees
\usepackage{tikz}
\usetikzlibrary{ arrows, automata, bbox, calc, positioning,  decorations.pathmorphing, decorations.pathreplacing, decorations.shapes, }
\tikzset{
% ->, % makes the edges directed
>=stealth', % makes the arrow heads bold
node distance=1cm, % specifies minimum distance between two nodes
% small/.style={},
every state/.style={thick}, % sets the properties for each ’state’ node
every node/.style={inner sep=1pt},
initial text=start, % sets the text that appears on the start arrow
}

% Place a figure env right here via [H] option
\usepackage{float}

% Side by side figure
\usepackage{subcaption}
% \usepackage{caption}
% \captionsetup{belowskip=0pt, aboveskip=0pt}


% -- Multi-Col layout --
\usepackage{multicol}

% No indentation
\setlength\parindent{0pt}
\setlength\abovedisplayskip{-5pt}
\setlength\belowdisplayskip{-5pt}
\setlength\abovedisplayshortskip{-4pt}
\setlength\belowdisplayshortskip{-4pt}
\newcommand{\gor}{\;|\;}
\newcommand{\num}{\texttt{\#}~}
\renewcommand{\arraystretch}{1.2}


\begin{document}
% Suppress page number for all pages
\pagestyle{empty}

% Each section goes into this env
\begin{multicols*}{3}

\section*{Object Types (declaration and inheritance)}
\begin{minipage}{.5\linewidth}
\begin{lstlisting}[language=c]
// S inherits from `Root'
type S = object {}
// object with fields only
type Shape = object {
  name: Text;
  border: bool
// optional ; for last field
}
// access fields like java
Rect.name; Rect.area;
// obj type with supertype
type Rect = Shape object {
  width: int;
  height: int;
}
\end{lstlisting}
\end{minipage}
\begin{minipage}{.5\linewidth}
  \flushleft
  \begin{itemize}
  \item \texttt{Root} is pre-declared, traced, and with no fields
  \item \texttt{object} is like \verb|^struct{}|
  \item Object can be \texttt{null}
  \item Objects \mr{cannot} be dereferenced
  \item To copy data record of one obj into another, the fields must be assigned one by one
  \item a field/method in subtype \mb{masks} any field/method with the same name in supertype
  \item Object assignment is \mb{reference assignment}
  \end{itemize}
\end{minipage}

\begin{minipage}{\linewidth}
\begin{lstlisting}[language=c]
// object type with methods
type Rect = Shape object {
  name: Text; width, height: int;
  useless();           // equivalent to useless() = null;
  area(): int = Area; // identifer sig = proc;
  // `area' takes no args, calls `Area' and returns int
}
// proc declared before/after obj def and must be global
proc Area(self: Rect): int {
  return self.width * self.height;
}
\end{lstlisting}
\end{minipage}
\begin{itemize}
\item In \mo{\texttt{= proc}}, \texttt{proc} must be a top-level procedure constant
\item If \texttt{= proc} \mo{omitted} (\texttt{useless();}), \mo{\texttt{= null}} is assigned
\item If \texttt{proc} is \mb{non-null}, its 1st param must have mode \texttt{val} and type some supertype of \texttt{T} (\texttt{Shape} in the above example)
\item For proc \texttt{Area}, dropping its 1st param (\texttt{self: Rect}) results in the exactly same sig as \texttt{area()} in the obj definition
\end{itemize}
\begin{minipage}{\linewidth}
\begin{lstlisting}[language=c]
type
  A = object { a: int; p() },
  AB = A object { b: int };
// since every `AB' is an `A', so `AB' has method `p'
// candidates for `p' of both `A' and `AB'
proc Pa(self: A) { ... }
proc Pab(self: AB) { ... }
\end{lstlisting}
\end{minipage}
\begin{minipage}{\linewidth}
\begin{lstlisting}[language=c]
// ok: `p' expects/accepts `AB' obj as its 1st arg
type T1 = AB object { p = Pab };
// ok: `p' expects `A' obj as its 1st arg
type T2 = A object { p = Pa };
// ok: `Pa' expects `A', but each `AB' is also an `A'
type T3 = AB object { p = Pa };
// static error: because not every `A' is an `AB'
type T4 = A object { p = Pab };
\end{lstlisting}
\end{minipage}
\section*{Object method overriding and overloading}
\begin{lstlisting}[language=c]
type
  A = object { m() = P },   // assume P(self: A)
  B = A object { m = Q },   // assume Q(self: A)
  C = A object { m() = Q };
// A,B,C are alloc-ed by `New'; a,b,c are refs -> them
var a = New(A), b = New(B), c = New(C);
a.m() // invoke P(a) where a is an instance of A
b.m() // invoke Q(b) where b is an instance of B and A
c.m() // invoke Q(c) where c is an instance of C and A

a = b; a.m(); // activates Q(b) as b's m overrides a's
a = c; a.m(); // activates P(c), c's 1st method
// to visit c's 2dn method, use c.m() <- overloading
\end{lstlisting}
\begin{itemize}
\item \texttt{c} method suite has \mb{2} methods: \texttt{Q(c)}, \texttt{P(c)} $\to$ \mo{overloading}
\item \texttt{b} method suite has \mb{1} methods: \texttt{Q(b)}$\to$ \mo{overriding}
\end{itemize}
\includegraphics*[width=\linewidth]{img/overloadevsoverride}
\end{multicols*}
\end{document}
