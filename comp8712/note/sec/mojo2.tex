\documentclass[10pt,a4paper,landscape]{article}
% -- Layout ----
\usepackage[top=0.6cm, bottom=0.6cm, left=0.5cm, right=0.5cm, landscape]{geometry}

% -- Titles ----
\usepackage[
  tiny,                     % text size title
  compact                   % reduce vertical space before/after title
]{titlesec}
% \titlespacing*
\titleformat{\section}{\normalfont\small\bfseries}{\thesection}{0em}{} % Remove space before and after section titles
\titleformat{\subsection}{\normalfont\small\bfseries}{\thesubsection}{0em}{} % Remove space before and after subsection titles
\titlespacing*{\section}{0pt}{0pt}{0pt} % Remove space before/after section titles
\titlespacing*{\subsection}{0pt}{0pt}{0pt} % Remove space before/after subsec titles

% -- Colors ----
\usepackage[dvipsnames]{xcolor}
\definecolor{dmm}{RGB}{192,192,192} % Define a custom dimmed text color
\definecolor{cmt}{RGB}{61,123,123}

% -- Math ------
\usepackage{mathtools}
\usepackage{amssymb}
\usepackage{turnstile}%better vdash

% -- Lists -----
\usepackage[inline]{enumitem}
\setlist{noitemsep}% Remove vspace between items
% Set vspace before and after  list environments as well as the left margin
\setlist[itemize,1]{leftmargin=.6em,labelindent=0pt,labelsep=2pt,
  topsep=1pt,partopsep=1pt}
\setlist[enumerate,1]{leftmargin=1em,labelindent=0pt,labelsep=2pt,
  topsep=1pt,partopsep=1pt}
\setlist[itemize,2]{leftmargin=.3em,labelindent=1pt,topsep=1pt,partopsep=1pt}
\setlist[enumerate,2]{leftmargin=0.2em,labelindent=1pt,topsep=1pt,partopsep=1pt}
\setlist[description]{labelwidth=\linewidth,font=\small\bfseries,leftmargin=1em,topsep=1pt,partopsep=1pt}
% -- Code listing ---
\usepackage{listings}
\lstset{
  aboveskip=3pt,
  belowskip=3pt,
  basicstyle=\small\ttfamily,
  breaklines=true,
  % commentstyle=\upshape\ttfamily,
  captionpos=b,
  commentstyle=\color{cmt},
  columns=flexible,
  frame=single,
  keepspaces=false,
  keywordstyle=\bfseries,
  showspaces=false,
  showstringspaces=false,
  showtabs=false,
  tabsize=2,
}

% Parse Trees
\usepackage{tikz}
\usetikzlibrary{ arrows, automata, bbox, calc, positioning, tikzmark, decorations.pathmorphing, decorations.pathreplacing, decorations.shapes, }
\tikzset{
  >=stealth',
  node distance=1cm,
  recstate/.style={
    circle,draw=blue!50,fill=blue!20,thick,font=\small\sffamily,rounded corners=3pt,
    minimum size=1cm,inner sep=1pt
  },
  ivp/.style={draw,->,auto,font=\small\sffamily,bend angle=60},
  msi/.style={draw=Brown,->,auto,font=\small\sffamily,bend angle=80},
  msibs/.style={draw=RoyalBlue,->,auto,font=\small\sffamily,bend angle=80},
  msinl/.style={font=\footnotesize\sffamily}, %msi node label
  every edge/.style={draw,auto,font=\small\sffamily},
  every loop/.style={looseness=4},
  initial text=start,initial where=right
}

% Place a figure env right here via [H] option
\usepackage{float}

% Side by side figure
\usepackage{subcaption}
% \usepackage{caption}
% \captionsetup{belowskip=0pt, aboveskip=0pt}
\usepackage{pifont}

% -- Multi-Col layout --
\usepackage{multicol}

% No indentation
\setlength\parindent{0pt}
\setlength\abovedisplayskip{-5pt}
\setlength\belowdisplayskip{-5pt}
\setlength\abovedisplayshortskip{-4pt}
\setlength\belowdisplayshortskip{-4pt}
\setlength\tabcolsep{5pt}
\newcommand{\gor}{\;|\;}
\newcommand{\num}{\texttt{\#}~}
\newcommand{\pro}[1]{\textcolor{Brown}{#1}}
\newcommand{\bus}[1]{\textcolor{RoyalBlue}{#1}}
\renewcommand{\arraystretch}{1.2}


\begin{document}
% Suppress page number for all pages
\pagestyle{empty}

% Each section goes into this env
\begin{multicols*}{3}
\section*{Ordinal Types: enumeration}
\begin{lstlisting}[language=c]
type T = enum {id_1, id_2, ..., id_n};
type E = enum {}; // The empty enumeration is allowed
\end{lstlisting}
\begin{itemize}
\item \texttt{id\_i} are distinct identifiers
\item The type \texttt{T} is an ordered set of \texttt{n} values
\item the expr \texttt{T.id\_i} denotes the \mb{ith} value of the type in \mo{increasing} order
\item Integers and enumeration elements are collectively called ordinal values
\item the base type of an ordinal value \texttt{v} is int if \texttt{v} is an integer
\item otherwise it is the unique enumeration type that contains \texttt{v}
\end{itemize}
\section*{Ordinal Types: subrange}
\begin{lstlisting}[language=c]
// Lo and Hi are all ordinal Val's with same base type
type T = Lo..Hi; type U = 1..100; // 1 to 100, inclusive
type E = 100..20; // E is an empty subrange
\end{lstlisting}
\begin{itemize}
\item The values of \texttt{T} are all the values from \texttt{Lo} to \texttt{Hi} \mo{inclusive}
\item \texttt{Lo} and \texttt{Hi} must be constant expressions
\item If \texttt{Lo} exceeds \texttt{Hi}, the subrange is empty
\end{itemize}
\section*{Pre-declared Ordinal Types }
\begin{itemize}
\item \texttt{int}: All integers naturally represented by the impl.
\item \texttt{byte}: Behaves just like the subrange \texttt{0..(1<<8)-1}.
\item \texttt{bool}: The enumeration enum \texttt{{false, true}}.
\item \texttt{char}: An enumeration containing at least 256 elements.
\end{itemize}
Each distinct enumeration type introduces a new collection of values, but a subrange type reuses the values from the underlying type. For example:
\begin{lstlisting}[language=c]
type
  T1 = enum {A, B, C}; // T1 and T2 are the same type
  T2 = enum {A, B, C}; // e.g. T1.C == T2.C
  U1 = T1.A..T1.C; // U1 and U2 are also the same type
  U2 = T1.A..T2.C; // T1 and U1 are distinct
  V = enum {A, B}; // V.A, V.B NOT related to T1.A, T1.B
\end{lstlisting}
\begin{itemize}
\item \texttt{T1} and \texttt{U1} contain same values but are distinct
\item the expanded definition of \texttt{T1} is an enumeration
\item the expanded definition of \texttt{U1} is a subrange
\end{itemize}
\section*{Reference Types}
\begin{lstlisting}[language=c]
type T = ^Type; // ^ is the ref operator
\end{lstlisting}
\begin{itemize}
\item A reference value is either \texttt{null} or the address of a variable, called the \mb{referent}
\item A reference type is either traced or untraced
\item When all traced refs to a piece of allocated storage are gone, the implementation reclaims the storage
\item A general type is traced if
  \begin{enumerate*}[label={\alph*.},font={\color{red!50!black}\bfseries}]
  \item it is a traced reference type
  \item a record type, any of whose field types is traced; or
  \item an array type whose element type is traced
  \end{enumerate*}
\end{itemize}
\section*{Pre-declared reference types}
\begin{itemize}
\item \texttt{Refany}: Contains all traced references.
\item \texttt{Address} : Contains all untraced references.
\item \texttt{Null}: Contains only \texttt{null}
\end{itemize}
\section*{Procedure Types: \texttt{null} or a triple consisting of the below}
\begin{enumerate}
\item the \mo{body}, which is a block,
\item the \mo{signature}, which specifies the procedure’s formal arguments. and result type
\item the \mo{environment}, which is the scope with respect to which variable names in the body will be interpreted.
\end{enumerate}
\begin{itemize}
\item \mb{function} proc: a procedure that returns a result
\item \mb{top-level} proc: declared in outermost scope of a module
\item \mb{local} proc: can be passed as a parameter but not assigned, since in a stack implementation a local proc becomes invalid when the frame for the proc containing it is popped
\item \mb{procedure constant}: an identifier declared as a proc, as opposed to a proc var which is a var declared with a proc type
\end{itemize}
\begin{lstlisting}[language=c]
// use `proc' keyword to define a procedure `p1'
proc p1(arg1: Type1, arg2: Type2): Type3 {
  // proc body that implements something
}
// declare a proc type `p2'
type T2 = proc p2(arg1: Type1, arg2: Type2);
// same-type args can use shorthand
type T3 = proc p3(arg1, arg2: SomeType): OtherType;
// each arg can have a mode of `val', `var', `readonly'
type T4 = proc p4(val arg1: Type1, var arg2: Type2);
// arg2 has the default mode: `val'
type T5 = proc p5(readonly arg1: Type1, arg2: Type2);
// mode has shorthand form too:
type T6 = proc p6(var arg1, arg2, arg3: SomeType);
{
   // use the above declared procs in this block
}
\end{lstlisting}
\texttt{signature1} \mo{covers} \texttt{signature2} if:
\begin{itemize}
\item They have the same number of params, and corresponding paras have the same type and mode
\item They have the same result type, or neither has a result type
\end{itemize}
\section*{Subtyping rules in general}
\begin{itemize}
\item \texttt{T<:U} means \texttt{T} is a \mb{subtype} of \texttt{U} and \texttt{U} is a \mb{supertype} of \texttt{T}
\item If \texttt{T<:U}, then every value of type T is also a value of type U. The converse does not hold
\end{itemize}
\section*{Ordinal subtyping}
For ordinal type \texttt{T} and \texttt{U}, we have \texttt{T<:U} if they have the same base type and every member of \texttt{T} is a member of \texttt{U}. That is, subtyping on ordinal types reflects the subset relation on the value sets
\section*{Array subtyping}
Array type \texttt{A} is a subtype of array type \texttt{B} if
\begin{itemize}
\item they have the same ultimate element type
\item the same number of dimensions
\item for each dimension:
  \begin{itemize}[label=$\ast$]
  \item either both are open (as in the first m dimensions), or
  \item A is fixed and B is open (as in the next n dimensions)
  \item they are both fixed and have the same size (as in the last p dimensions)
  \end{itemize}
\end{itemize}
\begin{lstlisting}[language=c]
// in 1st m dims, both A and B are open
// in next n dims, A is fixed and B is open
// in the last dim, both A and B are fixed
   ([])^m  [j_1] ... [j_n] [k_1] ... [k_p] A
<: ([])^m  ([])^n            [i_1] ... [i_p] B

if Number(i_i) == Number(k_i)  for i = 1, ..., p
\end{lstlisting}
\section*{Other subtyping rules}
\begin{itemize}
\item \texttt{Refany} contains all traced refs, and \texttt{null} is a member of every ref type
\begin{lstlisting}[language=c]
Null <: ^T <: Refany
\end{lstlisting}
\item \texttt{null} is a member of every procedure type
\begin{lstlisting}[language=c]
Null <: (A): R for any A and R
\end{lstlisting}
\item for procedure types, \texttt{T<:U} if they are the same except for param names
\begin{lstlisting}[language=c]
(A): Q <: (B): R // if sig (B): R covers sig (A): Q
\end{lstlisting}
\item every object is a reference, \texttt{null} is a member of every object type, and every subtype is included in its supertype
\begin{lstlisting}[language=c]
Root <: Refany
Null <: T object { ... } <: T
\end{lstlisting}
\item \texttt{<:} is \mo{reflexive} and \mo{transitive}
\begin{lstlisting}[language=c]
T <: T  for all T
T <: U  and  U <: V  implies  T <: V  for all T, U, V.
\end{lstlisting}
\item \texttt{T<:U} and \texttt{U<:T} \mr{\emph{does not}} imply that T and U are the same
\item The language predeclares the below type:
\begin{lstlisting}[language=c]
Text <: Refany
\end{lstlisting}
\end{itemize}

\end{multicols*}
\end{document}
