\section*{Q1: if a given grammar $G$ is regular or not and why?}
$G=(\Sigma,N,P,S)$ is regular \emph{only if} every prod rule has \emph{only one} of the following forms:
\begin{enumerate}
\item \mb{Right}-linear: non-terminals always on RHS of terminals
  \begin{itemize}
  \item $A\to aB$, where $A, B$ are non-terminals, $a$ is terminal
  \item $A\to a$, where $A$ is a non-terminal, $a$ is a terminal
  \item $A\to \varepsilon$, where $A$ is a non-terminal, $\varepsilon$ is empty string
  \end{itemize}
\item \mb{Left}-linear non-terminals always on LHS of terminals
  \begin{itemize}
  \item $A\to Ba$, where $A, B$ are non-terminals, $a$ is terminal
  \item $A\to a$, where $A$ is a non-terminal, $a$ is a terminal
  \item $A\to \varepsilon$, where $A$ is a non-terminal, $\varepsilon$ is empty string
  \end{itemize}
\end{enumerate}
\section*{Example 1: regular grammar (right-linear)}
\begin{align*}
  S&\to aS \gor bA \gor \varepsilon\\
  A&\to aA \gor b
\end{align*}
represented in regular expression: $a^+\gor ab(a^+b\gor b)\gor\varepsilon$
\section*{Example 2: regular grammar (right-linear, Mid-exam)}
\begin{align*}
  S&\to bD \tag{0} \\
  D&\to 0D \tag{1} \\
  D&\to 1D \tag{2} \\
  D&\to 0 \gor 1 \tag{3}
\end{align*}
regular exp: $b(0\gor 1)(0\gor 1)^+$ (binary numb with prefix $b$)

\begin{minipage}{.5\linewidth}
\section*{Example 3: not regular}
\[
  S\to aSb\gor \varepsilon
\]
The non-terminal $S$ has terminals on both sides: not right-/left-linear, thus \mr{not} regular.
\end{minipage}
\begin{minipage}{.5\linewidth}
\section*{Example 4: not regular (2 non-terms on RHS)}
\begin{align*}
  S&\to AB \\
  A&\to a\\
  B&\to b
\end{align*}
\end{minipage}
\section*{Q2: Construct context-free grammar}
\begin{lstlisting}[language=c]
  /* this is a comment /* with a nested comment */ */
\end{lstlisting}
Suppose tokens \texttt{/*}, \texttt{*/} and \texttt{c} for entire comment content
\begin{itemize}
\item construct a CFG production rules, terminals, non-terminals
  \begin{enumerate}
  \item Parse a string $S$ \mo{top down}: it's either a comment $C$ or $\varepsilon$
    \begin{equation}
      S\to C \gor \varepsilon
    \end{equation}
  \item A comment $C$ is enclosed in \texttt{/* T */}, $T$ is content
    \begin{equation}
      C\to /*\quad T\quad */
    \end{equation}
  \item Content $T$ can have zero (empty) or more items $I^*$
    \begin{equation}
      T\to I^*
    \end{equation}
  \item Item $I$ should be either a terminal or a comment $C$
    \begin{equation}
      T\to c \gor C
    \end{equation}
  \end{enumerate}
\section*{Q3: if given $G$ is context-free grammar (CFG) and why?}
\item The nested comment grammar is context-free because:
  \begin{enumerate}
  \item Each production has the form $A\to w$, where $A$ is non-terminal and $w$ is a string of terminals and non-terminals
  \item No production rule depends on the context surrounding the non-terminal being replaced
  \end{enumerate}
\item In general, to check if $G$ is a context-free grammar:
  \begin{enumerate}
  \item Examine the \mb{left side} of each production rule
    \begin{itemize}
    \item Must contain \textbf{exactly one} non-terminal symbol
    \item Cannot have terminals mixed with non-terminals
    \item Cannot have multiple symbols
    \end{itemize}
  \item Check for context dependencies
    \begin{itemize}
    \item Rules should not depend on surrounding symbols
    \item No conds like ``$A\to B$ only when $A$ appears after $C$''
    \end{itemize}
  \end{enumerate}
\item \mr{cannot} convert the CFG to a regular grammar because:
  \begin{enumerate}
  \item cannot count opening \texttt{/*} in regular grammar
  \item cannot match each with the paired \texttt{*/} in regular grammar
  \item cannot maintain proper nesting order in regular grammar
  \end{enumerate}
\item In general, regular languages cannot handle the following:
  \begin{itemize}[leftmargin=1em]
  \item Arbitrary nesting depth
  \item Matching pairs that require \textbf{memory} of previous symbols
  \item Center-embedded recursion
  \end{itemize}
\end{itemize}

\section*{Q4: if a given grammar $G$ is LL(1)}
\begin{itemize}
\item if it has any \mb{left-recursive} prod rule, it's \mr{not} LL(1): $E\to E+ T$ (need remove left-recursion first)
\item if \mb{common prefix} shared among some prod rules, it's \mr{not} LL(1) (need left refactoring):
  \begin{align*}
    T\to (T) \\
    E\to (E)
  \end{align*}
\item In general, for each non-terminal $A$ with prod $A\to \alpha_1\gor \alpha_2\gor\ldots\alpha_n$: $\mfn{first}(\alpha_1)\cup\mfn{first}(\alpha_2)\ldots\mfn{first}(\alpha_1) = \emptyset$
\item In the above, $\mfn{first}(T)\cup\mfn{first}(E) = \mset(\textsf{left-paren})\neq \emptyset$
\item In the below, $\mfn{first}(S)=\mfn{first}(a)\cup\mfn{first}(a)=\mset{a}\neq\emptyset$
  \[
    S\to aA \gor aB
  \]
\item if any $\alpha_i \overset{*}{\Rightarrow} \varepsilon$, then must $\mfn{first}(\alpha_i)\cup\mfn{follow}(A)=\emptyset$
\item In general, to verify a LL(1) grammar $G$:
  \begin{enumerate}
  \item eliminate left recursion in $G$ (if any)
  \item left refactor the grammar $G$ (if needed)
  \item Compute FIRST sets for all productions in $G$
  \item Compute FOLLOW sets for all non-terminals in $G$
  \item for each non-term, FIRST sets of its prod are disjoint
  \item For any nullable prod, verify FIRST and FOLLOW sets don't overlap
  \end{enumerate}
\end{itemize}
\section*{Q5: if a given $G$ is LR(0)}
\begin{minipage}{.5\linewidth}
\begin{align*}
  S&\to E\$ \\
  E&\to T \gor E;T \\
  T&\to \varepsilon \gor Ta
\end{align*}
\end{minipage}
\begin{minipage}{.5\linewidth}
  \centering
  \begin{tikzpicture}
    \node(i1)[lrst] {
      $S'\to\lrd S\$$\\
      $S\to\lrd E$\\
      $E\to\lrd T$\\
      $E\to\lrd E;T$\\
      $T\to\lrd \varepsilon$\\
      $T\to\lrd Ta$
    };
    \node[below right=0.1mm of i1.north east,font=\footnotesize\bf] {1};
  \end{tikzpicture}
\end{minipage}
\begin{itemize}
\item Add a new start symbol to the grammar: $S'\to S\$$
\item use \textbf{closure} to build the DFA: move the dot in each item
\end{itemize}
\begin{itemize}
\item use \textbf{goto} to create transition from $I$ to $I'$ on each shift
  \begin{itemize}[leftmargin=1em]
  \item for LR(0), reuse states with same items (prod, dot)
  % \item for LR(1), reuse states with same items (prod, dot, Las)
  \end{itemize}
\item if any state $I$ has \mb{R/R} or \mb{R/S} conflict, \mr{not} LR(0) or LR(1)
\item draw the parsing table that contains \mo{3} main parts
  \begin{minipage}{\linewidth}
    \centering
    \begin{tabular}{l|l|l}
      \hline
      States ($I$) & Actions (s/r, terms) & Gotos (non-terms)\\
      \hline
    \end{tabular}
  \end{minipage}
\end{itemize}
\section*{Q6: if a given $G$ is SLR(1) ($a.k.a$ SLR)}
\begin{itemize}
\item construct LR(0) parsing table (even there are conflicts)
\item identify all states with reduce items ($A\to\alpha \lrd$)
\item compute $\mfn{follow}(A)$ (get a set)
\item for each token $X$ in above set, $rn$ at $(I,X, A\to\alpha)$ in table
\item if still with conflict(s), \mr{not} SLR(1)
\end{itemize}
\section*{Q7: if a given $G$ is LR(1)}
\begin{minipage}{.5\linewidth}
\begin{align*}
  S&\to E\$ \\
  E&\to T \gor E;T \\
  T&\to \varepsilon \gor Ta
\end{align*}
\end{minipage}
\begin{minipage}{.5\linewidth}
  \centering
  \begin{tikzpicture}
    \node(i1)[lrst] {
      \begin{tabular}{ll}
        $S'\to\lrd S\$$ &,\$ \\
        $S\to\lrd E$    &,\$\\
        $E\to\lrd T$    &,$\$\gor ;$\\
        $E\to\lrd E;T$  &,$\$\gor ;$\\
        $T\to\lrd \varepsilon$ &,\$\\
        $T\to\lrd Ta$  &,\$
      \end{tabular}
    };
    \node[below right=0.1mm of i1.north east,font=\footnotesize\bf] {1};
  \end{tikzpicture}
\end{minipage}
\begin{itemize}
\item LR(1) items LR(0) items plus LA(s): $A\to \lrd X, \text{LAs}$
\item LAs can be a single token (e.g. $\$$) or a set of terminals $(a\gor b)$
\item In LR(1) DFA, same items have \emph{exactly} same prod, dot, LAs
\item In the above example, $S\to \lrd E$ intros 2 new items:
  \begin{align*}
    E&\to\lrd T   & &,\$\\
    E&\to\lrd E;T & &,\$
  \end{align*}
\item Similarly, $E\to \lrd E;T$ intros 2 more new items:
  \begin{align*}
    E&\to\lrd T    &  &,;\\
    E&\to\lrd E;T  &  &,;
  \end{align*}
\item drawing LR(1) parsing table is similar to that of LR(0)
\end{itemize}
\section*{Q8: if a given $G$ is LALR(1) ($a.k.a$ LALR)}
\begin{itemize}
\item follow the steps for creating the parsing table for LR(1)
\item try to merge states with similar items (same prod and dot pos, but difference LAs)
\item if no conflicts, it is LALR; otherwise, it is not
\end{itemize}
