% Need column 2 layout
\section*{Empty array}
\begin{minipage}{.45\linewidth}
\begin{lstlisting}[language=Java,morekeywords={var},frame=none]
// declare an empty array globally
// no IR tree, only a .data section
var v: [0..-1]int;
\end{lstlisting}
\end{minipage}
\begin{minipage}{.55\linewidth}
\begin{lstlisting}[language=bash,frame=none]
    .data
    .balign 8
emtyarry.v:
    .zero 0
\end{lstlisting}
\end{minipage}

\section*{Array index fixed (known at compile time) and variable (known at runtime)}
\begin{minipage}{.45\linewidth}
\begin{lstlisting}[language=Java,morekeywords={var},frame=none]
// global array into .data
var x: [0..9]int;
{
  var v = x[5];
}
\end{lstlisting}
\end{minipage}
\begin{minipage}{.55\linewidth}
\begin{lstlisting}[language=Python,frame=none]
    .data
    .balign 8
fixedarray.x:
	.zero 80
MOVE(     # v = x[5]
 TEMP v, # x[5] = x(,idx,8) = x(,5,8)
 MEM(NAME x, CONST 40, 8, true, false))
\end{lstlisting}
\end{minipage}

\begin{minipage}{.45\linewidth}
\begin{lstlisting}[language=Java,morekeywords={proc,var},frame=none]
var x: [0..9]int;
proc get(n: int): int {
  return x[n];
}
// global array alloc-ed as
//    .data
//    .balign 8
// x:
//   .zero 80
\end{lstlisting}
\end{minipage}
\begin{minipage}{.55\linewidth}
\begin{lstlisting}[language=Python,frame=none]
MOVE(TEMP t.0, TEMP n) # t.0 will be %rdi
BLT(TEMP t.0, CONST 0, badSub, L.1)
LABEL L.1 # i >= 0, low bound fine
BGT(TEMP t.0, CONST 9, badSub, L.2)
LABEL L.2 # i < 8, high bound fine
MOVE(TEMP %rax, # for return value
 MEM( # x[n] = x(,idx,8) = x(,n,8)
  ADD(NAME x, MUL(TEMP t.0, CONST 8)),
  CONST 0, 8, true, false))
JUMP(NAME L.0)
LABEL L.0
\end{lstlisting}
\end{minipage}

\section*{Array of records}
\begin{minipage}{.45\linewidth}
\begin{lstlisting}[language=Java,frame=none,morekeywords={New, struct}]
{
  var a = New(^[0..9] struct {
             x, y: int
  });
}
// a has 10 structs, each 16 bytes
// L1-2: malloc for a at mem t.0
// L3-4: set up low/high bounds
// L5: i = t.1
// L7-9: a[i] = t.0(,i,16) in t.3
// L10-12: init a[0].x to 0
// L13-15: init a[0].y to 0
// L16: i += 1 (t.1 += 1)
// L17: if t.1 < t.2, jump to L.0
// L18: else init done
// L19: make a points to MEM[t.0]
\end{lstlisting}
\end{minipage}
\begin{minipage}{.55\linewidth}
\begin{lstlisting}[language=Python,frame=none,numbers=left,numbersep=4pt]
MOVE(TEMP t.0,
 CALL(NAME malloc, CONST 0, CONST 160))
MOVE(TEMP t.2, CONST 10)
MOVE(TEMP t.1, CONST 0)
BGE(TEMP t.1, TEMP t.2, L.1, L.0)
LABEL L.0
MOVE(
 TEMP t.3,
 ADD(TEMP t.0, MUL(TEMP t.1, CONST 16)))
MOVE(
 MEM(TEMP t.3, CONST 0, 8, true, false),
 CONST 0)
MOVE(
 MEM(TEMP t.3, CONST 8, 8, true, false),
 CONST 0)
MOVE(TEMP t.1, ADD(TEMP t.1, CONST 1))
BLT(TEMP t.1, TEMP t.2, L.0, L.1)
LABEL L.1
MOVE(TEMP a, TEMP t.0)
\end{lstlisting}
\end{minipage}
\begin{itemize}
\item When there is an array in use, set up its \mb{low} (0) and \mb{high} (\texttt{array.len + 1}) bounds
\item \mb{low} bound \emph{may} be used as the loop variant \texttt{i}, or a new temp is set up as \texttt{i}:
  \begin{enumerate}
  \item at start, \texttt{BLT(idx, high, body, end)} (\texttt{idx} may be \texttt{low})
  \item in body, \texttt{BLT(idx, high, start, end)} (after increased \texttt{i})
  \end{enumerate}
\end{itemize}

\section*{1st byte of char array (string) stores the array's length}
\begin{minipage}{.45\linewidth}
\begin{lstlisting}[language=Java,frame=none,
morekeywords={proc,var,to,loop,Text,Ord,Address,Number,Loophole,}]
proc puts(s: Address): int;
proc println(t: Text) {
  var a = Loophole(t, ^[]byte);
  puts(&a[0]);
}
MOVE(TEMP println.a, TEMP println.t)
MOVE(TEMP t.1, TEMP println.a)
//  check array low bound
BEQ(TEMP t.1, CONST 0,
  println.badPtr, L.3)
LABEL L.3
// array header in t.3 now
MOVE(TEMP t.3, TEMP t.1)
MOVE(TEMP t.2, CONST 0) // idx in t.2
BLT(TEMP t.2, CONST 0,
 println.badSub, L.4)
\end{lstlisting}
\end{minipage}
\begin{minipage}{.6\linewidth}
\begin{lstlisting}[language=Python,frame=none,xleftmargin=-1em]
LABEL L.4
# MEM(t.3+8) stores array.length
BGE( # ensure idx <= array.length
 TEMP t.2,
 MEM(TEMP t.3, CONST 8, 8, false, false),
 println.badSub, L.5)
# offset t.2 will change each time puts called
LABEL L.5
EXP(
 CALL(
  NAME puts,
  CONST 0,
  ADD( # t.2(t.3,0,8)
   MEM(TEMP t.3, CONST 0, 8, false, false),
   TEMP t.2)))
\end{lstlisting}
\end{minipage}
\begin{minipage}{.45\linewidth}
\begin{lstlisting}[language=Java,morekeywords={Number,var,to,loop,Ord,Loophole},
frame=none,xleftmargin=-1em]
  proc append (t1, t2: Text): Text {
  var a1 = Loophole(t1, ^[]byte);
  var a2 = Loophole(t2, ^[]byte);
  // drop '\0' terminator
  var n1 = Number(a1^)-1;
  var n2 = Number(a2^)-1;
  var result = New(^[]byte, n1+n2+1);
  for i = 0 to n1-1 loop
    result[i] = a1[i];
  for i = 0 to n2-1 loop
    result[n1+i] = a2[i];
  // Add '\0' terminator
  result[n1+n2] = Ord('\000');
  return Loophole(result, Text);
}
MOVE(TEMP t.4, TEMP a1)
BEQ(TEMP t.4, CONST 0,badPtr, L.7)
LABEL L.7
MOVE(TEMP n1, // n1 := a1.len - 1
     SUB(MEM(TEMP t.4, // a1 == t.4
         CONST 8, 8, false, false),
     CONST 1))
MOVE(TEMP t.5, TEMP a2)
BEQ(TEMP t.5, CONST 0, badPtr, L.8)
LABEL L.8
MOVE(TEMP n2, // n2 := a2.len - 1
     SUB(MEM(TEMP t.5, // t.5 == a2
         CONST 8, 8, false, false),
     CONST 1))
MOVE(TEMP t.6, // ret = n1+n2+1
     ADD(ADD(TEMP n1, TEMP n2),
         CONST 1))
BLT(TEMP t.6, CONST 0, badSub, L.9)
\end{lstlisting}
\end{minipage}
\begin{minipage}{.6\linewidth}
\begin{lstlisting}[language=Python,frame=none,xleftmargin=-1em]
LABEL L.9
MOVE(TEMP t.7, TEMP t.6) # keep orig. ret in t.6
MOVE(TEMP t.8, CONST 16) # |head ptr| len | ret |
MOVE(TEMP t.8, ADD(TEMP t.8, TEMP t.7))
MOVE(TEMP t.9,
 CALL(NAME malloc, CONST 0, TEMP t.8))
MOVE(MEM(TEMP t.9, CONST 0, 8, false, false),
     ADD(TEMP t.9, CONST 16)) # store metadata
MOVE(MEM(TEMP t.9, CONST 8, 8, false, false),
     TEMP t.6) # update t.6 (ret) with metadata
MOVE(TEMP t.11, # ret length in t.11
     MEM(TEMP t.9, CONST 8, 8, false, false))
MOVE(TEMP t.12, # ret head ptr in t.12
     MEM(TEMP t.9, CONST 0, 8, false, false))
MOVE(TEMP t.10, CONST 0) # i = 0
BGE(TEMP t.10, TEMP t.11, L.12, L.10)
LABEL L.10
MOVE(MEM(ADD(TEMP t.12, TEMP t.10),
         CONST 0, 1, true, false),
     CONST 0) 0(ret,)
MOVE(TEMP t.10, ADD(TEMP t.10, CONST 1))
MOVE(TEMP result, TEMP t.9)
MOVE(TEMP t.13, CONST 0) #i=0
MOVE(TEMP t.14, SUB(TEMP n1, CONST 1)) # n-1
MOVE(TEMP t.15, CONST 1) # step = 1
JUMP(NAME L.14) # loop init done got L14
# LABEL L.13 - L.21 is loop body (see back)
\end{lstlisting}
% LABEL L.MOVE(TEMP i, TEMP t.13)
% MOVE(TEMP t.16, TEMP result)
% BEQ(TEMP t.16, CONST 0, badPtr, L.16)
% LABEL L.16
% MOVE(TEMP t.18, TEMP t.16)
% MOVE(TEMP t.17, TEMP i)
% BLT(TEMP t.17, CONST 0, badSub, L.17)
% LABEL L.17
% BGE(TEMP t.17,
%  MEM(TEMP t.18, CONST 8, 8, false, false),
%  badSub, L.18)
% LABEL L.18
% MOVE(TEMP t.19, TEMP a1)
% BEQ(TEMP t.19, CONST 0, badPtr, L.19)
% LABEL L.19
% MOVE(TEMP t.21, TEMP t.19)
% MOVE(TEMP t.20, TEMP i)
% BLT(TEMP t.20, CONST 0, badSub, L.20)
% LABEL L.20
% BGE(TEMP t.20, MEM(TEMP t.21,
%                CONST 8, 8, false, false),
%  badSub, L.21)
% LABEL L.21
\end{minipage}
