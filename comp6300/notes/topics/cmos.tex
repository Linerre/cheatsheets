\begin{itemize}
\item lowest voltage = 0V, called \emph{ground} or GND (sometimes, V$_{SS}$)
\item highest voltage dec. from 5V (1970s-80s) to $\leqslant$1.2V, called V$_{DD}$
\item \emph{short circut} occurs when both pull-up and pull-down network ON
\item output \emph{floats} when both pull-up and pull-down network OFF
\begin{tabular}[h]{cllcl}
  \hline
  Gate & nMOS & pMOS & Pass well & Network  \\
  \hline
  0/LOW/GND & OFF & ON  & 0 &  pull-dow\textbf{n}\\
  1/HIGH/V$_{DD}$ & ON & OFF & 1 & pull-u\textbf{p}\\
  \hline
\end{tabular}
\item CMOS design prefers NANDs and NORs (\emph{not associative})
\item wide NAND/NOR gates can't use chain/tree strategy
\item Any logic function can be implemented using only NANDs/NORs
\item Wide NANDs and NORs use trees of smaller devices
\end{itemize}

% Handy function to show the positions of nodes
\def\normalcoord(#1){coordinate(#1)}
\def\showcoord(#1){coordinate(#1) node[circle, red, draw, inner sep=1pt,
pin={[red, overlay, inner sep=0.5pt, font=\tiny, pin distance=0.1cm,
pin edge={red, overlay}]45:#1}](){}}
\let\coord=\normalcoord
\let\coord=\showcoord % coordinates markers
\def\killdepth#1{{\raisebox{0pt}[\height][0pt]{#1}}} % baseline tweaks

% Two-input NAND gate
\begin{minipage}{0.4\linewidth}
  \begin{circuitikz}
    [scale=0.7,
    transform shape,
    information text/.style={inner sep=1ex}]
  % Nodes
  \draw (2,1.5) node[pmos](p2){\large p2};
  \draw (3.5,1.5) node[pmos](p1){\large p1};
  \draw (3.5,0) node[nmos](n1){\large n1};
  \draw (3.5,-1.2) node[nmos](n2){\large n2} (3.5,-1.5) node[ground](GND){};
  \draw (0,0) node[left](A){A};
  \draw (0,-1.2) node[left](B){B};
  % Wires
  \draw (A) -- (n1.G);
  \draw (B) -- (n2.G);
  \draw (p1.G) -- (n1.G);
  % from p2.G to the intersection of two lines:
  % vertical line that passes p2.G and horizontal line that passes n2.G
  \draw (p2.G) -- (p2.G |- n2.G);
  % I/O
  \draw (p2.D) -- (p1.D) -- ++(0.5,0) node[circ]{} +(0.1,0) node[right](y){$Y$};
  \draw ($(p2.G |- p2.S)+(.7,0)$) -- (p2.S) -- +(1,0) node[above]{\large $V_{DD}$} -- (p1.S) -- +(0.3,0);

  % Caption
  \draw (GND) ($(GND) - (2,0)$) node[below, text width=2.5cm]
  {Two-input NAND gate schematic};
  % Information text, see https://tikz.dev/tutorial#sec-2.21
  % Whenever possible, use it
\end{circuitikz}
\end{minipage}
\begin{minipage}{0.55\linewidth}
  The nMOS transistors n1 and n2 are connected in series;\\
  The pMOS transistors p1 and p2 are in parallel.\\
  This acts as the base for multiple-input NAND gate schematic.
  For example, 4-input type would have 4 pMOS transistors in parallel
  and 4 nMOS transistors in series.
\end{minipage}
\begin{tabular}[h]{ccccc}
  \hline
  A & B & Pull-Dow\textbf{n} network & Pull-U\textbf{p} network & Y  \\
  \hline
  0 & 0 & OFF & ON  & 1 \\
  0 & 1 & OFF & ON  & 1 \\
  1 & 0 & OFF & ON  & 1 \\
  1 & 1 & ON & OFF  & 0 \\
  \hline
\end{tabular}

% Two-input NOR gate
\begin{minipage}{.4\linewidth}
\begin{circuitikz}
  [scale=0.7,
    transform shape,
    information text/.style={inner sep=2em}]
    % Input A is the origin
    \node at (0, 0) [left](A){A};
    \node at (0, -1.2) [left](B){B};
    \node at (3,0) [pmos](p1){\large p1};
    \node at (3,-1.2) [pmos](p2){\large p2};
    \node at (1.5,-2.7) [nmos](n1){\large n1};
    \node at (3,-2.7) [nmos](n2){\large n2};
    \node at (p1.S) [tground](vdd){};
    \node at ($(p1.S)+(0,.1)$) [above]{V$_{DD}$};
    \node at (n2.S) [sground](gnd){};
    \node at (gnd.east) [left, yshift=-.2cm, xshift=-.2cm, text width=2.5cm]{Two-input NOR Gate Schematic};

    \draw
    (A) -- (p1.G)
    (B) -- (p2.G)
    (p2.D) -- (n2.D)
    (n1.D) -- (n2.D) to[short, o-*] (n2.D) -- +(.5,0) node[right](Y){Y}
    (n1.S) -- (n2.S)
    (n1.G) -- (n1.G |- p1.G)
    (n2.G) -- (n2.G |- p2.G);
\end{circuitikz}
\end{minipage}
\begin{minipage}{.55\linewidth}
  The NOR gate should produce a 0 output if either input is 1.
  Hence, the pull-down network should have two nMOS transistors in parallel.
  By the conduction complements rule, the pMOS transistors must be in series.
\end{minipage}
