\begin{enumerate}
\item In a given truth table, look for rows that have an output of 1/TRUE
\item\label{bf:step2} For each such row, write a boolean equation such that inputs $\rightarrow$ 1/TRUE; usually using AND (multiplication $A\cdot B$) and NOT ($\overline{B}$)
\item Chain all the terms in \ref{bf:step2} using OR (production +)
  \[
Y = \overbracket{\overline{C}\overline{B}A + \overline{C}BA + \underbracket{CB\overline{A}}_{\mathclap{\text{product/implicant}}} + CBA}^{\text{sum of products}}
  \]
\item sum-of-products also called \emph{two-level logic} (AND connected to SUM)
\end{enumerate}
\begin{itemize}
\item chain: propagation delays increase linearly with number of inputs
\item tree: propagation delays increase logarithmatically with number of inputs
\item A boolean equation can be the sum of minterms (0-indexed) (but not necessarily the shortest): $f(A,B) = \Sigma(m_{i},m_{k},\ldots)$
\item A boolean equation can be the product of maxterms (0-index): $F(A,B) = \Pi(m_{i},m_{k},\ldots)$; each $m$ is FALSE
\item To simplify a boolean equation, use boolean algebra or K-map
\item To simplify a truth table, use ``don't care'' $X$
\item Simplifying a boolean equation for digital circuits may cause hazards
\end{itemize}

\begin{tabular}{l|c|c|c}
  \hline
  \multicolumn{4}{c}{$Y = \overline{A}B + B$} \\
  \hline
  literals & true form & complementary form & product/implicant \\
  \hline
  A, $\overline{A}$, B & A & $\overline{A}$ & $\overline{A}B$\\
  \hline
  \multicolumn{4}{l}{\emph{true form} does NOT mean A is TRUE, just that A has no overline} \\
  \hline
  \textbf{implicant} & \multicolumn{3}{l}{The AND of one or more literals, also \emph{product} } \\
  \textbf{sum} & \multicolumn{3}{l}{The OR of one or more literals} \\
  \textbf{minterm} & \multicolumn{3}{l}{a TRUE product involving all inputs to the boolean fn: $\overline{A}B$}\\
  \textbf{maxterm} & \multicolumn{3}{l}{a FALSE sum involving all inputs to the boolean fn: $A + B$}\\
  \hline
\end{tabular}
