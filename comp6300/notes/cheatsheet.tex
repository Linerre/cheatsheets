\documentclass[8pt,a4paper,landscape]{extarticle}
% -- Layout ----
\usepackage[top=0.6cm, bottom=0.6cm, left=0.5cm, right=0.5cm, landscape]{geometry}

% -- Titles ----
\usepackage[
  tiny,                     % text size title
  compact                   % reduce vertical space before/after title
]{titlesec}
% \titlespacing*
\titleformat{\section}{\normalfont\small\bfseries}{\thesection}{0em}{} % Remove space before and after section titles
\titleformat{\subsection}{\normalfont\small\bfseries}{\thesubsection}{0em}{} % Remove space before and after subsection titles
\titlespacing*{\section}{0pt}{0pt}{0pt} % Remove space before/after section titles
\titlespacing*{\subsection}{0pt}{0pt}{0pt} % Remove space before/after subsec titles

% -- Colors ----
\usepackage[dvipsnames]{xcolor}
\definecolor{dmm}{RGB}{192,192,192} % Define a custom dimmed text color
\definecolor{cmt}{RGB}{61,123,123}

% -- Math ------
\usepackage{mathtools}
\usepackage{amssymb}
\usepackage{turnstile}%better vdash

% -- Lists -----
\usepackage[inline]{enumitem}
\setlist{noitemsep}% Remove vspace between items
% Set vspace before and after  list environments as well as the left margin
\setlist[itemize,1]{leftmargin=.6em,labelindent=0pt,labelsep=2pt,
  topsep=1pt,partopsep=1pt}
\setlist[enumerate,1]{leftmargin=1em,labelindent=0pt,labelsep=2pt,
  topsep=1pt,partopsep=1pt}
\setlist[itemize,2]{leftmargin=.3em,labelindent=1pt,topsep=1pt,partopsep=1pt}
\setlist[enumerate,2]{leftmargin=0.2em,labelindent=1pt,topsep=1pt,partopsep=1pt}
\setlist[description]{labelwidth=\linewidth,font=\small\bfseries,leftmargin=1em,topsep=1pt,partopsep=1pt}
% -- Code listing ---
\usepackage{listings}
\lstset{
  aboveskip=3pt,
  belowskip=3pt,
  basicstyle=\small\ttfamily,
  breaklines=true,
  % commentstyle=\upshape\ttfamily,
  captionpos=b,
  commentstyle=\color{cmt},
  frame=single,
  keepspaces=false,
  keywordstyle=\bfseries,
  showspaces=false,
  showstringspaces=false,
  showtabs=false,
  tabsize=2,
}

% Parse Trees
\usepackage{tikz}
\usetikzlibrary{ arrows, automata, bbox, calc, positioning,  decorations.pathmorphing, decorations.pathreplacing, decorations.shapes, }
\tikzset{
% ->, % makes the edges directed
>=stealth', % makes the arrow heads bold
node distance=1cm, % specifies minimum distance between two nodes
% small/.style={},
every state/.style={thick}, % sets the properties for each ’state’ node
every node/.style={inner sep=1pt},
initial text=start, % sets the text that appears on the start arrow
}

% Place a figure env right here via [H] option
\usepackage{float}

% Side by side figure
\usepackage{subcaption}
% \usepackage{caption}
% \captionsetup{belowskip=0pt, aboveskip=0pt}


% -- Multi-Col layout --
\usepackage{multicol}

% No indentation
\setlength\parindent{0pt}
\setlength\abovedisplayskip{-5pt}
\setlength\belowdisplayskip{-5pt}
\setlength\abovedisplayshortskip{-4pt}
\setlength\belowdisplayshortskip{-4pt}
\newcommand{\gor}{\;|\;}
\newcommand{\num}{\texttt{\#}~}
\renewcommand{\arraystretch}{1.2}


\begin{document}
% Suppress page number for all pages
\pagestyle{empty}
% Math env spacing
\setlength{\abovedisplayskip}{1pt}
\setlength{\belowdisplayskip}{1pt}
\setlength{\abovedisplayshortskip}{0pt}
\setlength{\belowdisplayshortskip}{0pt}
% hl
\sethlcolor{dim}

% Notes begin
\begin{multicols*}{3}

% \section*{Decimal $\iff$ Binary (Unsigned, U)}
% \input{topics/binary_num}

% \pagebreak

\section*{Combinational v.s. Sequential}
% \begin{tabular}[h]{ll}
  \hline
  Type & Outputs depend on?  \\
  \hline
  Combinational & \emph{only} current inputs (combined inputs, memoryless)  \\
  Sequential & both current \emph{and} previous inputs (input sequence, memory) \\
  \hline
\end{tabular}


\section*{MOSFETS (pMOS, nMOS, CMOS)}
% \begin{itemize}
\item lowest voltage = 0V, called \emph{ground} or GND (sometimes, V$_{SS}$)
\item highest voltage dec. from 5V (1970s-80s) to $\leqslant$1.2V, called V$_{DD}$
\item \emph{short circut} occurs when both pull-up and pull-down network ON
\item output \emph{floats} when both pull-up and pull-down network OFF
\begin{tabular}[h]{cllcl}
  \hline
  Gate & nMOS & pMOS & Pass well & Network  \\
  \hline
  0/LOW/GND & OFF & ON  & 0 &  pull-dow\textbf{n}\\
  1/HIGH/V$_{DD}$ & ON & OFF & 1 & pull-u\textbf{p}\\
  \hline
\end{tabular}
\item CMOS design prefers NANDs and NORs (\emph{not associative})
\item wide NAND/NOR gates can't use chain/tree strategy
\item Any logic function can be implemented using only NANDs/NORs
\item Wide NANDs and NORs use trees of smaller devices
\end{itemize}

% Handy function to show the positions of nodes
\def\normalcoord(#1){coordinate(#1)}
\def\showcoord(#1){coordinate(#1) node[circle, red, draw, inner sep=1pt,
pin={[red, overlay, inner sep=0.5pt, font=\tiny, pin distance=0.1cm,
pin edge={red, overlay}]45:#1}](){}}
\let\coord=\normalcoord
\let\coord=\showcoord % coordinates markers
\def\killdepth#1{{\raisebox{0pt}[\height][0pt]{#1}}} % baseline tweaks

% Two-input NAND gate
\begin{minipage}{0.4\linewidth}
  \begin{circuitikz}
    [scale=0.7,
    transform shape,
    information text/.style={inner sep=1ex}]
  % Nodes
  \draw (2,1.5) node[pmos](p2){\large p2};
  \draw (3.5,1.5) node[pmos](p1){\large p1};
  \draw (3.5,0) node[nmos](n1){\large n1};
  \draw (3.5,-1.2) node[nmos](n2){\large n2} (3.5,-1.5) node[ground](GND){};
  \draw (0,0) node[left](A){A};
  \draw (0,-1.2) node[left](B){B};
  % Wires
  \draw (A) -- (n1.G);
  \draw (B) -- (n2.G);
  \draw (p1.G) -- (n1.G);
  % from p2.G to the intersection of two lines:
  % vertical line that passes p2.G and horizontal line that passes n2.G
  \draw (p2.G) -- (p2.G |- n2.G);
  % I/O
  \draw (p2.D) -- (p1.D) -- ++(0.5,0) node[circ]{} +(0.1,0) node[right](y){$Y$};
  \draw ($(p2.G |- p2.S)+(.7,0)$) -- (p2.S) -- +(1,0) node[above]{\large $V_{DD}$} -- (p1.S) -- +(0.3,0);

  % Caption
  \draw (GND) ($(GND) - (2,0)$) node[below, text width=2.5cm]
  {Two-input NAND gate schematic};
  % Information text, see https://tikz.dev/tutorial#sec-2.21
  % Whenever possible, use it
\end{circuitikz}
\end{minipage}
\begin{minipage}{0.55\linewidth}
  The nMOS transistors n1 and n2 are connected in series;\\
  The pMOS transistors p1 and p2 are in parallel.\\
  This acts as the base for multiple-input NAND gate schematic.
  For example, 4-input type would have 4 pMOS transistors in parallel
  and 4 nMOS transistors in series.
\end{minipage}
\begin{tabular}[h]{ccccc}
  \hline
  A & B & Pull-Dow\textbf{n} network & Pull-U\textbf{p} network & Y  \\
  \hline
  0 & 0 & OFF & ON  & 1 \\
  0 & 1 & OFF & ON  & 1 \\
  1 & 0 & OFF & ON  & 1 \\
  1 & 1 & ON & OFF  & 0 \\
  \hline
\end{tabular}

% Two-input NOR gate
\begin{minipage}{.4\linewidth}
\begin{circuitikz}
  [scale=0.7,
    transform shape,
    information text/.style={inner sep=2em}]
    % Input A is the origin
    \node at (0, 0) [left](A){A};
    \node at (0, -1.2) [left](B){B};
    \node at (3,0) [pmos](p1){\large p1};
    \node at (3,-1.2) [pmos](p2){\large p2};
    \node at (1.5,-2.7) [nmos](n1){\large n1};
    \node at (3,-2.7) [nmos](n2){\large n2};
    \node at (p1.S) [tground](vdd){};
    \node at ($(p1.S)+(0,.1)$) [above]{V$_{DD}$};
    \node at (n2.S) [sground](gnd){};
    \node at (gnd.east) [left, yshift=-.2cm, xshift=-.2cm, text width=2.5cm]{Two-input NOR Gate Schematic};

    \draw
    (A) -- (p1.G)
    (B) -- (p2.G)
    (p2.D) -- (n2.D)
    (n1.D) -- (n2.D) to[short, o-*] (n2.D) -- +(.5,0) node[right](Y){Y}
    (n1.S) -- (n2.S)
    (n1.G) -- (n1.G |- p1.G)
    (n2.G) -- (n2.G |- p2.G);
\end{circuitikz}
\end{minipage}
\begin{minipage}{.55\linewidth}
  The NOR gate should produce a 0 output if either input is 1.
  Hence, the pull-down network should have two nMOS transistors in parallel.
  By the conduction complements rule, the pMOS transistors must be in series.
\end{minipage}


\section*{Functional Specifications (Boolean Equations)}
% \begin{enumerate}
\item In a given truth table, look for rows that have an output of 1/TRUE
\item\label{bf:step2} For each such row, write a boolean equation such that inputs $\rightarrow$ 1/TRUE; usually using AND (multiplication $A\cdot B$) and NOT ($\overline{B}$)
\item Chain all the terms in \ref{bf:step2} using OR (production +)
  \begin{equation}\label{eq:fnspec1}
Y = \overbracket{\overline{C}\overline{B}A + \overline{C}BA + \underbracket{CB\overline{A}}_{\mathclap{\text{product/implicant/minterm}}} + CBA}^{\text{sum of products}}
  \end{equation}
\item sum-of-products also called \emph{two-level logic} (AND connected to SUM)
\end{enumerate}
\begin{tabular}[h]{cccc}
  \hline
  literals & true/complement form & product/implicant & sum \\
  $A, \overline{A}, B, C, \overline{C}$ & $A$/$\overline{A}$ & $\overline{A}BC$ & A + B \\
  \hline
\end{tabular}
\begin{itemize}
\item[] \emph{true form} does NOT mean A is TRUE, just that A has no overline
\item \textbf{minterm}: a TRUE prodt. involving all inputs to the boolean fn: $\overline{A}B$
\item \textbf{maxterm}: a FALSE sum involving all inputs to the boolean fn: $A + B$
\item chain: $t_{pd}$ increases linearly with number of inputs
\item tree: $t_{pd}$ increases logarithmatically with number of inputs
\item A boolean equation can be the sum of minterms (0-indexed) (but not necessarily the shortest): $f(A,B) = \Sigma(m_{i},m_{k},\ldots)$
\item A boolean equation can be the product of maxterms (0-index): $F(A,B) = \Pi(m_{i},m_{k},\ldots)$; each $m$ is FALSE
\item To simplify a boolean equation, use boolean algebra or K-map
\item To simplify a truth table, use ``don't care'' $X$
\item Simplifying a boolean equation for digital circuits may cause hazards
\end{itemize}


\section*{Bubble Pushing (backward: inputs $\leftarrow$ output)}
% % Step 1
\begin{enumerate}
  \item Begin at output and work \emph{back} toward inputs.
  \item Work on one gate each time, use Demorgan's law
  \item NAND(AB)

% NAND to NOT (OR)
\begin{circuitikz}
  [transform shape,
  information text/.style={inner sep=1ex}]
  \ctikzset{
      logic ports=ieee,
      logic ports/scale=0.6,
   }

   \draw (-2,.6) node[nand port](nd1){}
   (nd1.in 1) -- +(0,0) node[left](A){A}
   (nd1.in 2) -- +(0,0) node[left](B){B}
   (nd1.out) -- +(.3,0) node[right](o1){C$_{out}$};
   \draw [-{Latex[length=3mm]}] (o1.east) -- ++(1,0)
   node[above](tip1){$\qquad\overline{A\cdot B} = \overline{A} + \overline{B}$} --
   ++(2,0) node[](emp1){};

   \draw (emp1.east)  +(1.5,0) node[or port](or1){};
   \draw (or1.bin 1) -- +(-.05,0) node[ocirc]{};
   \draw (or1.bin 2) -- +(-.05,0) node[ocirc]{};
   \draw (or1.in 1) -- +(-.15,0) node[left]{A};
   \draw (or1.in 2) -- +(-.15,0) node[left]{B};
   \draw (or1.out) +(.2,0) node[right]{C$_{out}$};
\end{circuitikz}

\item NOR(AB)

% NOR to AND(NOT)
\begin{circuitikz}
  [transform shape,
  information text/.style={inner sep=1ex}]
  \ctikzset{
      logic ports=ieee,
      logic ports/scale=0.6,
   }

   \draw (-2,.6) node[nor port](nr2){}
   (nr2.in 1) -- +(0,0) node[left](A){A}
   (nr2.in 2) -- +(0,0) node[left](B){B}
   (nr2.out) -- +(.3,0) node[right](o2){C$_{out}$};
   \draw [-{Latex[length=3mm]}] (o2.east) -- ++(1,0)
   node[above](tip1){$\qquad\overline{A+B} = \overline{A} \cdot \overline{B}$} --
   ++(2,0) node[](emp2){};

   \draw (emp2.east)  +(1.5,0) node[and port](ad1){};
   \draw (ad1.bin 1) -- +(-.05,0) node[ocirc]{};
   \draw (ad1.bin 2) -- +(-.05,0) node[ocirc]{};
   \draw (ad1.in 1) -- +(-.15,0) node[left]{A};
   \draw (ad1.in 2) -- +(-.15,0) node[left]{B};
   \draw (ad1.out) +(.2,0) node[right]{C$_{out}$};
 \end{circuitikz}

\item Bubble canceling

\begin{circuitikz}[framed]
  [transform shape,
  information text/.style={inner sep=1ex}]
  \ctikzset{
      logic ports=ieee,
      logic ports/scale=0.5,
   }

   % original (left)
   \draw (-2,.6) node[nand port](nd1){}
   (nd1.in 1) -- +(0,0) node[left](a1){A}
   (nd1.in 2) -- +(0,0) node[left](b1){B}
   (nd1.out) -- ++(.3,0) node[anchor=east](emp1){}
   node[nand port, anchor=in 1](nd2){}
   (emp1) -- +(.2,0) -- (nd2.in 1)
   (nd2.in 2) -- +(-.2,0) node(emp2){}
   (emp2.center) -- +(0,-.3)node(emp3){} -- (nd1.in 2 |- emp3.center) node[left](c1){C}
   (nd2.out) -- +(0.1,0) node[right](y1){Y};

   % arrow
   \draw ($(nd2.out)+(.5,.5)$) node [above](note1){push this bubble first};
   \draw [-{Latex[length=2mm]}] (note1.south) -- (nd2.bout);

   \draw [-{Latex[length=2mm]}] (y1.east) -- ++(.5,0)
   node[above,xshift=2em](tip1){$\overline{D\cdot C} = \overline{D} + \overline{C}$} --
   ++(2,0) node(emp3){}
   node[below,below of=tip1,yshift=.5cm](tip2){$D=\overline{A\cdot B}$};


   % and to nand (left below)
   \draw ($(nd1.bin 1)+(0,-1.3)$) node[and port,scale=0.9](ada){};
   \draw [-{Latex[length=2mm]}] (ada.bout)  ++(.4,0)
   node(t1){} -- +(.5,0) node(){};
   \draw ($(ada.out)+(1.5,0)$) node[nand port,scale=0.9](na){};
   \draw (na.out) -- +(.2,0) node[not port,anchor=in,scale=0.9](noa){};

   \draw ($(nd1.bin 1)+(0,-2)$) node[or port,scale=0.9](ora){};
   \draw [-{Latex[length=2mm]}] (ora.bout)  ++(.4,0)
   node(t1){} -- +(.5,0) node(){};
   \draw ($(ora.out)+(1.5,0)$) node[nor port,scale=0.9](nb){};
   \draw (nb.out) -- +(.2,0) node[not port,anchor=in,scale=0.9](noa){};

   % bubble identifying (right)
   \draw ($(emp3)+(1,0)$) node[nand port,anchor=bin 2](nd3){}
   (nd3.in 1) -- +(0,0) node[left](a2){A}
   (nd3.in 2) -- +(0,0) node[left](b2){B}
   (nd3.out) -- ++(.3,0) node[anchor=east](emp4){}
   node[or port, anchor=in 1](or4){}
   (or4.bin 1)  +(-.05,0) node[ocirc](or4b1){}
   (or4.bin 2)  +(-.05,0) node[ocirc]{}

   (emp4) -- +(.2,0) -- (or4.in 1)
   (or4.in 2) -- +(-.2,0) node(emp5){}
   (emp5.center) -- +(0,-.3)node(emp6){} -- (nd3.in 2 |- emp6.center) node[left](c2){C}
   (or4.out) -- +(0,0) node[right](y2){Y};

   \draw ($(nd3.out)+(.2,.5)$) node [above](note2){can cancel each other};
   \draw [-{Latex[length=1.5mm]}] (note2.south) -- (nd3.bout);
   \draw [-{Latex[length=1.5mm]}] (note2.south) -- (or4b1);


   % bubble canceling (right below)
   \draw ($(nd3.bin 2)+(0,-1.2)$) node[and port,anchor=bin 2](ad1){}
   (ad1.in 1) -- +(0,0) node[left](a3){A}
   (ad1.in 2) -- +(0,0) node[left](b3){B}
   (ad1.out) -- ++(.3,0) node[below,anchor=east](emp7){}
   node[or port, anchor=in 1](or5){}
   (or5.bin 1)  +(-.05,0) node{}
   (or5.bin 2)  +(-.05,0) node[ocirc]{}

   (emp7) -- +(.2,0) -- (or5.in 1)
   (or5.in 2) -- +(-.2,0) node(emp7){}
   (emp7.center) -- +(0,-.3)node(emp8){} -- (ad1.in 2 |- emp8.center) node[left](c3){C}
   (or5.out) -- +(0,0) node[right](y2){Y};
\end{circuitikz}
\end{enumerate}


\section*{Karnaugh Maps (K-Maps works well for up to 4 vars)}
\begin{itemize}
\item each square in K-map corresponds to a row in Truth table
\item each square in K-map contains value of output for that row
\item adjacent squares share all the same literals except one
\item a var's true and compl. forms all in circle, it's excluded from implicant
\item K-map is cyclic, left edge adjacent to right,  top adjacent to bottom
\end{itemize}

Rules for finding a minimized equation from a K-map
\begin{enumerate}
\item draw recton K-Map where sum of squares (boolean fn) evals to 1
\item draw \textbf{fewest} rects necessary to cover all 1s
\item rect has a width and length that must be a power of 2: 1,2,4
\item rect can overlap other implicants (rule 1)
\item a prime implicant is not completely contained in any other implicant
\item implicant can be uniquely identified by a single product term
\item the larger the implicant, the smaller the product term (rule 2)
\item each rect keeps vars that appear only in true or complement forms
\item If a var is 0, use its complement form ($\overline{A}$)
\end{enumerate}
\begin{minipage}{\linewidth}
  \rowcolors{1}{Gray}{white}
\begin{tabular}{|c|c|c|c|c|}
  \hline
C\textbackslash AB & 00 & 01 & 11 & 10\tikzmark{t} \\
  \hline
  \hiderowcolors
  \cellcolor{Gray}0 & 0 & 0 & 1\tikzmark{i1} & 1\tikzmark{i2} \\
  \hline
  \cellcolor{Gray}1 & 0 & 1\tikzmark{i3} & 1\tikzmark{i4} & 0 \\
  \hline
\end{tabular}
\begin{tikzpicture}[remember picture, overlay]
  \tikzset{shape example/.style=
    {color=red, draw, line width=1pt, inner sep=0pt, minimum width=1cm,minimum height=0.3cm,anchor=west,rounded corners}}
  \draw node[name=im1,shape=rectangle,shape example] at ($(pic cs:i1)+(-.23,.1)$) {};
  \draw node[name=im2,shape=rectangle,shape example] at ($(pic cs:i3)+(-.23,.1)$) {};
  \draw node[inner sep=0pt](tip1) at ($(pic cs:i2)+(1,.3)$){$A\overline{C}$};
  \draw node[inner sep=0pt](tip2) at ($(pic cs:i3)+(-.8,-.5)$){$BC$};

  \draw [arrows={- Latex[length=1.5mm,bend,line width=0pt]}]
  (tip1) edge[bend right=30] (im1.north east);
  \draw [arrows={- Latex[length=1.5mm,bend,line width=0pt]}]
  (tip2) edge[bend right=30] (im2.south);

  \draw node[text width=4cm] at ($(pic cs:t.east)+(3.1,0)$) {
    In $A\overline{C}$, $B$ is excluded because it appears in both true and complement form.

    $Y= A\cdot \overline{C} + B\cdot C$
  }
  ;
\end{tikzpicture}
\end{minipage}

\begin{minipage}{\linewidth}
\begin{tikzpicture}
\begin{tabular}{cccc}

\end{tabular}
\end{tikzpicture}
\end{minipage}


\end{multicols*}
\end{document}
