\documentclass[8pt,a4paper,landscape]{extarticle}
% -- Layout ----
\usepackage[top=0.6cm, bottom=0.6cm, left=0.5cm, right=0.5cm, landscape]{geometry}

% -- Titles ----
\usepackage[
  tiny,                     % text size title
  compact                   % reduce vertical space before/after title
]{titlesec}
% \titlespacing*
\titleformat{\section}{\normalfont\small\bfseries}{\thesection}{0em}{} % Remove space before and after section titles
\titleformat{\subsection}{\normalfont\small\bfseries}{\thesubsection}{0em}{} % Remove space before and after subsection titles
\titlespacing*{\section}{0pt}{0pt}{0pt} % Remove space before/after section titles
\titlespacing*{\subsection}{0pt}{0pt}{0pt} % Remove space before/after subsec titles

% -- Colors ----
\usepackage[dvipsnames]{xcolor}
\definecolor{dmm}{RGB}{192,192,192} % Define a custom dimmed text color
\definecolor{cmt}{RGB}{61,123,123}

% -- Math ------
\usepackage{mathtools}
\usepackage{amssymb}
\usepackage{turnstile}%better vdash

% -- Lists -----
\usepackage[inline]{enumitem}
\setlist{noitemsep}% Remove vspace between items
% Set vspace before and after  list environments as well as the left margin
\setlist[itemize,1]{leftmargin=.6em,labelindent=0pt,labelsep=2pt,
  topsep=1pt,partopsep=1pt}
\setlist[enumerate,1]{leftmargin=1em,labelindent=0pt,labelsep=2pt,
  topsep=1pt,partopsep=1pt}
\setlist[itemize,2]{leftmargin=.3em,labelindent=1pt,topsep=1pt,partopsep=1pt}
\setlist[enumerate,2]{leftmargin=0.2em,labelindent=1pt,topsep=1pt,partopsep=1pt}
\setlist[description]{labelwidth=\linewidth,font=\small\bfseries,leftmargin=1em,topsep=1pt,partopsep=1pt}
% -- Code listing ---
\usepackage{listings}
\lstset{
  aboveskip=3pt,
  belowskip=3pt,
  basicstyle=\small\ttfamily,
  breaklines=true,
  % commentstyle=\upshape\ttfamily,
  captionpos=b,
  commentstyle=\color{cmt},
  columns=flexible,
  frame=single,
  keepspaces=false,
  keywordstyle=\bfseries,
  showspaces=false,
  showstringspaces=false,
  showtabs=false,
  tabsize=2,
}

% Parse Trees
\usepackage{tikz}
\usetikzlibrary{ arrows, automata, bbox, calc, positioning, tikzmark, decorations.pathmorphing, decorations.pathreplacing, decorations.shapes, }
\tikzset{
  >=stealth',
  node distance=1cm,
  recstate/.style={
    circle,draw=blue!50,fill=blue!20,thick,font=\small\sffamily,rounded corners=3pt,
    minimum size=1cm,inner sep=1pt
  },
  ivp/.style={draw,->,auto,font=\small\sffamily,bend angle=60},
  msi/.style={draw=Brown,->,auto,font=\small\sffamily,bend angle=80},
  msibs/.style={draw=RoyalBlue,->,auto,font=\small\sffamily,bend angle=80},
  msinl/.style={font=\footnotesize\sffamily}, %msi node label
  every edge/.style={draw,auto,font=\small\sffamily},
  every loop/.style={looseness=4},
  initial text=start,initial where=right
}

% Place a figure env right here via [H] option
\usepackage{float}

% Side by side figure
\usepackage{subcaption}
% \usepackage{caption}
% \captionsetup{belowskip=0pt, aboveskip=0pt}
\usepackage{pifont}

% -- Multi-Col layout --
\usepackage{multicol}

% No indentation
\setlength\parindent{0pt}
\setlength\abovedisplayskip{-5pt}
\setlength\belowdisplayskip{-5pt}
\setlength\abovedisplayshortskip{-4pt}
\setlength\belowdisplayshortskip{-4pt}
\setlength\tabcolsep{5pt}
\newcommand{\gor}{\;|\;}
\newcommand{\num}{\texttt{\#}~}
\newcommand{\pro}[1]{\textcolor{Brown}{#1}}
\newcommand{\bus}[1]{\textcolor{RoyalBlue}{#1}}
\renewcommand{\arraystretch}{1.2}


\begin{document}
% Suppress page number for all pages
\pagestyle{empty}
% Math env spacing
\setlength{\abovedisplayskip}{1pt}
\setlength{\belowdisplayskip}{1pt}
\setlength{\abovedisplayshortskip}{0pt}
\setlength{\belowdisplayshortskip}{0pt}
% hl
\sethlcolor{dim}

% Notes begin
\begin{multicols*}{3}

% \section*{Decimal $\iff$ Binary (Unsigned, U)}
% \begin{flalign*}
B2U(X) & = \displaystyle\sum_{i=0}^{w-1}x_{i}\cdot 2^{i}, w = \text{number of digits} \\
  10110_{2} & =  1 \times 2^{4} + 0 \times 2^{3} + 1 \times 2^{2} + 1 \times 2^{1} + 0 \times 2^{0}\\
  & = 22 &&
\end{flalign*}
\begin{enumerate}
\item\label{item:1} Divide the N$_{10}$ by 2
\item Get the \textbf{Q}uotient for the next iteration, and
\item Get the \textbf{R}mainder
\item Repeat the above steps until \textbf{Q} becomes 0 (1/2)
\item In the \emph{reverse} order, write all the \textbf{R}s.
\end{enumerate}
\begin{tabular}[h]{l|l|l}
   \hline
    Division by 2 & \textbf{Q}uotient & \textbf{R}emainder \\
    \hline
     333/2 $\Rightarrow$      & 166 & 1 \\
     166/2 $\Rightarrow$      & 83 & 0 \\
     83/2 $\Rightarrow$     & 41 & 1 \\
     41/2 $\Rightarrow$     & 20 & 1 \\
     20/2 $\Rightarrow$     & 10 & 0 \\
     10/2 $\Rightarrow$     & 5 & 0 \\
     5/2  $\Rightarrow$    & 2 & 1 \\
     2/2  $\Rightarrow$    & 1 & 0 \\
     1/2  $\Rightarrow$   & \textbf{0} & 1 \\
    \hline
    \multicolumn{3}{c}{333$_{10}$ = 101001101$_{2}$}\\
    \multicolumn{3}{c}{The above goes for Base$_{n}$ to Base$_{m}$}\\
    \hline
\end{tabular}
\section*{Hexadecimal (Memorize 10, 12, 14, 15)}
\begin{tabular}[h]{r|c|c}
  \hline
Decimal & Binary & Hex \\
  \hline
  0 & 0000 & 0 \\
  1 & 0001 & 1 \\
  2 & 0010 & 2 \\
  3 & 0011 & 3 \\
  4 & 0100 & 4 \\
  5 & 0101 & 5 \\
  6 & 0110 & 6 \\
  7 & 0111 & 7 \\
  8 & 1000 & 8 \\
  9 & 1001 & 9 \\
  10 & \textbf{1010} & \textbf{A} \\
  11 & 1011 & B \\
  12 & \textbf{1100} & \textbf{C} \\
  13 & 1101 & D \\
  14 & \textbf{1110} & \textbf{E} \\
  15 & 1111 & F \\
  \hline
\end{tabular}

\section*{Binary $\iff$ Hexadecimal}
As shown above, 1 hex digit = 4 binary digits
\begin{flalign*}
  2ED_{16} = \underbracket{0010}_{2_{16}}\overbracket{1110}^{E_{16}}\underbracket{1101}_{D_{16}} \quad(\text{Binary}) &&
\end{flalign*}
Binary-to-hex starts from right of the binary:
\begin{flalign*}
  (\text{Binary})\quad \overbracket{111}^{7_{16}}\underbracket{1010}_{A_{16}} = 7A_{16} &&
\end{flalign*}

\section*{Sign/Magnitude numbers}
\begin{equation*}
  \underbracket{0}_{\mathclap{\text{most significant bit as sign}}}101_{2} = 5_{10}\qquad\qquad\qquad\qquad 1\overbracket{101}_{2}^{\mathclap{\text{remaining } N-1 \text{ bits as magnitude (abs value)}}} = -5_{10}
\end{equation*}

\section*{Two's Complement (T)}
Most significant bit (MSB) indicates sign (same as above):
\begin{itemize}
\item 0 for non-negative (including zero)
\item 1 for negative
\end{itemize}
\begin{flalign*}
B2T(X) & = -x_{w-1}\cdot 2^{w-1} + \displaystyle\sum_{i=0}^{w-2}x_{i}\cdot 2^{i} \\
  10110_{2} & = 1 \times -(2^{4}) + 0 \times 2^{3} + 1 \times 2^{2} + 1 \times 2^{1} + 0 \times 2^{0}\\
       & = -16 + 0 + 4 + 2 + 0 \\
  & = -10_{10} &&
\end{flalign*}
\begin{tabular}[h]{c|c|c|c}
  \hline
  0 & -1 & most positive & most negative \\
  \hline
  $00\ldots 000_{2}$ & 11\ldots 111$_{2}$ & $01\ldots 111_{2} = 2^{N-1} - 1$ & $10\ldots 000_{2} = -2^{N-1}$ \\
  \hline
\end{tabular}
\section*{Taking the Two's Complement}
% Method One
\begin{minipage}{0.5\linewidth}
  Method A
\begin{enumerate}
\item\label{item:2} invert all the bits in the number
\item add 1 to the \emph{least significant} bit (LSB) position
\end{enumerate}
\end{minipage}
% Method Two
\begin{minipage}{0.5\linewidth}
  Method B
\begin{enumerate}
\item Check MSB for the sign (1-/0+)
\item Treat the binary number as unsigned and get its decimal value
\item Use value from above steps to mod $2^{w}$ until remainder $<2^{w}$
\end{enumerate}
\end{minipage}

\begin{minipage}{0.48\linewidth}
-2$_{10}$ as a 4-bit two's complement number
\begin{align*}
  2_{10} & = 0010_{2} \\
         & \xrightarrow{\text{invert all bits}} 1101_{2}\\
         & \xrightarrow{\text{add 1 to LSB}} 1110_{2} = -2_{10}
\end{align*}
\end{minipage}
\begin{minipage}{0.5\linewidth}
Find decimal value of two's compl. number $1001_{2}$
\begin{align*}
  1001_{2} & \Rightarrow{\text{has leading 1, must be negative}} \\
           & \xrightarrow{\text{invert all bits}} 0110_{2} \\
           & \xrightarrow{\text{add 1 to LSB}} 0111_{2} = -7_{10}
\end{align*}
\end{minipage}
% Addition
\begin{minipage}{0.5\linewidth}
\begin{align*}
  -2_{10} + 1_{10} & = 1110_{2} + 0001_{2} \\
                   & = 1111_{2} = -1_{10} \\
  -7_{10} + 7_{10} & = 1001_{2} + 0111_{2} \\
                   & = 10000_{2} \\
                   &\xrightarrow{\text{discard carry}} \dimtxt{1}0000_{2} \\
  & = 0_{10}
\end{align*}
\end{minipage}
% Subtraction
\begin{minipage}{0.5\linewidth}
\begin{align*}
  5_{10} - 3_{10} & = 5_{10} + (-3_{10}) = 0101_{2} + 1101_{2} \\
                  & = 10010_{2} \xrightarrow{\text{discard carry}} \dimtxt{1}0010_{2}\\
                  & = 2_{10} \\
  3_{10} - 5_{10} & = 3_{10} + (-5_{10}) = 0011_{2} + 1011_{2} \\
                  & = 1110_{2} \quad (\text{must be negative}) \\
                  & = -2_{10}
\end{align*}
\end{minipage}

\begin{itemize}
\item Adding 2 $N$-bit positive or negative numbers may cause overflow ($> 2^{N-1} \text{or} < -2^{N-1}$)
\item Adding a positive to a negative \emph{never} cause overflow
\item A carry out of most significant bit does \emph{not} indicate overflow
\item Overflow occurs when two numbers have the same sign bit and result has the opposite sign bit
\end{itemize}
\begin{tabular}[h]{l|l|l}
  \hline
  System & Min  & Max \\
  \hline
  Unsigned & 0 & $2^{N} - 1$ \\
  \hline
  Sign/Magnitude & $-2^{N-1} + 1$ & $2^{N-1} - 1$ \\
  \hline
  Two's Complement & $-2^{N-1}$ & $2^{N-1} - 1$ \\
  \hline
\end{tabular}

% Extension and truncation
\textbf{Sign extension}: Extend $w$-bit signed integer to \emph{w}+\emph{k}-bit integer with same value:
\begin{itemize}
\item Make $k$ copies of sign bit for signed integer
\[
X' = \underbracket{x_{w-1},\ldots,x_{w-1}}_{k\; \text{copies of MSB}},x_{w-1},x_{w-2},\ldots,x_{0}
\]
\item Prepend $k$ copies of zeros for unsigned integer
\[
X' = \underbracket{0,\ldots,0,}_{k\; \text{copies of }0}x_{w-1},x_{w-2},\ldots,x_{0}
\]
\end{itemize}

\begin{flalign*}
  3_{10} = 0011_{2} \quad\text{(4-bit)} \xrightarrow{\text{copy sign bit into 3 new upper bits}} 0000011_{2} \quad\text{(7-bit)} \\
  -3_{10} = 1101_{2} \quad\text{(4-bit)} \xrightarrow{\text{copy sign bit into 3 new upper bits}} 1111101_{2} \quad\text{(7-bit)} \\
\end{flalign*}
\textbf{Truncation} Truncate k+$w$-bit un-/signed integer $X$ to $w$-bit $X'$.
\begin{itemize}
\item For small un-/signed numbers, result remains the same (mod 2$^{w}$ until remainder $< 2^{w}$)
\begin{align*}
  2_{10}  & = 00010_{2} \xrightarrow[\text{truncate MSB}]{2 \modu 16} \dimtxt{0}0010_{2} = 2_{10} \\
  -6_{10} & = 11010_{2} \xrightarrow[\text{truncate MSB}]{-6 \modu 16 = 26\text{U} \modu 16 = 10\text{U} = -6 } \\
  & = \dimtxt{1}1010_{2} = -6_{10}\qquad
\end{align*}
\item For larger numbers, sign will change after truncation

  \begin{align*}
  10_{10} & = 01010_{2} \xrightarrow[\text{truncate MSB}]{2 \modu 16} \dimtxt{0}1010_{2} = -6_{10} \\
  -10_{10} & = 10110_{2} \xrightarrow[\text{truncate MSB}]{-10 \modu 16 = 22\text{U} \modu 16 = 6\text{U} = 6 } \\
  & = \dimtxt{1}0110_{2} = 6_{10}
\end{align*}
\end{itemize}


%TODO: Full adder here?
% \pagebreak

% \section*{Combinational v.s. Sequential}
% \begin{tabular}[h]{ll}
  \hline
  Type & Outputs depend on?  \\
  \hline
  Combinational & \emph{only} current inputs (combined inputs, memoryless)  \\
  Sequential & both current \emph{and} previous inputs (input sequence, memory) \\
  \hline
\end{tabular}


% \section*{MOSFETS (pMOS, nMOS, CMOS)}
% \begin{itemize}
\item lowest voltage = 0V, called \emph{ground} or GND (sometimes, V$_{SS}$)
\item highest voltage dec. from 5V (1970s-80s) to $\leqslant$1.2V, called V$_{DD}$
\item \emph{short circut} occurs when both pull-up and pull-down network ON
\item output \emph{floats} when both pull-up and pull-down network OFF
\begin{tabular}[h]{cllcl}
  \hline
  Gate & nMOS & pMOS & Pass well & Network  \\
  \hline
  0/LOW/GND & OFF & ON  & 0 &  pull-dow\textbf{n}\\
  1/HIGH/V$_{DD}$ & ON & OFF & 1 & pull-u\textbf{p}\\
  \hline
\end{tabular}
\item CMOS design prefers NANDs and NORs (\emph{not associative})
\item wide NAND/NOR gates can't use chain/tree strategy
\item Any logic function can be implemented using only NANDs/NORs
\item Wide NANDs and NORs use trees of smaller devices
\end{itemize}

% Handy function to show the positions of nodes
\def\normalcoord(#1){coordinate(#1)}
\def\showcoord(#1){coordinate(#1) node[circle, red, draw, inner sep=1pt,
pin={[red, overlay, inner sep=0.5pt, font=\tiny, pin distance=0.1cm,
pin edge={red, overlay}]45:#1}](){}}
\let\coord=\normalcoord
\let\coord=\showcoord % coordinates markers
\def\killdepth#1{{\raisebox{0pt}[\height][0pt]{#1}}} % baseline tweaks

% Two-input NAND gate
\begin{minipage}{0.4\linewidth}
  \begin{circuitikz}
    [scale=0.7,
    transform shape,
    information text/.style={inner sep=1ex}]
  % Nodes
  \draw (2,1.5) node[pmos](p2){\large p2};
  \draw (3.5,1.5) node[pmos](p1){\large p1};
  \draw (3.5,0) node[nmos](n1){\large n1};
  \draw (3.5,-1.2) node[nmos](n2){\large n2} (3.5,-1.5) node[ground](GND){};
  \draw (0,0) node[left](A){A};
  \draw (0,-1.2) node[left](B){B};
  % Wires
  \draw (A) -- (n1.G);
  \draw (B) -- (n2.G);
  \draw (p1.G) -- (n1.G);
  % from p2.G to the intersection of two lines:
  % vertical line that passes p2.G and horizontal line that passes n2.G
  \draw (p2.G) -- (p2.G |- n2.G);
  % I/O
  \draw (p2.D) -- (p1.D) -- ++(0.5,0) node[circ]{} +(0.1,0) node[right](y){$Y$};
  \draw ($(p2.G |- p2.S)+(.7,0)$) -- (p2.S) -- +(1,0) node[above]{\large $V_{DD}$} -- (p1.S) -- +(0.3,0);

  % Caption
  \draw (GND) ($(GND) - (2,0)$) node[below, text width=2.5cm]
  {Two-input NAND gate schematic};
  % Information text, see https://tikz.dev/tutorial#sec-2.21
  % Whenever possible, use it
\end{circuitikz}
\end{minipage}
\begin{minipage}{0.55\linewidth}
  The nMOS transistors n1 and n2 are connected in series;\\
  The pMOS transistors p1 and p2 are in parallel.\\
  This acts as the base for multiple-input NAND gate schematic.
  For example, 4-input type would have 4 pMOS transistors in parallel
  and 4 nMOS transistors in series.
\end{minipage}
\begin{tabular}[h]{ccccc}
  \hline
  A & B & Pull-Dow\textbf{n} network & Pull-U\textbf{p} network & Y  \\
  \hline
  0 & 0 & OFF & ON  & 1 \\
  0 & 1 & OFF & ON  & 1 \\
  1 & 0 & OFF & ON  & 1 \\
  1 & 1 & ON & OFF  & 0 \\
  \hline
\end{tabular}

% Two-input NOR gate
\begin{minipage}{.4\linewidth}
\begin{circuitikz}
  [scale=0.7,
    transform shape,
    information text/.style={inner sep=2em}]
    % Input A is the origin
    \node at (0, 0) [left](A){A};
    \node at (0, -1.2) [left](B){B};
    \node at (3,0) [pmos](p1){\large p1};
    \node at (3,-1.2) [pmos](p2){\large p2};
    \node at (1.5,-2.7) [nmos](n1){\large n1};
    \node at (3,-2.7) [nmos](n2){\large n2};
    \node at (p1.S) [tground](vdd){};
    \node at ($(p1.S)+(0,.1)$) [above]{V$_{DD}$};
    \node at (n2.S) [sground](gnd){};
    \node at (gnd.east) [left, yshift=-.2cm, xshift=-.2cm, text width=2.5cm]{Two-input NOR Gate Schematic};

    \draw
    (A) -- (p1.G)
    (B) -- (p2.G)
    (p2.D) -- (n2.D)
    (n1.D) -- (n2.D) to[short, o-*] (n2.D) -- +(.5,0) node[right](Y){Y}
    (n1.S) -- (n2.S)
    (n1.G) -- (n1.G |- p1.G)
    (n2.G) -- (n2.G |- p2.G);
\end{circuitikz}
\end{minipage}
\begin{minipage}{.55\linewidth}
  The NOR gate should produce a 0 output if either input is 1.
  Hence, the pull-down network should have two nMOS transistors in parallel.
  By the conduction complements rule, the pMOS transistors must be in series.
\end{minipage}


% \section*{Functional Specifications (Boolean Equations)}
% \begin{enumerate}
\item In a given truth table, look for rows that have an output of 1/TRUE
\item\label{bf:step2} For each such row, write a boolean equation such that inputs $\rightarrow$ 1/TRUE; usually using AND (multiplication $A\cdot B$) and NOT ($\overline{B}$)
\item Chain all the terms in \ref{bf:step2} using OR (production +)
  \[
Y = \overbracket{\overline{C}\overline{B}A + \overline{C}BA + \underbracket{CB\overline{A}}_{\mathclap{\text{product/implicant}}} + CBA}^{\text{sum of products}}
  \]
\item sum-of-products also called \emph{two-level logic} (AND connected to SUM)
\end{enumerate}
\begin{itemize}
\item chain: propagation delays increase linearly with number of inputs
\item tree: propagation delays increase logarithmatically with number of inputs
\item A boolean equation can be the sum of minterms (0-indexed) (but not necessarily the shortest): $f(A,B) = \Sigma(m_{i},m_{k},\ldots)$
\item A boolean equation can be the product of maxterms (0-index): $F(A,B) = \Pi(m_{i},m_{k},\ldots)$; each $m$ is FALSE
\item To simplify a boolean equation, use boolean algebra or K-map
\item To simplify a truth table, use ``don't care'' $X$
\item Simplifying a boolean equation for digital circuits may cause hazards
\end{itemize}

\begin{tabular}{l|c|c|c}
  \hline
  \multicolumn{4}{c}{$Y = \overline{A}B + B$} \\
  \hline
  literals & true form & complementary form & product/implicant \\
  \hline
  A, $\overline{A}$, B & A & $\overline{A}$ & $\overline{A}B$\\
  \hline
  \multicolumn{4}{l}{\emph{true form} does NOT mean A is TRUE, just that A has no overline} \\
  \hline
  \textbf{implicant} & \multicolumn{3}{l}{The AND of one or more literals, also \emph{product} } \\
  \textbf{sum} & \multicolumn{3}{l}{The OR of one or more literals} \\
  \textbf{minterm} & \multicolumn{3}{l}{a TRUE product involving all inputs to the boolean fn: $\overline{A}B$}\\
  \textbf{maxterm} & \multicolumn{3}{l}{a FALSE sum involving all inputs to the boolean fn: $A + B$}\\
  \hline
\end{tabular}
\begin{tikzpicture}[framed]

\end{tikzpicture}


% \section*{Bubble Pushing (backward: inputs $\leftarrow$ output)}
% % Step 1
\begin{enumerate}
  \item Begin at output and work \emph{back} toward inputs.
  \item Work on one gate each time, use Demorgan's law
  \item NAND(AB)

% NAND to NOT (OR)
\begin{circuitikz}
  [transform shape,
  information text/.style={inner sep=1ex}]
  \ctikzset{
      logic ports=ieee,
      logic ports/scale=0.6,
   }

   \draw (-2,.6) node[nand port](nd1){}
   (nd1.in 1) -- +(0,0) node[left](A){A}
   (nd1.in 2) -- +(0,0) node[left](B){B}
   (nd1.out) -- +(.2,0) node[right](o1){C$_{out}$};
   \draw [-{Latex[length=2mm]}] (o1.east) -- ++(1,0)
   node[above](tip1){$\qquad\overline{A\cdot B} = \overline{A} + \overline{B}$} --
   ++(1.5,0) node[](emp1){};

   \draw (emp1.east)  +(1.5,0) node[or port](or1){};
   \draw (or1.bin 1) -- +(-.05,0) node[ocirc]{};
   \draw (or1.bin 2) -- +(-.05,0) node[ocirc]{};
   \draw (or1.in 1) -- +(-.1,0) node[left]{A};
   \draw (or1.in 2) -- +(-.1,0) node[left]{B};
   \draw (or1.out) +(.2,0) node[right]{C$_{out}$};
\end{circuitikz}

\item NOR(AB)

% NOR to AND(NOT)
\begin{circuitikz}
  [transform shape,
  information text/.style={inner sep=1ex}]
  \ctikzset{
      logic ports=ieee,
      logic ports/scale=0.6,
   }

   \draw (-2,.6) node[nor port](nr2){}
   (nr2.in 1) -- +(0,0) node[left](A){A}
   (nr2.in 2) -- +(0,0) node[left](B){B}
   (nr2.out) -- +(.2,0) node[right](o2){C$_{out}$};
   \draw [-{Latex[length=2mm]}] (o2.east) -- ++(1,0)
   node[above](tip1){$\qquad\overline{A+B} = \overline{A} \cdot \overline{B}$} --
   ++(1.5,0) node[](emp2){};

   \draw (emp2.east)  +(1.5,0) node[and port](ad1){};
   \draw (ad1.bin 1) -- +(-.05,0) node[ocirc]{};
   \draw (ad1.bin 2) -- +(-.05,0) node[ocirc]{};
   \draw (ad1.in 1) -- +(-.1,0) node[left]{A};
   \draw (ad1.in 2) -- +(-.1,0) node[left]{B};
   \draw (ad1.out) +(.2,0) node[right]{C$_{out}$};
 \end{circuitikz}

\item Bubble canceling
\end{enumerate}

\begin{circuitikz}[framed]
  [transform shape,
  information text/.style={inner sep=1ex}]
  \ctikzset{
      logic ports=ieee,
      logic ports/scale=0.5,
   }

   % original (left)
   \draw (-2,.6) node[nand port](nd1){}
   (nd1.in 1) -- +(0,0) node[left](a1){A}
   (nd1.in 2) -- +(0,0) node[left](b1){B}
   (nd1.out) -- ++(.3,0) node[anchor=east](emp1){}
   node[nand port, anchor=in 1](nd2){}
   (emp1) -- +(.2,0) -- (nd2.in 1)
   (nd2.in 2) -- +(-.2,0) node(emp2){}
   (emp2.center) -- +(0,-.3)node(emp3){} -- (nd1.in 2 |- emp3.center) node[left](c1){C}
   (nd2.out) -- +(0.1,0) node[right](y1){Y};

   % arrow
   \draw ($(nd2.out)+(.5,.5)$) node [above](note1){push this bubble first};
   \draw [-{Latex[length=2mm]}] (note1.south) -- (nd2.bout);

   \draw [-{Latex[length=2mm]}] (y1.east) -- ++(.3,0)
   node[above,xshift=2em](tip1){$\overline{D\cdot C} = \overline{D} + \overline{C}$} --
   ++(1.5,0) node(emp3){}
   node[below,below of=tip1,yshift=.5cm](tip2){$D=\overline{A\cdot B}$};


   % and to nand (left below)
   \draw ($(nd1.bin 1)+(0,-1.3)$) node[and port,scale=0.9](ada){};
   \draw [-{Latex[length=2mm]}] (ada.bout)  ++(.4,0)
   node(t1){} -- +(.5,0) node(){};
   \draw ($(ada.out)+(1.5,0)$) node[nand port,scale=0.9](na){};
   \draw (na.out) -- +(.2,0) node[not port,anchor=in,scale=0.9](noa){};

   \draw ($(nd1.bin 1)+(0,-2)$) node[or port,scale=0.9](ora){};
   \draw [-{Latex[length=2mm]}] (ora.bout)  ++(.4,0)
   node(t1){} -- +(.5,0) node(){};
   \draw ($(ora.out)+(1.5,0)$) node[nor port,scale=0.9](nb){};
   \draw (nb.out) -- +(.2,0) node[not port,anchor=in,scale=0.9](noa){};

   % bubble identifying (right)
   \draw ($(emp3)+(1,0)$) node[nand port,anchor=bin 2](nd3){}
   (nd3.in 1) -- +(0,0) node[left](a2){A}
   (nd3.in 2) -- +(0,0) node[left](b2){B}
   (nd3.out) -- ++(.3,0) node[anchor=east](emp4){}
   node[or port, anchor=in 1](or4){}
   (or4.bin 1)  +(-.05,0) node[ocirc](or4b1){}
   (or4.bin 2)  +(-.05,0) node[ocirc]{}

   (emp4) -- +(.2,0) -- (or4.in 1)
   (or4.in 2) -- +(-.2,0) node(emp5){}
   (emp5.center) -- +(0,-.3)node(emp6){} -- (nd3.in 2 |- emp6.center) node[left](c2){C}
   (or4.out) -- +(0,0) node[right](y2){Y};

   \draw ($(nd3.out)+(.2,.5)$) node [above](note2){can cancel each other};
   \draw [-{Latex[length=1.5mm]}] (note2.south) -- (nd3.bout);
   \draw [-{Latex[length=1.5mm]}] (note2.south) -- (or4b1);


   % bubble canceling (right below)
   \draw ($(nd3.bin 2)+(0,-1.2)$) node[and port,anchor=bin 2](ad1){}
   (ad1.in 1) -- +(0,0) node[left](a3){A}
   (ad1.in 2) -- +(0,0) node[left](b3){B}
   (ad1.out) -- ++(.3,0) node[below,anchor=east](emp7){}
   node[or port, anchor=in 1](or5){}
   (or5.bin 1)  +(-.05,0) node{}
   (or5.bin 2)  +(-.05,0) node[ocirc]{}

   (emp7) -- +(.2,0) -- (or5.in 1)
   (or5.in 2) -- +(-.2,0) node(emp7){}
   (emp7.center) -- +(0,-.3)node(emp8){} -- (ad1.in 2 |- emp8.center) node[left](c3){C}
   (or5.out) -- +(0,0) node[right](y2){Y};
\end{circuitikz}


% \section*{Karnaugh Maps (K-Maps works well for up to 4 vars)}
% \begin{itemize}
\item each square in K-map corresponds to a row in Truth table
\item each square in K-map contains value of output for that row
\item adjacent squares share all the same literals except one
\item a var's true and compl. forms all in circle, it's excluded from implicant
\item K-map is cyclic, left edge adjacent to right,  top adjacent to bottom
\end{itemize}

Rules for finding a minimized equation from a K-map
\begin{enumerate}
\item draw recton K-Map where sum of squares (boolean fn) evals to 1
\item draw \textbf{fewest} rects (largest, greedy) necessary to cover all 1s
  \begin{enumerate}
  \item circle the greatest possible rectangle (rule 2,3,4)
  \item repeat until all prime implicants circled
  \end{enumerate}
\item rect has a width and length that must be a power of 2: 1,2,4
\item rect can overlap other implicants (rule 1)
\item a prime implicant is not completely contained in any other implicant
\item implicant can be uniquely identified by a single product term
\item the larger the implicant, the smaller the product term (rule 2)
\item each rect keeps vars that appear only in true or complement forms
\item If a var is 0, use its complement form ($\overline{A}$)
\end{enumerate}
In the below K-map, the minimal SOP, with the 3rd column (in blue circle) removed, may cause
hazards/glitches in digital circuits. See ENGN6213 Lab1-3, ARM section 2.9.2, and \href{https://ocw.mit.edu/courses/6-004-computation-structures-spring-2017/pages/c4/c4s2/c4s2v5/}{MIT6.004-4.2.5 Karnaugh}

\begin{minipage}{\linewidth}
  \rowcolors{1}{Gray}{white}
  \begin{tabular}{|c|c|c|c|c|}
    \hline
    C\textbackslash AB & 00 & 01 & 11 & 10\tikzmark{t} \\
    \hline
    \hiderowcolors
  \cellcolor{Gray}0 & 0 & 0 & 1\tikzmark{i1} & 1\tikzmark{i2} \\
  \hline
  \cellcolor{Gray}1 & 0 & 1\tikzmark{i3} & 1\tikzmark{i4} & 0 \\
  \hline
\end{tabular}
\begin{tikzpicture}[remember picture, overlay]
  \tikzset{
    shape example/.style={color=red, draw, line width=1pt, inner sep=0pt, minimum width=1cm,minimum height=0.3cm,anchor=west,rounded corners=1ex},
    hazard/.style={color=blue,draw,line width=0.5pt,inner sep=0pt}
  }
  \draw node[name=im1,shape=rectangle,shape example] at ($(pic cs:i1)+(-.23,.1)$) {};
  \draw node[name=im2,shape=rectangle,shape example] at ($(pic cs:i3)+(-.23,.1)$) {};
  \draw node[inner sep=0pt](tip1) at ($(pic cs:i2)+(.8,.3)$){$A\overline{C}$};
  \draw node[inner sep=0pt](tip2) at ($(pic cs:i3)+(-.8,-.3)$){$BC$};

  % Hazard
  \draw[hazard] ($(pic cs:i1)+(-.2,.2)$) rectangle ($(pic cs:i4)+(.1,-.1)$);

  \draw [arrows={- Latex[length=1.5mm,bend,line width=0pt]}]
  (tip1) edge[bend right=30] (im1.north east);
  \draw [arrows={- Latex[length=1.5mm,bend,line width=0pt]}]
  (tip2) edge[bend right=30] (im2.south);

  \draw node[text width=4cm] at ($(pic cs:t.east)+(3,0)$) {
   $B$ is excluded from $A\overline{C}$, as it is in both true and complementary form.

    $Y= A\overline{C} + BC$
  }
  ;
\end{tikzpicture}
\end{minipage}
\vspace{1.2em}

\begin{minipage}{\linewidth}
  \rowcolors{1}{Gray}{white}
  \begin{tabular}{|c|c|c|c|c|}
    {\small $\frac{AB\rightarrow}{CD\downarrow}$} & 00 & 01 & 11 & 10 \\
    \hline
    \hiderowcolors
    \cellcolor{Gray}00 & 0 & \tikzmark{a}1 & \tikzmark{b}1 & \tikzmark{c}1 \\
    \hline
    \cellcolor{Gray}01 & \tikzmark{d}1 & \tikzmark{e}1 & \tikzmark{f}1 & \tikzmark{g}1 \\
    \hline
    \cellcolor{Gray}11 & \tikzmark{h}1 & 1 & 1 & \tikzmark{i}1 \\
    \hline
    \cellcolor{Gray}10 & \tikzmark{j}1 & 0 & 0 & \tikzmark{k}1 \\
    \hline
  \end{tabular}
  \begin{tikzpicture}[remember picture,overlay]
    \tikzset{shape example/.style=
      {color=red, draw, line width=1pt, inner sep=0pt, minimum width=1cm,minimum height=0.3cm,anchor=west,rounded corners=1ex},
      wraps/.style={color=red,draw,line width=1pt,inner sep=0pt},
      pointer/.style={color=cyan,line width=.5pt,arrows={-Latex[length=1.1mm]}},
    }
    \draw[shape example] ($(pic cs:a)+(-.1,.25)$) rectangle ($(pic cs:f)+(.2,.-.05)$);
    \draw[shape example] ($(pic cs:b)+(-.1,.2)$) rectangle ($(pic cs:g)+(.2,.-.1)$);
    \draw[shape example] ($(pic cs:d)+(-.1,.22)$) rectangle ($(pic cs:i)+(.22,.-.05)$);
    \draw[wraps] ($(pic cs:i)+(-.1,.2)$) -- ($(pic cs:i)+(.5,.2)$)
    ($(pic cs:i)+(-.1,.2)$) -- ($(pic cs:k)+(-.1,-.2)$) -- ($(pic cs:k)+(.5,-.2)$);
    \draw[wraps] ($(pic cs:h)+(.3,.2)$) -- ($(pic cs:h)+(-.3,.2)$)
    ($(pic cs:h)+(.3,.2)$) -- ($(pic cs:j)+(.3,-.2)$) -- ($(pic cs:j)+(-.3,-.2)$);

    \draw node[text width=4cm](mbf) at ($(pic cs:g.east)+(3,.1)$) {
      $Y= D + B\overline{C} + A\overline{C} + \overline{B}C$
    };

    % D
    \path[pointer] (pic cs:i) edge[bend right,out=-20,in=-100] ($(mbf.west)+(.9,-.15)$);

    % B AND (NOT C)
    \path[pointer] ($(pic cs:a)+(0,.2)$) edge[bend left,out=40,in=140] ($(mbf.center)+(-.4,.2)$);

    % A AND (NOT C)
    \path[pointer] ($(pic cs:c)+(.1,.2)$) edge[bend left,out=20,in=105] ($(mbf.center)+(.42,.2)$);

    % (NOT B) AND C
    \path[pointer] ($(pic cs:k)+(.1,0)$) edge[bend left,out=-5,in=-90] ($(mbf.east)+(-1,-.2)$);
  \end{tikzpicture}

  \vspace{.5em}
  In the above K-map, the rightmost column can also be counted as an implicant, leading to a different sum of products $Y=D + B\overline{C} + \textcolor{red}{A\overline{B}} + \overline{B}C$. Minimal SOP is \emph{not} necessarily unique.
\end{minipage}


% \section*{Multiplexer(MUX), Decoder, Timing, Glitches}
% % MUX
\begin{tabular}[h]{ccc|c}
  S & A & B & Y\\
  \hline
  0 & 0 & X & 0\\
  0 & 1 & X & \tikzmark{m}1\\
  1 & X & 0 & 0\\
  1 & X & 1 & 1\\
\end{tabular}
\begin{circuitikz}[remember picture,overlay,scale=0.8, transform shape]
  \tikzset{
    mux2by1/.style={muxdemux, muxdemux def={Lh=3.2,Rh=1.5,NL=2,NT=1,NB=0,NR=1, w=1.2}},
    mux4by1/.style={muxdemux, muxdemux def={Lh=3.2,Rh=2.5,NL=4,NT=2,NB=0,NR=1, w=1.5}}
  }

  % 2by1 mux
  \node [mux2by1,anchor=west](mux) at ($(pic cs:m)+(1.2,0)$) {};
  \draw (mux.lpin 1) -- +(0,0) node[left](A){A};
  \draw (mux.blpin 1) +(.05,0) node[right]{0};
  \draw (mux.lpin 2) -- +(0,0) node[left](B){B};
  \draw (mux.blpin 2) +(.05,0) node[right]{1};
  \draw (mux.tpin 1) -- +(0,-.05) node[above](S){\textbf{S}};
  \draw (mux.rpin 1) -- +(0,0) node[right](Y){Y};

  \node [text width=3cm] at ($(Y)+(1.2,.8)$) {
    $N$:1 mux usually needs $\log_{2}N$ select lines
  };

  \node [text width=3cm] at ($(Y)+(2,-.5)$) {
    $2^N$-input mux can be programmed to perform any $N$-input logic function
  };
  % 2-input logic function
  \node [mux4by1,anchor=west](mux2) at ($(Y)+(5,0)$) {};
  \draw (mux2.lpin 1) -- +(-1,0) node(em1){} -- (em1 |- mux2.lpin 4) node[sground]{};
  \draw (mux2.lpin 2) -- +(-1,0);
  \draw (mux2.lpin 3) -- +(-1,0);
  \draw (mux2.lpin 4) -- +(-.3,0) node[rground,yscale=-1]{};
  \draw (mux2.blpin 1) +(.05,0) node[right]{00};
  \draw (mux2.blpin 2) +(.05,0) node[right]{01};
  \draw (mux2.blpin 3) +(.05,0) node[right]{10};
  \draw (mux2.blpin 4) +(.05,0) node[right]{11};
  \draw (mux2.tpin 1) -- +(0,0) node[above]{A};
  \draw (mux2.tpin 2) -- +(0,.1) node[above]{B};
  \draw (mux2.rpin 1) -- +(0,0) node[right]{Y};
\end{circuitikz}

\vspace{.5em}
See ARM book (Example 2.12) for making 8:1 MUX $\rightarrow $ 3-input MUX

% Decoder
\begin{tabular}[h]{p{.2cm}p{.2cm}|p{.2cm}p{.2cm}p{.2cm}p{.2cm}}
  $A_1$ & $A_0$ & $Y_3$ & $Y_2$ & $Y_1$ & $Y_1$\\
  \hline
  0 & 0 & 0 & 0 & 0 & 1\\
  0 & 1 & 0 & 0 & 1 & \tikzmark{t}0\\
  1 & 0 & 0 & 1 & 0 & 0\\
  1 & 1 & 1 & 0 & 0 & \tikzmark{t2}0\\
\end{tabular}
\begin{circuitikz}[remember picture,overlay,scale=0.8,transform shape]
  \tikzset{
    decoder/.style={muxdemux, muxdemux def={Lh=3.5,Rh=3.5,NL=2,NB=0,NR=4, inset Rh=1,w=2.2}},
  }
  \ctikzset{
    logic ports=ieee,
    logic ports/scale=0.5,
  }
  \node [decoder](dec1) at ($(pic cs:t)+(2,0)$) {};
  \node [text width=6cm] at ($(pic cs:t2)+(-.2,-.8)$) {
    $N$-input fn with $M$ 1s in truth table $\rightarrow$ an $N$:$2^N$ decodr + $M$-input OR gate (attached to all minterms)
  };
  % 2:4 Decoder
  \draw (dec1.lpin 1)  +(0,0) node[left](a1){$A_1$};
  \draw (dec1.lpin 2)  +(0,0) node[left](b1){$A_0$};
  \draw (dec1.rpin 1)  +(0,0) node[right](y3){$Y_3$};
  \draw (dec1.rpin 2)  +(0,0) node[right](y2){$Y_2$};
  \draw (dec1.rpin 3)  +(0,0) node[right](y1){$Y_1$};
  \draw (dec1.rpin 4)  +(0,0) node[right](y0){$Y_0$};
  \draw (dec1.brpin 1)  +(0,0) node[left]{$11$};
  \draw (dec1.brpin 2)  +(0,0) node[left]{$10$};
  \draw (dec1.brpin 3)  +(0,0) node[left]{$01$};
  \draw (dec1.brpin 4)  +(0,0) node[left]{$00$};
  \draw (dec1.top)  +(0,.4) node[below](ration){2:4 Decoder};

  % logic function using 2:4 Decoder
  \node [decoder](dec2) at ($(dec1.right)+(3,0)$) {};
  \draw (dec2.top)  +(0,.4) node[below](name2){2:4 Decoder};
  \draw (dec2.bottom)  +(-.2,-.2) node[below](logic){$Y=\overline{A\oplus B}$};
  \draw (dec2.lpin 1)  +(0,0) node[left](a){$A$};
  \draw (dec2.lpin 2)  +(0,0) node[left](b){$B$};
  \draw (dec2.rpin 1) --  +(.8,0) node[right](o1){$AB$};
  \draw (dec2.rpin 2) --  +(.8,0) node[right](o2){$A\overline{B}$};
  \draw (dec2.rpin 3) --  +(.8,0) node[right](o3){$\overline{A}B$};
  \draw (dec2.rpin 4) --  +(.8,0) node[right](o4){$\overline{AB}$};
  \draw (o1)  +(0,.1) node[above](tip){Minterm};

  \draw (dec2.brpin 1)  +(0,0) node[left]{$11$};
  \draw (dec2.brpin 2)  +(0,0) node[left]{$10$};
  \draw (dec2.brpin 3)  +(0,0) node[left]{$01$};
  \draw (dec2.brpin 4)  +(0,0) node[left]{$00$};

  \draw (dec2.bottom right) +(.5,-.5) node[or port,rotate=-90](or){};
  \draw (or.in 2) -- (or.in 2 |- dec2.rpin 1);
  \draw (or.in 1) -- (or.in 1 |- dec2.rpin 4);
  \draw (or.out) +(0,0) node[below]{Y};
\end{circuitikz} \\


\vspace{1cm}
% \begin{circuitikz}
  \ctikzset{
    logic ports=ieee,
    logic ports/scale=0.6,
  }

  \node[](D) at (0, 0){D};
  \node[](C) at (0, 0.4){C};
  \node[](A) at (0, 1.2){A};
  \node[and port,anchor=in 1] (ad1) at ($(A)+(.2,0)$) {};
  \node[](B) at ($(ad1.in 2)+(-.2,0)$) {B};
  \node[or port,anchor=in 2] (or) at ($(C)+(2.2,0)$) {};
  \node[and port,anchor=in 2] (ad2) at ($(D)+(4.2,0)$) {};

  \draw (C) -- (or.in 2);
  \draw[color=gray,line width=1pt] (D) -- (ad2.in 2) -- (ad2.bin 2);
  \draw (ad2.out) +(0,0) node[right](y){Y};
  \draw[color=cyan,line width=1pt] (ad1.bin 1) -- (ad1.in 1);
  \draw[color=cyan,line width=1pt] (ad1.bin 2) -- (ad1.in 2);
  \draw[color=cyan,line width=1pt] (ad1.bout) -- (ad1.out)
  -- ++(.2,0) node[above,color=cyan](inset1){n1} --
  (inset1 |- or.in 1) -- (or.in 1) -- (or.bin 1);

  \draw[color=cyan,line width=1pt] (or.bout) -- (or.out)
  -- ++(.2,0) node[above,color=cyan](inset2){n2}
  -- (inset2 |- ad2.in 1) -- (ad2.in 1) -- (ad2.bin 1);

  \draw[<-,cyan] (or.in 1) -- +(.3,.6) node[above]{\small Critical Path (longest,slow)};
  \draw[<-,gray] (ad2.in 2) ++(-2,0) -- +(0,-.3) node[below]{\small Short Path (fast)};

  \node[text width=5cm](info) at ($(y)+(.5,1)$) {
    \begin{align}
      t_{pd} &= 2t_{pd\_\text{AND}} + t_{pd\_\text{OR}} \\
      t_{cd} &= t_{cd\_\text{AND}}
    \end{align}
  };

  \node[text width=3.5cm](info) at ($(y)+(2,-.2)$) {
    See ARM book Figure-2.69 (pp90) for the timing diagrams of both paths
  };

  % \draw[color=cyan,line width=1pt] (ad1.out) -- +(.2,0) node[](inset1){};
  % % \draw[color=cyan,line width=1pt] (ad1.out) -- +(.5,0) -- (ad1.out |- or.in 1);
\end{circuitikz}

\begin{itemize}
\item LOW $\rightarrow$ HIGH is \emph{rising edge}; HIGH $\rightarrow$ LOW is \emph{falling edge}
\item circuits slowing down when hot and speeding up when cold
\item manufacturers provide data sheets showing delays for each gate
\item propagation delay $t_{pd} = \Sigma e$ where $e$ is an element along \textbf{critical} path
\item[] $t_{pd}$ is the min time from when an input changes until the output(s) reach their final value.
\item combinational delay $t_{cd} = \Sigma e$ where $g$ is an element long \textbf{short} path
\item[] $t_{cd}$ is the min time from when an input changes until any output starts changing
\item delays usually occur on order of $10^{-12}$(pico)/$10^{-9}$(nano) second
\end{itemize}

% \begin{tikzpicture}
  \tikzset{
    circle/.style={line width=.5pt,color=cyan!60,rounded corners=4pt},
  }

  % K-map Table
  \foreach \x in {0,0.5,1,1.5}{
    \draw (\x,0) rectangle (\x+0.5,0.5);
    \draw (\x,0.5) rectangle (\x+0.5,1);
  }

  \draw (0,1) -- +(-0.3,0.3) node[above]{\footnotesize $Y$};
  \draw (0,1) +(-0.1,0) node[left]{\footnotesize$C$};
  \draw (0,1) +(0.1,0.15) node[above]{\footnotesize$AB$};

  \node [left] at (0,0.25) {1};
  \node [left] at (0,0.75) {0};

  \node at (0.25,0.25) {1};
  \node at (0.75,0.25) {1};
  \node at (1.25,0.25) {1};
  \node at (1.75,0.25) {0};
  \node at (0.25,0.75) {1};
  \node at (0.75,0.75) {0};
  \node at (1.25,0.75) {0};
  \node at (1.75,0.75) {0};
  \node at (0.25,1.1) {\footnotesize 00};
  \node at (0.75,1.1) {\footnotesize 01};
  \node at (1.25,1.1) {\footnotesize 11};
  \node at (1.75,1.1) {\footnotesize 10};

  % Circles (prime implicants)
  \draw[circle] (0.05,0.05) rectangle (0.45,0.95);
  \draw[circle] (0.55,0.05) rectangle (1.45,0.45);

  % Boolean function
  \draw (0,-0.2) node[right]{\footnotesize$Y=\overline{A}\overline{B} + BC$};
  % Arrow
  \draw[cyan,-{Latex[length=1mm]}] (0.7,0.25) -- (0.3,0.25);
  % Text
  \draw (2,0.5) node[text width=2.2cm,right]{
    \footnotesize K-map A with possible glitch
  };


  % K-map without glitch
  \foreach \x in {5,5.5,6,6.5}{
    \draw (\x,0) rectangle (\x+0.5,0.5);
    \draw (\x,0.5) rectangle (\x+0.5,1);
  }

  \draw (5,1) -- +(-0.3,0.3) node[above]{\footnotesize $Y$};
  \draw (5,1) +(-0.1,0) node[left]{\footnotesize$C$};
  \draw (5,1) +(0.1,0.15) node[above]{\footnotesize$AB$};
  \node [left] at (5,0.25) {1};
  \node [left] at (5,0.75) {0};

  \node at (5.25,0.25) {1};
  \node at (5.75,0.25) {1};
  \node at (6.25,0.25) {1};
  \node at (6.75,0.25) {0};
  \node at (5.25,0.75) {1};
  \node at (5.75,0.75) {0};
  \node at (6.25,0.75) {0};
  \node at (6.75,0.75) {0};
  \node at (5.25,1.1) {\footnotesize 00};
  \node at (5.75,1.1) {\footnotesize 01};
  \node at (6.25,1.1) {\footnotesize 11};
  \node at (6.75,1.1) {\footnotesize 10};

  % Circles (prime implicants)
  \draw[circle] (5.05,0.05) rectangle (5.45,0.95);
  \draw[circle] (5.55,0.05) rectangle (6.45,0.45);
  \draw[circle,cyan] (5.05,0.05) rectangle (5.95,0.45);
  \draw[-{Latex[length=1.1mm]},cyan] (4.8,-0.1) node[left]{\footnotesize$\overline{A}C$}
  -- (5.1,0.1);
  \draw (4.8,-0.2) node[right]{\footnotesize$Y=\overline{A}\overline{B} + BC + \textcolor{cyan}{\overline{A}C}$};
  \draw (7,0.5) node[text width=2.2cm,right]{
    \footnotesize K-map B without glitch
  };
\end{tikzpicture}
\begin{itemize}
\item input change crosses the boundary of two prime implicants indicates possible glitch; not to eliminate them but to know they exist
\end{itemize}


% \pagebreak

\section*{Bistable (cross-coupled inverter pair)}
\begin{circuitikz}
  \ctikzset {
    logic ports=ieee,
    logic ports/scale=0.7,
  }

  % coupled-inverter pair
  \node [not port,anchor=in] (i2) at(.3,1) {I2} ;
  \node [not port,anchor=in] (i1) at($(i2.out)+(0.5,0)$) {I1};
  \draw (0,0) |- (i2.in);
  \draw (i2.out) -- +(.2,0) node[above] (qcomp) {$\overline{Q}$} -- (i1.in);
  \draw (i1.out) -- +(.2,0) node[above] (q) {$Q$};
  \draw (0,0) -| (q);

  \node[text width=3.5cm,right] at (0,1.8) {
    When power first applied, initial state is unknown
  };

  % cross-coupled inverter pair
  \node [not port,anchor=south west] (ib2) at(4.5,0) {I2};
  \node [not port,anchor=south west] (ib1) at(4.5,1.6) {I1};

  \node [left,blue,xshift=-.3,yshift=-.2] at (ib1.in) {1};
  \node [above,blue,xshift=.25] at (ib1.out) {0};
  \node [left,blue,xshift=-.3,yshift=.2] at (ib2.in) {0};
  \node [above,blue,xshift=.25] at (ib2.out) {1};

  \draw (ib2.in) -- ++(-.1,0) -- ++(0,.6) -- ($(ib1.out)+(.2,-.6)$) |- (ib1.out);
  \draw (ib2.out) -- ++(.2,0) -- +(0,.6) -- ($(ib1.in)+(-.1,-.6)$) |- (ib1.in);
  \draw (ib1.out) -- ++(.5,0) node[right]{$Q$};
  \draw (ib2.out) -- ++(.5,0) node[right]{$\overline{Q}$};

  % Cross-coupled inverter pair
  \node [not port,anchor=south west] (ic2) at(7.5,0) {I2};
  \node [not port,anchor=south west] (ic1) at(7.5,1.6) {I1};

  \node [left,blue,xshift=-.3,yshift=-.2] at (ic1.in) {0};
  \node [above,blue,xshift=.25] at (ic1.out) {1};
  \node [left,blue,xshift=-.3,yshift=.2] at (ic2.in) {1};
  \node [above,blue,xshift=.25] at (ic2.out) {0};

  \draw (ic2.in) -- ++(-.1,0) -- ++(0,.6) -- ($(ic1.out)+(.2,-.6)$) |- (ic1.out);
  \draw (ic2.out) -- ++(.2,0) -- +(0,.6) -- ($(ic1.in)+(-.1,-.6)$) |- (ic1.in);
  \draw (ic1.out) -- ++(.5,0) node[right]{$Q$};
  \draw (ic2.out) -- ++(.5,0) node[right]{$\overline{Q}$};
\end{circuitikz}
\section*{SR Latch (not synchronous)}

\section*{D Latch (transparent/level-sensitive)}
\begin{itemize}
\item Enable (maybe with a CLK) controls \emph{when} data flows through the latch
\item Enable determines whether or nor to block the circuit
\item Avoid gating Enable with Clock!
\item $E = 1$, latch is \textbf{transparent} and D flows to Q
\item $E = 0$, latch is \textbf{opaque} and blocks D from flowing to Q
\end{itemize}
\section*{D Flip Flop (synchronous)}
\begin{itemize}
\item it copies input \emph{D} to output \emph{Q} \textbf{only} on the rising edge of clock
\item it remembers its state at all other time
\item \emph{D} input specifies what the new state will be
\item clock edge indicates when the old should be updated
\end{itemize}
N-bit register needs a bank of N (D) flip-flops with a shared CLK.



\section*{Synchronous vs Asynchronous}
\begin{minipage}[h]{0.5\linewidth}
\begin{itemize}
\item no cyclic paths
\item stable (or bistable, metastable)
\item outputs settle after t$_{pd}$
\item use/insert registers to break cyclic path
\item states of registers \textbf{synchronized} to CLK; easier to design/analyze
\item virtually all digital systems are synchronous
\end{itemize}
\end{minipage}
\begin{minipage}[h]{0.5\linewidth}
\begin{itemize}
\item with cyclic paths
\item non-stable (astable)
\item infamous for race conditions, oscillations
\item Hard to analyze (depending on which path through logic gates is faster)
\item more general (only) in theory
\end{itemize}
\end{minipage}

\section*{Synchronous sequential circuits composition rule}
\begin{itemize}
\item Each circuit element is either a register or a combinational circuit (notation: C\kern-.4em\raisebox{-.3em}{L}, combinational logic)
\item At least one circuit element is a register
\item All registers receive the \textbf{same} clock signal
\item Every cyclic path contains least one register
\end{itemize}

\section*{Finite State Machine (FSM, Deterministic)}
All FSMs in circuit design courses are deterministic: \textbf{only one} path from current state to the next.
There should be no ambiguity.

\section*{Clock}
\begin{tabular}[h]{llll}
  Frequency & Clock Period & =? s & =? ns \\
  \hline
  1 Hz & 1 second & 1s & $10^{9}$ \\
  1 KHz & 1 millisecond (ms) & $10^{-3}$s & $10^6$ \\
  1 MHz & 1 microsecond ($\mu$s or us) &  $10^{-6}$s & $10^3$ \\
  1 GHz & 1 nanosecond (ns) & $10^{-9}$s & 1
\end{tabular}
\begin{itemize}
\item 1 Hz clock means 1 clock cycle per second
\item 100 MHz clock means $100\_000\_000$ cycles/s: how long per cycle?
  \[
    T_{cycle} = \frac{1}{\text{freq}} = \frac{1\text{s}}{(100 * 10^{6})} = 10^{-8}\text{s} = 10\text{ns}
  \]
  That is, 100 MHz clock has a clock period of 10ns or 1 clock cycle needs 10ns to complete
\item One clock period is the time taken to complete 1 clock cycle.
\item Given a clock cycle, the amount of time during which CLK is HIGH.
\item \textbf{Slower} clock means \textbf{longer} clock period and \textbf{smaller/lower} freq
\end{itemize}
Use an overflow counter to divide clock. Counter overflows and resets to zero automatically. The MSB connects to a LED port so it increases at the slowest speed as the counter increases.


\end{multicols*}
\end{document}
