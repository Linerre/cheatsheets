\documentclass[8pt,a4paper,landscape]{extarticle}
% -- Layout ----
\usepackage[top=0.6cm, bottom=0.6cm, left=0.5cm, right=0.5cm, landscape]{geometry}

% -- Titles ----
\usepackage[
  tiny,                     % text size title
  compact                   % reduce vertical space before/after title
]{titlesec}
% \titlespacing*
\titleformat{\section}{\normalfont\small\bfseries}{\thesection}{0em}{} % Remove space before and after section titles
\titleformat{\subsection}{\normalfont\small\bfseries}{\thesubsection}{0em}{} % Remove space before and after subsection titles
\titlespacing*{\section}{0pt}{0pt}{0pt} % Remove space before/after section titles
\titlespacing*{\subsection}{0pt}{0pt}{0pt} % Remove space before/after subsec titles

% -- Colors ----
\usepackage[dvipsnames]{xcolor}
\definecolor{dmm}{RGB}{192,192,192} % Define a custom dimmed text color
\definecolor{cmt}{RGB}{61,123,123}

% -- Math ------
\usepackage{mathtools}
\usepackage{amssymb}
\usepackage{turnstile}%better vdash

% -- Lists -----
\usepackage[inline]{enumitem}
\setlist{noitemsep}% Remove vspace between items
% Set vspace before and after  list environments as well as the left margin
\setlist[itemize,1]{leftmargin=.6em,labelindent=0pt,labelsep=2pt,
  topsep=1pt,partopsep=1pt}
\setlist[enumerate,1]{leftmargin=1em,labelindent=0pt,labelsep=2pt,
  topsep=1pt,partopsep=1pt}
\setlist[itemize,2]{leftmargin=.3em,labelindent=1pt,topsep=1pt,partopsep=1pt}
\setlist[enumerate,2]{leftmargin=0.2em,labelindent=1pt,topsep=1pt,partopsep=1pt}
\setlist[description]{labelwidth=\linewidth,font=\small\bfseries,leftmargin=1em,topsep=1pt,partopsep=1pt}
% -- Code listing ---
\usepackage{listings}
\lstset{
  aboveskip=3pt,
  belowskip=3pt,
  basicstyle=\small\ttfamily,
  breaklines=true,
  % commentstyle=\upshape\ttfamily,
  captionpos=b,
  commentstyle=\color{cmt},
  columns=flexible,
  frame=single,
  keepspaces=false,
  keywordstyle=\bfseries,
  showspaces=false,
  showstringspaces=false,
  showtabs=false,
  tabsize=2,
}

% Parse Trees
\usepackage{tikz}
\usetikzlibrary{ arrows, automata, bbox, calc, positioning, tikzmark, decorations.pathmorphing, decorations.pathreplacing, decorations.shapes, }
\tikzset{
  >=stealth',
  node distance=1cm,
  recstate/.style={
    circle,draw=blue!50,fill=blue!20,thick,font=\small\sffamily,rounded corners=3pt,
    minimum size=1cm,inner sep=1pt
  },
  ivp/.style={draw,->,auto,font=\small\sffamily,bend angle=60},
  msi/.style={draw=Brown,->,auto,font=\small\sffamily,bend angle=80},
  msibs/.style={draw=RoyalBlue,->,auto,font=\small\sffamily,bend angle=80},
  msinl/.style={font=\footnotesize\sffamily}, %msi node label
  every edge/.style={draw,auto,font=\small\sffamily},
  every loop/.style={looseness=4},
  initial text=start,initial where=right
}

% Place a figure env right here via [H] option
\usepackage{float}

% Side by side figure
\usepackage{subcaption}
% \usepackage{caption}
% \captionsetup{belowskip=0pt, aboveskip=0pt}
\usepackage{pifont}

% -- Multi-Col layout --
\usepackage{multicol}

% No indentation
\setlength\parindent{0pt}
\setlength\abovedisplayskip{-5pt}
\setlength\belowdisplayskip{-5pt}
\setlength\abovedisplayshortskip{-4pt}
\setlength\belowdisplayshortskip{-4pt}
\setlength\tabcolsep{5pt}
\newcommand{\gor}{\;|\;}
\newcommand{\num}{\texttt{\#}~}
\newcommand{\pro}[1]{\textcolor{Brown}{#1}}
\newcommand{\bus}[1]{\textcolor{RoyalBlue}{#1}}
\renewcommand{\arraystretch}{1.2}


\begin{document}
% Suppress page number for all pages
\pagestyle{empty}
% Math env spacing
\setlength{\abovedisplayskip}{1pt}
\setlength{\belowdisplayskip}{1pt}
\setlength{\abovedisplayshortskip}{0pt}
\setlength{\belowdisplayshortskip}{0pt}
% hl
\sethlcolor{dim}

% Notes begin
\begin{multicols*}{3}
\section*{Decimal $\iff$ Binary (Unsigned, U)}
% \begin{flalign*}
B2U(X) & = \displaystyle\sum_{i=0}^{w-1}x_{i}\cdot 2^{i}, w = \text{number of digits} \\
  10110_{2} & =  1 \times 2^{4} + 0 \times 2^{3} + 1 \times 2^{2} + 1 \times 2^{1} + 0 \times 2^{0}\\
  & = 22 &&
\end{flalign*}
\begin{enumerate}
\item\label{item:1} Divide the N$_{10}$ by 2
\item Get the \textbf{Q}uotient for the next iteration, and
\item Get the \textbf{R}mainder
\item Repeat the above steps until \textbf{Q} becomes 0 (1/2)
\item In the \emph{reverse} order, write all the \textbf{R}s.
\end{enumerate}
\begin{tabular}[h]{l|l|l}
   \hline
    Division by 2 & \textbf{Q}uotient & \textbf{R}emainder \\
    \hline
     333/2 $\Rightarrow$      & 166 & 1 \\
     166/2 $\Rightarrow$      & 83 & 0 \\
     83/2 $\Rightarrow$     & 41 & 1 \\
     41/2 $\Rightarrow$     & 20 & 1 \\
     20/2 $\Rightarrow$     & 10 & 0 \\
     10/2 $\Rightarrow$     & 5 & 0 \\
     5/2  $\Rightarrow$    & 2 & 1 \\
     2/2  $\Rightarrow$    & 1 & 0 \\
     1/2  $\Rightarrow$   & \textbf{0} & 1 \\
    \hline
    \multicolumn{3}{c}{333$_{10}$ = 101001101$_{2}$}\\
    \multicolumn{3}{c}{The above goes for Base$_{n}$ to Base$_{m}$}\\
    \hline
\end{tabular}
\section*{Hexadecimal (Memorize 10, 12, 14, 15)}
\begin{tabular}[h]{r|c|c}
  \hline
Decimal & Binary & Hex \\
  \hline
  0 & 0000 & 0 \\
  1 & 0001 & 1 \\
  2 & 0010 & 2 \\
  3 & 0011 & 3 \\
  4 & 0100 & 4 \\
  5 & 0101 & 5 \\
  6 & 0110 & 6 \\
  7 & 0111 & 7 \\
  8 & 1000 & 8 \\
  9 & 1001 & 9 \\
  10 & \textbf{1010} & \textbf{A} \\
  11 & 1011 & B \\
  12 & \textbf{1100} & \textbf{C} \\
  13 & 1101 & D \\
  14 & \textbf{1110} & \textbf{E} \\
  15 & 1111 & F \\
  \hline
\end{tabular}

\section*{Binary $\iff$ Hexadecimal}
As shown above, 1 hex digit = 4 binary digits
\begin{flalign*}
  2ED_{16} = \underbracket{0010}_{2_{16}}\overbracket{1110}^{E_{16}}\underbracket{1101}_{D_{16}} \quad(\text{Binary}) &&
\end{flalign*}
Binary-to-hex starts from right of the binary:
\begin{flalign*}
  (\text{Binary})\quad \overbracket{111}^{7_{16}}\underbracket{1010}_{A_{16}} = 7A_{16} &&
\end{flalign*}

\section*{Sign/Magnitude numbers}
\begin{equation*}
  \underbracket{0}_{\mathclap{\text{most significant bit as sign}}}101_{2} = 5_{10}\qquad\qquad\qquad\qquad 1\overbracket{101}_{2}^{\mathclap{\text{remaining } N-1 \text{ bits as magnitude (abs value)}}} = -5_{10}
\end{equation*}

\section*{Two's Complement (T)}
Most significant bit (MSB) indicates sign (same as above):
\begin{itemize}
\item 0 for non-negative (including zero)
\item 1 for negative
\end{itemize}
\begin{flalign*}
B2T(X) & = -x_{w-1}\cdot 2^{w-1} + \displaystyle\sum_{i=0}^{w-2}x_{i}\cdot 2^{i} \\
  10110_{2} & = 1 \times -(2^{4}) + 0 \times 2^{3} + 1 \times 2^{2} + 1 \times 2^{1} + 0 \times 2^{0}\\
       & = -16 + 0 + 4 + 2 + 0 \\
  & = -10_{10} &&
\end{flalign*}
\begin{tabular}[h]{c|c|c|c}
  \hline
  0 & -1 & most positive & most negative \\
  \hline
  $00\ldots 000_{2}$ & 11\ldots 111$_{2}$ & $01\ldots 111_{2} = 2^{N-1} - 1$ & $10\ldots 000_{2} = -2^{N-1}$ \\
  \hline
\end{tabular}
\section*{Taking the Two's Complement}
% Method One
\begin{minipage}{0.5\linewidth}
  Method A
\begin{enumerate}
\item\label{item:2} invert all the bits in the number
\item add 1 to the \emph{least significant} bit (LSB) position
\end{enumerate}
\end{minipage}
% Method Two
\begin{minipage}{0.5\linewidth}
  Method B
\begin{enumerate}
\item Check MSB for the sign (1-/0+)
\item Treat the binary number as unsigned and get its decimal value
\item Use value from above steps to mod $2^{w}$ until remainder $<2^{w}$
\end{enumerate}
\end{minipage}

\begin{minipage}{0.48\linewidth}
-2$_{10}$ as a 4-bit two's complement number
\begin{align*}
  2_{10} & = 0010_{2} \\
         & \xrightarrow{\text{invert all bits}} 1101_{2}\\
         & \xrightarrow{\text{add 1 to LSB}} 1110_{2} = -2_{10}
\end{align*}
\end{minipage}
\begin{minipage}{0.5\linewidth}
Find decimal value of two's compl. number $1001_{2}$
\begin{align*}
  1001_{2} & \Rightarrow{\text{has leading 1, must be negative}} \\
           & \xrightarrow{\text{invert all bits}} 0110_{2} \\
           & \xrightarrow{\text{add 1 to LSB}} 0111_{2} = -7_{10}
\end{align*}
\end{minipage}
% Addition
\begin{minipage}{0.5\linewidth}
\begin{align*}
  -2_{10} + 1_{10} & = 1110_{2} + 0001_{2} \\
                   & = 1111_{2} = -1_{10} \\
  -7_{10} + 7_{10} & = 1001_{2} + 0111_{2} \\
                   & = 10000_{2} \\
                   &\xrightarrow{\text{discard carry}} \dimtxt{1}0000_{2} \\
  & = 0_{10}
\end{align*}
\end{minipage}
% Subtraction
\begin{minipage}{0.5\linewidth}
\begin{align*}
  5_{10} - 3_{10} & = 5_{10} + (-3_{10}) = 0101_{2} + 1101_{2} \\
                  & = 10010_{2} \xrightarrow{\text{discard carry}} \dimtxt{1}0010_{2}\\
                  & = 2_{10} \\
  3_{10} - 5_{10} & = 3_{10} + (-5_{10}) = 0011_{2} + 1011_{2} \\
                  & = 1110_{2} \quad (\text{must be negative}) \\
                  & = -2_{10}
\end{align*}
\end{minipage}

\begin{itemize}
\item Adding 2 $N$-bit positive or negative numbers may cause overflow ($> 2^{N-1} \text{or} < -2^{N-1}$)
\item Adding a positive to a negative \emph{never} cause overflow
\item A carry out of most significant bit does \emph{not} indicate overflow
\item Overflow occurs when two numbers have the same sign bit and result has the opposite sign bit
\end{itemize}
\begin{tabular}[h]{l|l|l}
  \hline
  System & Min  & Max \\
  \hline
  Unsigned & 0 & $2^{N} - 1$ \\
  \hline
  Sign/Magnitude & $-2^{N-1} + 1$ & $2^{N-1} - 1$ \\
  \hline
  Two's Complement & $-2^{N-1}$ & $2^{N-1} - 1$ \\
  \hline
\end{tabular}

% Extension and truncation
\textbf{Sign extension}: Extend $w$-bit signed integer to \emph{w}+\emph{k}-bit integer with same value:
\begin{itemize}
\item Make $k$ copies of sign bit for signed integer
\[
X' = \underbracket{x_{w-1},\ldots,x_{w-1}}_{k\; \text{copies of MSB}},x_{w-1},x_{w-2},\ldots,x_{0}
\]
\item Prepend $k$ copies of zeros for unsigned integer
\[
X' = \underbracket{0,\ldots,0,}_{k\; \text{copies of }0}x_{w-1},x_{w-2},\ldots,x_{0}
\]
\end{itemize}

\begin{flalign*}
  3_{10} = 0011_{2} \quad\text{(4-bit)} \xrightarrow{\text{copy sign bit into 3 new upper bits}} 0000011_{2} \quad\text{(7-bit)} \\
  -3_{10} = 1101_{2} \quad\text{(4-bit)} \xrightarrow{\text{copy sign bit into 3 new upper bits}} 1111101_{2} \quad\text{(7-bit)} \\
\end{flalign*}
\textbf{Truncation} Truncate k+$w$-bit un-/signed integer $X$ to $w$-bit $X'$.
\begin{itemize}
\item For small un-/signed numbers, result remains the same (mod 2$^{w}$ until remainder $< 2^{w}$)
\begin{align*}
  2_{10}  & = 00010_{2} \xrightarrow[\text{truncate MSB}]{2 \modu 16} \dimtxt{0}0010_{2} = 2_{10} \\
  -6_{10} & = 11010_{2} \xrightarrow[\text{truncate MSB}]{-6 \modu 16 = 26\text{U} \modu 16 = 10\text{U} = -6 } \\
  & = \dimtxt{1}1010_{2} = -6_{10}\qquad
\end{align*}
\item For larger numbers, sign will change after truncation

  \begin{align*}
  10_{10} & = 01010_{2} \xrightarrow[\text{truncate MSB}]{2 \modu 16} \dimtxt{0}1010_{2} = -6_{10} \\
  -10_{10} & = 10110_{2} \xrightarrow[\text{truncate MSB}]{-10 \modu 16 = 22\text{U} \modu 16 = 6\text{U} = 6 } \\
  & = \dimtxt{1}0110_{2} = 6_{10}
\end{align*}
\end{itemize}

% \pagebreak

\section*{Combinational Digital Logic}
\begin{itemize}
\item lowest voltage = 0V, called \emph{ground} or GND (sometimes, V$_{SS}$)
\item highest voltage dec. from 5V (1970s-80s) to $\leqslant$1.2V, called V$_{DD}$
\item \emph{short circut} occurs when both pull-up and pull-down network ON
\item output \emph{floats} when both pull-up and pull-down network OFF
\end{itemize}
\begin{tabular}[h]{cllcl}
  \hline
  Gate & nMOS & pMOS & Pass well & Network  \\
  \hline
  0/LOW/GND & OFF & ON  & 0 &  pull-dow\textbf{n}\\
  1/HIGH/V$_{DD}$ & ON & OFF & 1 & pull-u\textbf{p}\\
  \hline
\end{tabular}

% Handy function to show the positions of nodes
\def\normalcoord(#1){coordinate(#1)}
\def\showcoord(#1){coordinate(#1) node[circle, red, draw, inner sep=1pt,
pin={[red, overlay, inner sep=0.5pt, font=\tiny, pin distance=0.1cm,
pin edge={red, overlay}]45:#1}](){}}
\let\coord=\normalcoord
\let\coord=\showcoord % coordinates markers
\def\killdepth#1{{\raisebox{0pt}[\height][0pt]{#1}}} % baseline tweaks

% Two-input NAND gate
\begin{minipage}{0.4\linewidth}
  \begin{circuitikz}
    [scale=0.7,
    transform shape,
    information text/.style={inner sep=1ex}]
  % Nodes
  \draw (2,1.5) node[pmos](p2){\large p2};
  \draw (3.5,1.5) node[pmos](p1){\large p1};
  \draw (3.5,0) node[nmos](n1){\large n1};
  \draw (3.5,-1.2) node[nmos](n2){\large n2} (3.5,-1.5) node[ground](GND){};
  \draw (0,0) node[left](A){A};
  \draw (0,-1.2) node[left](B){B};
  % Wires
  \draw (A) -- (n1.G);
  \draw (B) -- (n2.G);
  \draw (p1.G) -- (n1.G);
  % from p2.G to the intersection of two lines:
  % vertical line that passes p2.G and horizontal line that passes n2.G
  \draw (p2.G) -- (p2.G |- n2.G);
  % I/O
  \draw (p2.D) -- (p1.D) -- ++(0.5,0) node[circ]{} +(0.1,0) node[right](y){$Y$};
  \draw ($(p2.G |- p2.S)+(.7,0)$) -- (p2.S) -- +(1,0) node[above]{\large $V_{DD}$} -- (p1.S) -- +(0.3,0);

  % Caption
  \draw (GND) ($(GND) - (2,0)$) node[below, text width=2.5cm]
  {Two-input NAND gate schematic};
  % Information text, see https://tikz.dev/tutorial#sec-2.21
  % Whenever possible, use it
\end{circuitikz}
\end{minipage}
\begin{minipage}{0.55\linewidth}
  The nMOS transistors n1 and n2 are connected in series;
  The pMOS transistors p1 and p2 are in parallel.
  This acts as the base for multiple-input NAND gate schematic.
  For example, 4-input type would have 4 pMOS transistors in parallel
  and 4 nMOS transistors in series.
\end{minipage}
\begin{tabular}[h]{ccccc}
  \hline
  A & B & Pull-Dow\textbf{n} network & Pull-U\textbf{p} network & Y  \\
  \hline
  0 & 0 & OFF & ON  & 1 \\
  0 & 1 & OFF & ON  & 1 \\
  1 & 0 & OFF & ON  & 1 \\
  1 & 1 & ON & OFF  & 0 \\
  \hline
\end{tabular}

% Two-input NOR gate
\begin{minipage}{\linewidth}
\begin{circuitikz}
  [scale=0.7,
    transform shape,
    information text/.style={inner sep=2em}]
    % Input A is the origin
    \node at (0, 0) [left](A){A};
    \node at (0, -1.2) [left](B){B};
    \node at (3,0) [pmos](p1){\large p1};
    \node at (3,-1.2) [pmos](p2){\large p2};
    \node at (1.5,-2.7) [nmos](n1){\large n1};
    \node at (3,-2.7) [nmos](n2){\large n2};
    \node at (p1.S) [tground](vdd){};
    \node at ($(p1.S)+(0,.1)$) [above]{V$_{DD}$};
    \node at (n2.S) [sground](gnd){};
    \node at (gnd.east) [left, yshift=-.2cm, xshift=-.2cm, text width=2.5cm]{Two-input NOR Gate Schematic};

    \draw
    (A) -- (p1.G)
    (B) -- (p2.G)
    (p2.D) -- (n2.D)
    (n1.D) -- (n2.D) to[short, o-*] (n2.D) -- +(.5,0) node[right](Y){Y}
    (n1.S) -- (n2.S)
    (n1.G) -- (n1.G |- p1.G)
    (n2.G) -- (n2.G |- p2.G);

    % Text information
    \draw (Y.east) node[right, xshift=.3cm, text width=6cm, information text]
    {\LARGE The NOR gate should produce a 0 output if either input is 1.

      Hence, the pull-down network should have two nMOS transistors in parallel.

      By the conduction complements rule, the pMOS transistors must be in series.
    };

\end{circuitikz}
\end{minipage}

\end{multicols*}
\end{document}
